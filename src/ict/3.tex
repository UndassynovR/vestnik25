\id{МРНТИ 20.53.19}

{\bfseries ИДЕНТИФИКАЦИИ МАТЕМАТИЧЕСКИХ МОДЕЛЕЙ МЕТОДОМ КВАЗИЛИНЕАРИЗАЦИИ}

{\bfseries \textsuperscript{1,2}Т.Ж. Мазаков\textsuperscript{\envelope },
\textsuperscript{1}Ш.А. Джомартова, \textsuperscript{1}А.Т. Мазакова,
\textsuperscript{3}Г.Ч. Тойкенов,}

{\bfseries \textsuperscript{1,2}М.С. Алиаскар, \textsuperscript{3}У.Г.
Тойкенова}

\textsuperscript{1}Казахский национальный университет имени аль-Фараби,
Алматы, Казахстан,

\textsuperscript{2}Международный инженерно-технологический университет,
Алматы, Казахстан,

\textsuperscript{3}Казахский национальный женский педагогический
университет, Алматы, Казахстан

\raggedright {\bfseries \textsuperscript{\envelope }}Correspondent-author: \href{mailto:tmazakov@mail.ru}{\nolinkurl{tmazakov@mail.ru}}

Целью данной работы является определение параметров математической
модели, описывающей прорывы селевых потоков через гидротехнические
объекты, такие как дамбы и плотины. Достижение этой цели позволит
соотнести разработанную модель с реальными наблюдениями и данными.
Важность исследования обусловлена увеличением частоты и интенсивности
катастрофических наводнений, вызванных разрушением таких сооружений, что
подчеркивает необходимость разработки эффективных методов для
прогнозирования и предотвращения подобных инцидентов. Методы и подходы,
предложенные в данной работе, обладают высоким практическим значением
для оценки вероятных рисков, организации эвакуационных мероприятий и
снижения ущерба от аварий, связанных с разрушением дамб и плотин.

{\bfseries Ключевые слова:} гидротехнические сооружения (ГТС), параметры,
идентификация, прорыв дамбы, селевой поток, математическое
моделирование.

{\bfseries МАТЕМАТИКАЛЫҚ МОДЕЛЬДЕРДІ КВАЗИЛИНЕАРИЗАЦИЯ ӘДІСІМЕН
СӘЙКЕСТЕНДІРУ}

{\bfseries \textsuperscript{1,2}Т.Ж. Мазаков\textsuperscript{\envelope },
\textsuperscript{1}Ш.А. Джомартова, \textsuperscript{1}А.Т. Мазақова,
\textsuperscript{3}Г.Ч Тойкенов,}

{\bfseries \textsuperscript{1,2}М.С. Әлиасқар, \textsuperscript{3}Тойкенова
У.Г.}

¹Әл-Фараби атындағы Қазақ ұлттық университеті, Алматы, Қазақстан,\\
²Халықаралық инженерлік-технологиялық университет, Алматы, Қазақстан,\\
³Қазақ ұлттық қыздар педагогикалық университеті, Алматы, Қазақстан,\\
e-mail: \href{mailto:tmazakov@mail.ru}{\nolinkurl{tmazakov@mail.ru}}

Бұл зерттеудің мақсаты --- гидротехникалық құрылыстар, соның ішінде
бөгеттер мен дамбалар арқылы болатын сел ағындарының жарылуын
сипаттайтын математикалық модельдің параметрлерін анықтау. Бұл мақсатқа
қол жеткізу әзірленген модельді нақты бақылаулармен және деректермен
салыстыруға мүмкіндік береді. Зерттеудің өзектілігі осындай
құрылыстардың бұзылуы салдарынан болатын апатты су тасқындарының жиілігі
мен қарқындылығының артуымен ерекшеленеді, бұл осындай оқиғаларды болжау
және алдын алу үшін тиімді әдістерді әзірлеу қажеттігін көрсетеді. Осы
жұмыста ұсынылған әдістер мен тәсілдер ықтимал тәуекелдерді бағалауға,
эвакуация шараларын ұйымдастыруға және дамбалар мен бөгеттердің
бұзылуымен байланысты апаттардан келетін шығындарды азайтуға үлкен
практикалық мәнге ие.

{\bfseries Түйін сөздер}: гидротехникалық құрылыстар (ГТҚ), параметрлер,
сәйкестендіру, дамбаның жарылуы, сел ағыны, математикалық модельдеу.

{\bfseries IDENTIFICATION OF MATHEMATICAL MODELS USING THE
QUASILINEARIZATION METHOD}

{\bfseries \textsuperscript{1,2} T.Zh. Mazakov\textsuperscript{\envelope }, ¹S
\textsuperscript{1}h.A. Jomartova, \textsuperscript{1}A.T. Mazakova,
\textsuperscript{3}G.Ch. Toykenov,\\
\textsuperscript{1,2}M.S. Aliaskar, \textsuperscript{3}U.G. Toykenova\\
}

¹Al-Farabi Kazakh National University, Almaty, Kazakhstan,\\
²International Engineering and Technology University, Almaty,
Kazakhstan,\\
³Kazakh National Women' s Teacher Training University,
Almaty, Kazakhstan,\\
e-mail: \href{mailto:tmazakov@mail.ru}{\nolinkurl{tmazakov@mail.ru}}

The aim of this study is to determine the parameters of a mathematical
model that describes debris flow breaches through hydraulic structures,
such as dams and embankments. Achieving this goal will allow the
developed model to be correlated with real observations and data. The
importance of the research is underscored by the increasing frequency
and intensity of catastrophic floods caused by the failure of such
structures, emphasizing the need to develop effective methods for
predicting and preventing such incidents. The methods and approaches
proposed in this work hold significant practical value for assessing
potential risks, organizing evacuation measures, and reducing the damage
from accidents related to dam and embankment failures.

{\bfseries Keywords}: hydraulic structures (HS), parameters,
identification, dam breach, debris flow, mathematical modeling.

{\bfseries Введение}. За последнее столетие зарегистрировано более тысячи
разрушений гидротехнических сооружений по всему миру, где основными
причинами выступают как природные, так и антропогенные факторы. Среди
основных факторов аварий на ГТС {[}1{]} выделяются:

\begin{itemize}
\item
  прохождение экстремальных объемов воды, что может вызвать переполнение
  водохранилища и нарушить нормальную работу сбросных сооружений,
  приводя к переливу через гребень плотины и формированию прорыва;
\item
  износ основных компонентов плотин и гидромеханического оборудования
  вследствие длительной эксплуатации, что создает риск возникновения
  прорывов;
\item
  ошибки персонала, недостаток мониторинга опасных ситуаций и
  недостаточные прогнозные данные по паводкам;
\item
  террористические акты, направленные на разрушение плотин.
\end{itemize}

Примеры значимых происшествий:

\begin{enumerate}
\def\labelenumi{\arabic{enumi}.}
\item
  В Италии, в 1963 году, прорыв дамбы Вайонт был вызван оползнем объёмом
  около 260 млн куб. метров, который обрушился в водохранилище, вызвав
  перелив и уничтожение нескольких деревень, погибло около 2000 человек.
\item
  В Китае, прорыв дамбы Банкиао в 1975 году, спровоцированный
  чрезмерными осадками, привёл к гибели более 170 000 человек, а
  миллионы остались без жилья.
\item
  Аварии в Бразилии на хвостохранилищах горнодобывающих предприятий,
  включая прорыв дамбы в 2019 году в Брумадинью, вызвали масштабные
  экологические катастрофы и многочисленные жертвы.
\end{enumerate}

Кроме того, техногенная катастрофа на Саяно-Шушенской ГЭС в России,
произошедшая 17 августа 2009 года, привела к гибели 75 человек и
нанесению серьезного ущерба объекту. Прорыв дамбы в селе Кызылагаш
Алматинской области в 2010 году также привел к человеческим жертвам и
разрушениям.

{\bfseries Математические модели селевых потоков} служат важным
инструментом для анализа и прогнозирования этих сложных и разрушительных
природных явлений. Селевые потоки представляют собой массу воды, грунта
и камней, которая перемещается под действием гравитации по склонам и
руслам рек, требуя тщательного анализа и моделирования.

Основные подходы к моделированию селевых потоков:

\begin{enumerate}
\def\labelenumi{\arabic{enumi}.}
\item
  Эмпирические модели, опирающиеся на статистические данные прошлых
  событий.
\item
  Физические модели, воссоздаваемые в лабораторных условиях для изучения
  поведения селевых потоков.
\item
  Численные модели, использующие уравнения гидродинамики и механики
  грунтов.
\end{enumerate}

Применение математического моделирования позволяет:

\begin{enumerate}
\def\labelenumi{\arabic{enumi}.}
\item
  Оценивать риски и уязвимость:

  \begin{itemize}
  \item
    Определение зон потенциального воздействия селевых потоков.
  \item
    Использование при планировании территорий и разработке мер по
    снижению рисков.
  \end{itemize}
\item
  Проектировать инженерные сооружения:

  \begin{itemize}
  \item
    Поддержка при проектировании дамб, каналов и барьеров.
  \item
    Оценка эффективности защитных сооружений.
  \end{itemize}
\item
  Создавать системы раннего предупреждения:

  \begin{itemize}
  \item
    Интеграция моделей с метеорологическими данными для прогнозирования
    селевых потоков.
  \end{itemize}

  \begin{itemize}
  \item
    Обеспечение своевременного оповещения населения и служб
    реагирования.
  \end{itemize}
\end{enumerate}

Примеры математических моделей для анализа селевых потоков и оползней
{[}2-3{]}:

\emph{{\bfseries 1) Модель DEBRIS-2D}}

DEBRIS-2D --- это специализированная двумерная численная модель,
предназначенная для симуляции движения селевых потоков и грязекаменных
лавин по сложному рельефу местности. Она позволяет анализировать
динамику селевых масс с учётом их реологических характеристик и
взаимодействия с топографией. Модель учитывает различные реологические
подходы, такие как модели Бингема, Гершеля-Балкли и турбулентного
потока, что помогает описывать неньютоновские свойства селевых потоков,
включая зависимость вязкости от напряжения сдвига.\\
DEBRIS-2D использует высокоточные цифровые модели рельефа (DEM), что
обеспечивает детальное представление поверхности и точное
прогнозирование путей движения потоков. Модель решает двухмерные
уравнения глубинно-усредненного потока с использованием методов конечных
объемов или конечных разностей, что позволяет эффективно моделировать
нестационарные процессы. Она применяется для определения вероятных
маршрутов распространения селевых потоков и зон повышенного риска, а
также при планировании мер по снижению рисков, эвакуации и размещению
защитных сооружений.

\emph{{\bfseries 2) Модель FLO-2D}}

FLO-2D --- двумерная гидродинамическая модель, предназначенная для
моделирования поверхностного течения жидкости на обширных территориях.
Изначально разработанная для прогнозирования паводков, модель FLO-2D
была адаптирована для симуляции селевых потоков, лахаров и других
гидрологических явлений с учётом неньютоновских свойств. Модель решает
полные уравнения мелкой воды без упрощений, что позволяет учитывать
сложные потоки с переменными скоростями и направлениями. Она использует
квадратную расчётную сетку, что упрощает моделирование и повышает его
эффективность при обработке больших объемов данных.\\
FLO-2D поддерживает различные реологические модели для вязкопластичных
материалов, таких как селевые потоки, и учитывает влияние концентрации
частиц и вязкости на движение потока. Модель совместима с ГИС, что
облегчает подготовку данных и анализ результатов. FLO-2D широко
используется для прогнозирования распространения паводковых вод, оценки
риска затопления и моделирования высококонцентрированных смесей воды и
твёрдых частиц, характерных для селевых потоков. Она также применяется
при проектировании гидротехнических сооружений, таких как дамбы и
каналы.

\emph{{\bfseries 3) Модель DAN (Динамический анализ оползней)}}

DAN (Dynamic Analysis of Landslides) --- одномерная численная модель,
разработанная для анализа быстродвижущихся оползней, селевых потоков и
обвалов. Эта модель ориентирована на анализ глубинно-усредненного
движения массы вдоль заданной траектории с учётом изменения
реологических характеристик материала в процессе движения.\\
DAN поддерживает несколько реологических моделей, таких как:

\begin{itemize}
\item
  {\bfseries Модель сухого трения} --- используется для материалов, у
  которых сопротивление движению пропорционально нормальному напряжению;
\item
  {\bfseries Модель Бингема} --- учитывает начальное напряжение сдвига и
  постоянную вязкость;
\item
  {\bfseries Модель Гершеля-Балкли} --- обобщает модель Бингема, вводя
  показатель степени для напряжения сдвига;
\item
  {\bfseries Модель Воеллми} --- сочетает сухое трение и турбулентное
  сопротивление, часто применяемая для снежных лавин и обвалов.
\end{itemize}

Модель DAN позволяет изменять реологические параметры по траектории
движения, учитывая процессы разжижения или уплотнения материала. Она
помогает прогнозировать дальность, скорость и время прибытия оползневых
масс, а также оценивает потенциальный ущерб зданиям, дорогам и другим
объектам. Модель активно используется в планировании защитных инженерных
мер и разработке систем раннего предупреждения.

{\bfseries Материалы и методы.} Хотя все упомянутые модели описываются
уравнениями в частных производных, в работе {[}4{]} был предложен
алгоритм, позволяющий свести их к системе обыкновенных дифференциальных
уравнений. После того как структура математической модели определена
(т.е. задан вид уравнений с неуточнёнными параметрами), встает задача
параметрической идентификации --- нахождения численных значений
параметров на основе известных входных и выходных данных.

Ввиду сложности моделируемого объекта применимы лишь пассивные методы
идентификации, предполагающие использование данных, полученных в
процессе нормального функционирования объекта.

Рассмотрим следующую математическую модель, описываемую системой
обыкновенных дифференциальных уравнений:

% \begin{longtable}[]{@{}
%   >{\raggedright\arraybackslash}p{(\columnwidth - 2\tabcolsep) * \real{0.9470}}
%   >{\raggedright\arraybackslash}p{(\columnwidth - 2\tabcolsep) * \real{0.0530}}@{}}
% \toprule\noalign{}
% \begin{minipage}[b]{\linewidth}\raggedright
% \(\dot{y} = f(y,p,t)\),
% \end{minipage} & \begin{minipage}[b]{\linewidth}\raggedright
% (1)
% \end{minipage} \\
% \midrule\noalign{}
% \endhead
% \bottomrule\noalign{}
% \endlastfoot
% & \\
% \(y\left( t_{0} \right) = y_{0}\). & (2) \\
% \end{longtable}

Здесь \(y\) -- ny-мерный вектор состояния модели, p -- np-мерный вектор
параметров.

В качестве меры близости системы и модели выберем функционал:

% \begin{longtable}[]{@{}
%   >{\raggedright\arraybackslash}p{(\columnwidth - 2\tabcolsep) * \real{0.9470}}
%   >{\raggedright\arraybackslash}p{(\columnwidth - 2\tabcolsep) * \real{0.0530}}@{}}
% \toprule\noalign{}
% \begin{minipage}[b]{\linewidth}\raggedright
% \(S = \sum_{m = 1}^{ny}{\sum_{j = 1}^{l}{\sum_{i = 0}^{n}{({\widehat{y}}_{m}^{\left\langle j \right\rangle}\left( t_{i} \right) - y_{m}^{\left\langle j \right\rangle}\left( t_{i} \right))}^{2}}}\),
% \end{minipage} & \begin{minipage}[b]{\linewidth}\raggedright
% (3)
% \end{minipage} \\
% \midrule\noalign{}
% \endhead
% \bottomrule\noalign{}
% \endlastfoot
% \end{longtable}

где \(y_{m}^{\left\langle j \right\rangle}(t_{i})\) -- значение
выходного сигнала \(y_{m}\) в момент времени \(t_{i}\) в год j,
полученное с использованием математической модели.

Таким образом, задача параметрической идентификации сводится к задаче
минимизации функционала \emph{S} по параметрам \emph{p}:

% \begin{longtable}[]{@{}
%   >{\raggedright\arraybackslash}p{(\columnwidth - 2\tabcolsep) * \real{0.9470}}
%   >{\raggedright\arraybackslash}p{(\columnwidth - 2\tabcolsep) * \real{0.0530}}@{}}
% \toprule\noalign{}
% \begin{minipage}[b]{\linewidth}\raggedright
% \[\min_{p \in P}S\]
% \end{minipage} & \begin{minipage}[b]{\linewidth}\raggedright
% (4)
% \end{minipage} \\
% \midrule\noalign{}
% \endhead
% \bottomrule\noalign{}
% \endlastfoot
% \end{longtable}

где \emph{P} -- пространство допустимых значений параметров \emph{p}.

Метод квазилинеаризации, предложенный Беллманом и Калабой {[}5{]} для
решения краевых задач нелинейных дифференциальных уравнений, может быть
также использован для задачи параметрической идентификации при условии,
что параметры постоянны. Этот метод характеризуется высокой степенью
сходимости.

{\bfseries Результаты и обсуждение.} Дополним систему обыкновенных
дифференциальных уравнений (1)-(2) приобретает следующий вид с
добавлением уравнения:

% \begin{longtable}[]{@{}
%   >{\raggedright\arraybackslash}p{(\columnwidth - 2\tabcolsep) * \real{0.9470}}
%   >{\raggedright\arraybackslash}p{(\columnwidth - 2\tabcolsep) * \real{0.0530}}@{}}
% \toprule\noalign{}
% \begin{minipage}[b]{\linewidth}\raggedright
% \(\dot{p} = O\),
% \end{minipage} & \begin{minipage}[b]{\linewidth}\raggedright
% (5)
% \end{minipage} \\
% \midrule\noalign{}
% \endhead
% \bottomrule\noalign{}
% \endlastfoot
% \end{longtable}

где \(p\) -- параметры модели, подлежащие идентификации.

Введем обозначения: \(z = (y,\ p)\),
\(\varphi(z,t) = \left( f(z,t),\ O \right)\), где \(z\) - вектор
размерности -- \(nz = (ny + \ np\)\emph{).} Тогда систему
дифференциальных уравнений (1) и (5) можно записать как:

% \begin{longtable}[]{@{}
%   >{\raggedright\arraybackslash}p{(\columnwidth - 2\tabcolsep) * \real{0.9470}}
%   >{\raggedright\arraybackslash}p{(\columnwidth - 2\tabcolsep) * \real{0.0530}}@{}}
% \toprule\noalign{}
% \begin{minipage}[b]{\linewidth}\raggedright
% \[\dot{z} = \varphi\ (z,t)\]
% \end{minipage} & \begin{minipage}[b]{\linewidth}\raggedright
% (6)
% \end{minipage} \\
% \midrule\noalign{}
% \endhead
% \bottomrule\noalign{}
% \endlastfoot
% \end{longtable}

Предположим, что \emph{k}-я оценка вектора состояния
\(z^{\left\langle k \right\rangle}(t)\),
\(t \in \lbrack t_{0},t_{1}\rbrack\) известна. Разлагая правую часть (6)
в окрестности траектории \(z^{\left\langle k \right\rangle}(t)\), в ряд
Тейлора и ограничиваясь линейной частью, получаем систему
дифференциальных уравнений для \((k + 1)\)-го приближения вектора:

% \begin{longtable}[]{@{}
%   >{\raggedright\arraybackslash}p{(\columnwidth - 2\tabcolsep) * \real{0.9470}}
%   >{\raggedright\arraybackslash}p{(\columnwidth - 2\tabcolsep) * \real{0.0530}}@{}}
% \toprule\noalign{}
% \begin{minipage}[b]{\linewidth}\raggedright
% \[{\dot{z}}^{\left\langle k + 1 \right\rangle} = A^{\left\langle k \right\rangle}(t)z^{\left\langle k + 1 \right\rangle} + q^{\left\langle k \right\rangle}(t)\]
% \end{minipage} & \begin{minipage}[b]{\linewidth}\raggedright
% (7)
% \end{minipage} \\
% \midrule\noalign{}
% \endhead
% \bottomrule\noalign{}
% \endlastfoot
% \end{longtable}

где

\(A^{\left\langle k \right\rangle}(t) = \frac{\partial\varphi(z^{\left\langle k \right\rangle},t)}{\partial z}\),

\(q^{\left\langle k \right\rangle}(t) = \varphi\left( z^{\left\langle k \right\rangle},t \right) - A^{\left\langle k \right\rangle}(t)z^{\left\langle k \right\rangle}\).

Решение уравнения (7) имеет вид:

% \begin{longtable}[]{@{}
%   >{\raggedright\arraybackslash}p{(\columnwidth - 2\tabcolsep) * \real{0.9470}}
%   >{\raggedright\arraybackslash}p{(\columnwidth - 2\tabcolsep) * \real{0.0530}}@{}}
% \toprule\noalign{}
% \begin{minipage}[b]{\linewidth}\raggedright
% \(z^{\left\langle k + 1 \right\rangle}(t) = Ф^{\left\langle k + 1 \right\rangle}(t)z^{\left\langle k + 1 \right\rangle}\left( t_{0} \right) + g^{\left\langle k \right\rangle}(t)\),
% 
% \(t \in \left\lbrack t_{0},t_{1} \right\rbrack\),
% \end{minipage} & \begin{minipage}[b]{\linewidth}\raggedright
% (8)
% \end{minipage} \\
% \midrule\noalign{}
% \endhead
% \bottomrule\noalign{}
% \endlastfoot
% \end{longtable}

где \(Ф^{\left\langle k + 1 \right\rangle}(t)\) - является решением
следующей матричной системы дифференциальных уравнений:

% \begin{longtable}[]{@{}
%   >{\raggedright\arraybackslash}p{(\columnwidth - 2\tabcolsep) * \real{0.9470}}
%   >{\raggedright\arraybackslash}p{(\columnwidth - 2\tabcolsep) * \real{0.0530}}@{}}
% \toprule\noalign{}
% \begin{minipage}[b]{\linewidth}\raggedright
% \({\dot{Ф}}^{\left\langle k + 1 \right\rangle} = A^{\left\langle k \right\rangle}(t)Ф^{\left\langle k + 1 \right\rangle},\ \ \ \ \ \ \ \ Ф^{\left\langle k + 1 \right\rangle}\left( t_{0} \right) = E\),
% \end{minipage} & \begin{minipage}[b]{\linewidth}\raggedright
% (9)
% \end{minipage} \\
% \midrule\noalign{}
% \endhead
% \bottomrule\noalign{}
% \endlastfoot
% \end{longtable}

а \(g^{\left\langle k \right\rangle}(t)\) -- решением неоднородного
дифференциального уравнения:

% \begin{longtable}[]{@{}
%   >{\raggedright\arraybackslash}p{(\columnwidth - 2\tabcolsep) * \real{0.9341}}
%   >{\raggedright\arraybackslash}p{(\columnwidth - 2\tabcolsep) * \real{0.0659}}@{}}
% \toprule\noalign{}
% \begin{minipage}[b]{\linewidth}\raggedright
% \({\dot{g}}^{\left\langle k + 1 \right\rangle} = A^{\left\langle k \right\rangle}(t)g^{\left\langle k + 1 \right\rangle} + q^{\left\langle k \right\rangle},\ \ \ \ \ \ \ {g(t}_{0}) = O\).
% \end{minipage} & \begin{minipage}[b]{\linewidth}\raggedright
% (10)
% \end{minipage} \\
% \midrule\noalign{}
% \endhead
% \bottomrule\noalign{}
% \endlastfoot
% \end{longtable}

Здесь E -- единичная матрица.

Разобьем матрицу \(Ф^{\left\langle k + 1 \right\rangle}\) на блочные
матрицы:

\[Ф^{\left\langle k + 1 \right\rangle} = \begin{pmatrix}
Ф_{1}^{\left\langle k + 1 \right\rangle} & Ф_{2}^{\left\langle k + 1 \right\rangle} \\
Ф_{3}^{\left\langle k + 1 \right\rangle} & Ф_{4}^{\left\langle k + 1 \right\rangle}
\end{pmatrix}.\]

Подставим решение (8) в функционал (3):

% \begin{longtable}[]{@{}
%   >{\raggedright\arraybackslash}p{(\columnwidth - 2\tabcolsep) * \real{0.9341}}
%   >{\raggedright\arraybackslash}p{(\columnwidth - 2\tabcolsep) * \real{0.0659}}@{}}
% \toprule\noalign{}
% \begin{minipage}[b]{\linewidth}\raggedright
% \(S^{\left\langle k + 1 \right\rangle} = \ \sum_{m = 1}^{ny}{\sum_{j = 1}^{l}{\sum_{i = 0}^{n}\left( {\widehat{y}}_{m}^{\left\langle j \right\rangle}\left( t_{i} \right) - z_{m}^{\left\langle k + 1 \right\rangle\left\langle j \right\rangle}(t_{i}) \right)}}\),
% \end{minipage} & \begin{minipage}[b]{\linewidth}\raggedright
% (11)
% \end{minipage} \\
% \midrule\noalign{}
% \endhead
% \bottomrule\noalign{}
% \endlastfoot
% \end{longtable}

где
\(z_{m}^{\left\langle k + 1 \right\rangle\left\langle j \right\rangle}(t_{i})\)
-- значение \((k + 1)\)-го приближения выходного сигнала \(z_{m}\) в
момент времени \(t_{i}\), для года j. С учётом обозначений:

\(z_{m}^{\left\langle k + 1 \right\rangle\left\langle j \right\rangle}\left( t_{0} \right) = {\widehat{y}}_{m}^{\left\langle j \right\rangle}(t_{0})\),
для m=\(\overline{1,ny}\) и ∀ k, j,

\(z_{m}^{\left\langle k + 1 \right\rangle\left\langle j \right\rangle}\left( t_{0} \right) = p^{\left\langle k + 1 \right\rangle}\),
для m=\(\overline{ny + 1,nz}\) и ∀ k, j,

получим следующий функционал, зависящий от неизвестных
\(p^{\left\langle k + 1 \right\rangle}\):

% \begin{longtable}[]{@{}
%   >{\raggedright\arraybackslash}p{(\columnwidth - 2\tabcolsep) * \real{0.9341}}
%   >{\raggedright\arraybackslash}p{(\columnwidth - 2\tabcolsep) * \real{0.0659}}@{}}
% \toprule\noalign{}
% \begin{minipage}[b]{\linewidth}\raggedright
% \(S^{\left\langle k + 1 \right\rangle} = \sum_{j = 1}^{l}{\sum_{i = 0}^{n}{\left( {\widehat{y}}^{\left\langle j \right\rangle}\left( t_{i} \right) - Ф_{1}^{\left\langle k + 1 \right\rangle}\left( t_{i} \right){\widehat{y}}^{\left\langle j \right\rangle}\left( t_{0} \right) - Ф_{2}^{\left\langle k + 1 \right\rangle}\left( t_{i} \right)p^{\left\langle k + 1 \right\rangle} - g^{\left\langle k \right\rangle}(t_{i}) \right)\left( {\widehat{y}}^{\left\langle j \right\rangle}\left( t_{i} \right) - Ф_{1}^{\left\langle k + 1 \right\rangle}\left( t_{i} \right){\widehat{y}}^{\left\langle j \right\rangle}\left( t_{0} \right) - Ф_{2}^{\left\langle k + 1 \right\rangle}\left( t_{i} \right)p^{\left\langle k + 1 \right\rangle} - g^{\left\langle k \right\rangle}(t_{i}) \right)}}\).
% \end{minipage} & \begin{minipage}[b]{\linewidth}\raggedright
% (12)
% \end{minipage} \\
% \midrule\noalign{}
% \endhead
% \bottomrule\noalign{}
% \endlastfoot
% \end{longtable}

Приравняем к нулю частные производные от
\(S^{\left\langle k + 1 \right\rangle}\) по
\(p^{\left\langle k + 1 \right\rangle}\):

% \begin{longtable}[]{@{}
%   >{\raggedright\arraybackslash}p{(\columnwidth - 2\tabcolsep) * \real{0.9341}}
%   >{\raggedright\arraybackslash}p{(\columnwidth - 2\tabcolsep) * \real{0.0659}}@{}}
% \toprule\noalign{}
% \begin{minipage}[b]{\linewidth}\raggedright
% \[\sum_{j = 1}^{l}{\sum_{i = 0}^{n}{\left( Ф_{2}^{\left\langle k + 1 \right\rangle}\left( t_{i} \right)Ф_{2}^{\left\langle k + 1 \right\rangle}\left( t_{i} \right) \right)p^{\left\langle k + 1 \right\rangle}}} = \sum_{j = 1}^{l}{\sum_{i = 0}^{n}{Ф_{2}^{\left\langle k + 1 \right\rangle}(t_{i})}}\left( {\widehat{y}}^{\left\langle j \right\rangle}\left( t_{i} \right) - Ф_{1}^{\left\langle k + 1 \right\rangle}\left( t_{i} \right){\widehat{y}}^{\left\langle j \right\rangle}\left( t_{0} \right) - g^{\left\langle k \right\rangle}(t_{i}) \right)\]
% \end{minipage} & \begin{minipage}[b]{\linewidth}\raggedright
% (13)
% \end{minipage} \\
% \midrule\noalign{}
% \endhead
% \bottomrule\noalign{}
% \endlastfoot
% \end{longtable}

которую решим методом Гаусса для определения
\(p^{\left\langle k + 1 \right\rangle}\). Это позволит найти траекторию
\(z^{\left\langle k + 1 \right\rangle}(t)\), на интервале
\(t \in \lbrack t_{0},t_{1}\rbrack\), тем самым определяя \((k + 1)\)- ю
оценку вектора состояния \(z\). Процесс продолжается, пока очередные
оценки \(p^{\left\langle k \right\rangle}\) и
\(p^{\left\langle k + 1 \right\rangle}\) не станут близки по заданному
критерию.

Таким образом, алгоритм идентификации на основе квазилинеаризации можно
представить следующим образом:

Шаг 1. Задаем начальные значения \(p^{\left\langle 0 \right\rangle}\),
инициализируем \(k = 0\).

Шаг 2. Для определения \(k\)-го приближения
\(z^{\left\langle k \right\rangle}(t)\) интегрируем систему:

\[{\dot{z}}^{\left\langle k \right\rangle} = \varphi(z^{\left\langle k \right\rangle},u,t)\]

при начальных условиях

\(z_{m}^{\left\langle k \right\rangle}\left( t_{0} \right) = \frac{1}{l}\sum_{j = 1}^{l}{{\widehat{y}}_{m}^{\left\langle j \right\rangle}(t_{0})}\),
для m=\(\overline{1,ny}\),

\(z_{m}^{\left\langle k \right\rangle}\left( t_{0} \right) = p^{\left\langle k \right\rangle}\),
для m=\(\overline{ny + 1,nz}\).

Шаг 3. Решаем задачи Коши (9) и (10).

Шаг 4. Для получения \((k + 1)\)-й оценки параметров \emph{p} решаем
систему линейных алгебраических уравнений (13).

Шаг 5. Если
\(\left| p^{\left\langle k + 1 \right\rangle} - p^{\left\langle k \right\rangle} \right| < \varepsilon_{0}\),
процесс идентификации завершается; иначе --- возврат к шагу 2 (где
\(\varepsilon_{0}\) -- требуемая точность).

Таким образом, математические модели для анализа угроз прорыва дамб
обеспечивают безопасность гидротехнических объектов и позволяют:
прогнозировать последствия возможных аварий, разрабатывать меры для
снижения рисков и повышать осведомлённость населения к ЧС. Развитие
вычислительных технологий и интеграция с геоинформационными системами
делают модели более точными, хотя их результативность зависит от
качества исходных данных и опыта специалистов.

Предложенный алгоритм реализован программно и проходит экспериментальные
расчёты.

{\bfseries Выводы.} Модели DEBRIS-2D, FLO-2D и DAN представляют собой
ключевые инструменты в геотехническом и гидрологическом моделировании
опасных природных процессов. Их использование позволяет: а) углубленно
изучить механизмы движения селевых потоков и оползней; б) точно
прогнозировать зоны потенциального воздействия и пути распространения
опасных потоков; в) разрабатывать эффективные стратегии управления
рисками и планирования территорий. Выбор конкретной модели определяется
спецификой задачи, доступными данными и уровнем детализации, необходимым
для анализа. Интеграция результатов моделирования с практическими мерами
по снижению рисков способствует повышению уровня безопасности населения
и устойчивости инфраструктуры перед воздействием опасных геологических
процессов.

В данной работе разработана схема идентификации параметров моделей,
позволяющая адаптировать их к конкретным водоемам. Этот алгоритм
предоставляет возможность более точного моделирования процессов
разрушения плотин и распространения селевых волн, что является важным
аспектом для предотвращения катастрофических наводнений. Результаты
численных расчетов показывают, что предложенная схема эффективно
применяется для идентификации параметров выбранных математических
моделей. Программное обеспечение, реализующее данный алгоритм,
демонстрирует высокую точность и надежность расчетов, что подтверждается
сравнением с существующими моделями и экспериментальными данными.

Перспективы дальнейших исследований связаны с интеграцией разработанной
схемы идентификации параметров моделей с современными системами
мониторинга и предупреждения о чрезвычайных ситуациях. Это позволит
повысить эффективность мер по защите населения и инфраструктуры от
последствий селевых прорывов и других гидрологических угроз.

\emph{{\bfseries Финансирование.} Работа выполнена при финансовой поддержке
Комитета науки Министерства науки и высшего образования Республики
Казахстан в рамках гранта AP19678157 «Разработка программно-аппаратного
комплекса мониторинга состояния уровня заполняемости
водоёма»~(2023-2025)}

{\bfseries Литература}

1.Стихийные бедствия и техногенные катастрофы: Превентивные меры/ The
World Bank and The United Nations; пер. с англ.-М.: Альпина
Паблишер.-2011.-312 с.

ISBN 978-5-9614-1527-8

2.
\href{https://www.researchgate.net/profile/Francesca-Aureli?_tp=eyJjb250ZXh0Ijp7ImZpcnN0UGFnZSI6InB1YmxpY2F0aW9uIiwicGFnZSI6InB1YmxpY2F0aW9uIn19}{Francesca
Aureli},
\href{https://www.researchgate.net/profile/Paolo-Mignosa?_tp=eyJjb250ZXh0Ijp7ImZpcnN0UGFnZSI6InB1YmxpY2F0aW9uIiwicGFnZSI6InB1YmxpY2F0aW9uIn19}{Paolo
Mignosa},
\href{https://www.researchgate.net/profile/Massimo-Tomirotti?_tp=eyJjb250ZXh0Ijp7ImZpcnN0UGFnZSI6InB1YmxpY2F0aW9uIiwicGFnZSI6InB1YmxpY2F0aW9uIn19}{Massimo
Tomirotti}. Numerical simulation and experimental verification of
Dam-Break flows with
shocks//\href{https://www.researchgate.net/journal/Journal-of-Hydraulic-Research-1814-2079?_tp=eyJjb250ZXh0Ijp7ImZpcnN0UGFnZSI6InB1YmxpY2F0aW9uIiwicGFnZSI6InB1YmxpY2F0aW9uIn19}{Journal
of Hydraulic Research}.-2010.-Vol.2000(3).-P.197-206. DOI
\href{http://dx.doi.org/10.1080/00221680009498337}{10.1080/00221680009498337}

3.
\href{https://www.researchgate.net/profile/Tomaz-Podobnikar?_tp=eyJjb250ZXh0Ijp7ImZpcnN0UGFnZSI6InB1YmxpY2F0aW9uIiwicGFnZSI6InB1YmxpY2F0aW9uIn19}{Tomaž
Podobnikar} Methods for visual quality assessment of a digital terrain
model// S.A.P.I.EN.S
\href{https://www.researchgate.net/journal/Surveys-and-Perspectives-Integrating-Environment-and-Society-1993-3819?_tp=eyJjb250ZXh0Ijp7ImZpcnN0UGFnZSI6InB1YmxpY2F0aW9uIiwicGFnZSI6InB1YmxpY2F0aW9uIn19}{Surveys
and Perspectives Integrating Environment and Society}, Special issue
/-2009.-Vol.2(2).-P.1-10. URL
http://journals.openedition.org/sapiens/738

4. Mazakova A.T., Jomartova Sh.A., Mazakov T.Zh., Ziyatbekova G.Z.,
Begaliyeva K.B.

Digital Determination of the Thermal Conductivity of a Square Bar by
Reduction to a System of Integral Equations//Ecological Footprint of the
Modern Economy and the ways to Reduce it Advances in Science,
Technology\&amp;Innovation.-2024.-P.167-171.

DOI:10.1007/978-3-031-49711-7\_29

5. Гроп Д. Методы идентификации систем.// Перевод с англ. В.А.
Васильева, В.И. Лопатина. - Под ред. Е.И. Кринецкого М.: Мир, 1979. --
302 с.

{\bfseries References}

1.Stihijnye bedstvija i tehnogennye katastrofy: Preventivnye mery/ The
World Bank and The United Nations; per. s angl.-M.:
Al' pina Pablisher.-2011.-312 s.

ISBN 978-5-9614-1527-8. {[}in Russian{]}

2.
\href{https://www.researchgate.net/profile/Francesca-Aureli?_tp=eyJjb250ZXh0Ijp7ImZpcnN0UGFnZSI6InB1YmxpY2F0aW9uIiwicGFnZSI6InB1YmxpY2F0aW9uIn19}{Francesca
Aureli},
\href{https://www.researchgate.net/profile/Paolo-Mignosa?_tp=eyJjb250ZXh0Ijp7ImZpcnN0UGFnZSI6InB1YmxpY2F0aW9uIiwicGFnZSI6InB1YmxpY2F0aW9uIn19}{Paolo
Mignosa},
\href{https://www.researchgate.net/profile/Massimo-Tomirotti?_tp=eyJjb250ZXh0Ijp7ImZpcnN0UGFnZSI6InB1YmxpY2F0aW9uIiwicGFnZSI6InB1YmxpY2F0aW9uIn19}{Massimo
Tomirotti}. Numerical simulation and experimental verification of
Dam-Break flows with
shocks//\href{https://www.researchgate.net/journal/Journal-of-Hydraulic-Research-1814-2079?_tp=eyJjb250ZXh0Ijp7ImZpcnN0UGFnZSI6InB1YmxpY2F0aW9uIiwicGFnZSI6InB1YmxpY2F0aW9uIn19}{Journal
of Hydraulic Research}.-2010.-Vol.2000(3).-P.197-206. DOI
\href{http://dx.doi.org/10.1080/00221680009498337}{10.1080/00221680009498337}

3.
\href{https://www.researchgate.net/profile/Tomaz-Podobnikar?_tp=eyJjb250ZXh0Ijp7ImZpcnN0UGFnZSI6InB1YmxpY2F0aW9uIiwicGFnZSI6InB1YmxpY2F0aW9uIn19}{Tomaž
Podobnikar} Methods for visual quality assessment of a digital terrain
model// S.A.P.I.EN.S
\href{https://www.researchgate.net/journal/Surveys-and-Perspectives-Integrating-Environment-and-Society-1993-3819?_tp=eyJjb250ZXh0Ijp7ImZpcnN0UGFnZSI6InB1YmxpY2F0aW9uIiwicGFnZSI6InB1YmxpY2F0aW9uIn19}{Surveys
and Perspectives Integrating Environment and Society}, Special issue
/-2009.-Vol.2(2).-P.1-10. URL
http://journals.openedition.org/sapiens/738

4. Mazakova A.T., Jomartova Sh.A., Mazakov T.Zh., Ziyatbekova G.Z.,
Begaliyeva K.B.

Digital Determination of the Thermal Conductivity of a Square Bar by
Reduction to a System of Integral Equations//Ecological Footprint of the
Modern Economy and the ways to Reduce it Advances in Science,
Technology\&amp;Innovation.-2024.-P.167-171.

DOI:10.1007/978-3-031-49711-7\_29

5. Grop D. Metody identifikacii sistem.// Perevod s angl. V.A.
Vasil' eva, V.I. Lopatina. - Pod red. E.I. Krineckogo M.:
Mir, 1979.- 302 s. {[}in Russian{]}

\emph{{\bfseries Сведение об авторах}}

Мазаков Т.Ж.-доктор физико-математических наук, профессор, Казахский
национальный университетим. аль-Фараби, Алматы, Казахстан, e-mail:
tmazakov@mail.ru;

Джомартова Ш.А. - доктор технических наук, доцент, Казахский
национальный университет им. аль-Фараби, Алматы, Казахстан, e-mail:
jomartova@mail.ru;

Мазакова А.Т. -- докторант Казахского национального университета
им.аль-Фараби, e-mail:
\href{mailto:aigerym97@mail.ru}{\nolinkurl{aigerym97@mail.ru}};

Тойкенов Г.Ч. - кандидат физико-математических наук, доцент, Казахский
национальный женский педагогический университет, Алматы, Казахстан,
e-mail: gumyrbektoike@mail.ru;

Алиаскар М.С. - докторант Казахского национального университета
им.аль-Фараби, Алматы, Казахстан, e-mail: m.alyasqar@gmail.ru;

Тойкенова У.Г.- докторант Казахского национального женского
педагогического университета, Алматы, Казахстан, e-mail:
ulpantoikenova@gmail.com

\emph{{\bfseries Information about the authors}}

Mazakov T.Zh. -- Doctor of Physical and mathematical sciences,
professor, Al-Farabi Kazakh National University, Almaty, Kazakhstan,
e-mail: tmazakov@mail.ru;

Jomartova Sh.A. - Doctor of Technical Sciences, Associate Professor,
Al-Farabi Kazakh National University,Almaty, Kazakhstan, e-mail:
jomartova@mail.ru;

Mazakova A.T. - PhD student of the Al-Farabi Kazakh National University,
Almaty, Kazakhstan, e-mail:
\href{mailto:aigerym97@mail.ru}{\nolinkurl{aigerym97@mail.ru}};

Tokenov G.Ch. - Candidate of Physical and Mathematical Sciences,
Associate Professor, Kazakh National Women' s Pedagogical
University, Almaty, Kazakhstan, e-mail: gumyrbektoike@mail.ru;

M.S. Aliaskar - PhD student of the Al-Farabi Kazakh National University,
Almaty, Kazakhstan, e-mail:
\href{mailto:m.alyasqar@gmail.ru}{\nolinkurl{m.alyasqar@gmail.ru}};

Toikenova U.G. - PhD student of the Kazakh National
Women' s Pedagogical University, Almaty, Kazakhstan,
e-mail: ulpantoikenova@gmail.com
