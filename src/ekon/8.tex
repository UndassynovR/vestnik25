\id{МРНТИ 06.71.07}{https://doi.org/10.58805/kazutb.v.4.25-624}

\begin{articleheader}
    \sectionwithauthors{S. Saginova, D. Saparova, V. Stukach}{CURRENT STATE OF AGRICULTURAL BUSINESSES IN THE REPUBLIC OF
KAZAKHSTAN AND THEIR DEVELOPMENT TRENDS IN MODERN CONDITIONS}

{\bfseries \textsuperscript{1}S. Saginova\textsuperscript{\envelope },
\textsuperscript{2}D. Saparova, \textsuperscript{3}V. Stukach}
\end{articleheader}
\begin{affiliation}

\textsuperscript{1}K. Kulazhanov Kazakh University of Technology and
Business, Astana, Kazakhstan,

\textsuperscript{2}Turan-Astana University, Astana, Kazakhstan,

\textsuperscript{3}Omsk State Agrarian University Named After P.A.
Stolypin, Omsk, Russia

\raggedright{\bfseries \textsuperscript{\envelope }}Corresponding author:
\href{mailto:saginova.s@gmail.com}{\nolinkurl{saginova.s@gmail.com}}
\end{affiliation}

The purpose of this article is to study the current state and prospects
of development of agricultural formations in the Republic of Kazakhstan.
The work sets short-term objectives aimed at analysing the current state
of agricultural production, investment activity, crop and livestock
production, as well as identifying the main factors affecting their
development. The long-term objectives of the study are to forecast
future trends, develop strategies to improve the sustainability and
efficiency of the agro-industrial complex, and form recommendations for
improving management decisions. This approach allows taking into account
both existing problems and prospects for the transformation of the
industry, including the introduction of innovative technologies and the
transition to sustainable forms of management.

Methodology used to analyze the development of agricultural businesses
in the Republic of Kazakhstan covers a comprehensive approach that
includes both quantitative and qualitative methods of analysis, as well
as a wide range of information sources to ensure reliability and
objectivity of the results.

This work is based on econometric calculations stemming from the
analysis of economic indicators affecting development of agricultural
businesses with various forms of ownership in the Republic of
Kazakhstan. Forecast analysis of their future development trends will
improve planning in agro-industrial complex companies and enterprises,
as well as in the process of making management decisions to improve the
situation in agricultural sectors. The results obtained can be used both
to exchange data and for the process of making management decisions, as
well as to provide recommendations to improve the situation in the
agro-industrial complex of the Republic of Kazakhstan.

{\bfseries Keywords:} agricultural policy, agriculture, investments,
agricultural business, trends, trend model, means of production.
\begin{articleheader}

{\bfseries ҚАЗАҚСТАН РЕСПУБЛИКАСЫНДАҒЫ АУЫЛ ШАРУАШЫЛЫҒЫ ҚҰРЫЛЫМДАРЫНЫҢ
ҚАЗІРГІ ЖАҒДАЙЫ ЖӘНЕ ОЛАРДЫҢ СОҢҒЫ ЖАҒДАЙДАҒЫ ДАМУ ТЕНДЕНЦИЯЛАРЫ}

{\bfseries \textsuperscript{1}С.А. Сагинова\textsuperscript{\envelope },
\textsuperscript{2}Д.А. Сапарова, \textsuperscript{3} В.Ф. Стукач}
\end{articleheader}
\begin{affiliation}

\textsuperscript{1}Қ. Құлажанов атындағы Қазақ технология және бизнес
университеті, Астана, Казахстан,

\textsuperscript{2}Тұран-Астана университеті, Астана, Қазахстан,

\textsuperscript{3}П.А. Столыпин атындағы Омбы мемлекеттік аграрлық
университеті, Омбы, Ресей,

e-mail:
\href{mailto:saginova.s@gmail.com}{\nolinkurl{saginova.s@gmail.com}}
\end{affiliation}

Бұл мақаланың мақсаты -- Қазақстан Республикасындағы ауылшаруашылық
құрылымдарының қазіргі жағдайы мен даму болашағын зерттеу. Жұмыс аясында
ауыл шаруашылығы өндірісінің, инвестициялық белсенділіктің, өсімдік
шаруашылығы мен мал шаруашылығының ағымдағы жағдайын талдауға, сондай-ақ
олардың дамуына әсер ететін негізгі факторларды анықтауға бағытталған
қысқа мерзімді міндеттер белгіленді. Зерттеудің ұзақ мерзімді мақсаттары
болашақ тенденцияларды болжау, аграрлық сектордың тұрақтылығы мен
тиімділігін арттыру стратегияларын әзірлеу және басқару шешімдерін
жетілдіру бойынша ұсыныстарды қалыптастыру болып табылады. Бұл тәсіл
инновациялық технологияларды енгізуді және экономиканы басқарудың
тұрақты нысандарына көшуді қоса алғанда, саланы қайта құрудың бар
проблемаларын да, перспективаларын да ескеруге мүмкіндік береді.

Қазақстан Республикасында ауыл шаруашылығы кәсіпорындарының дамуын
талдау үшін қолданылатын әдістеме сандық және сапалық талдау әдістерін,
сондай-ақ нәтижелердің сенімділігі мен объективтілігін қамтамасыз ету
үшін ақпарат көздерінің кең спектрін қамтитын кешенді тәсілді қамтиды.

Бұл жұмыс Қазақстан Республикасында меншіктің әртүрлі нысандары бар
ауылшаруашылық кәсіпорындарының дамуына әсер ететін экономикалық
көрсеткіштерді талдаудан туындайтын эконометрикалық есептеулерге
негізделген. Олардың болашақтағы даму тенденцияларын болжамды талдау
агроөнеркәсіптік кешендегі компаниялар мен кәсіпорындарда, сондай-ақ
агроөнеркәсіптік кешендегі жағдайды жақсарту бойынша басқарушылық
шешімдер қабылдау процесінде жоспарлауды жақсартуға мүмкіндік береді.
Алынған нәтижелер деректер алмасу үшін де, басқару шешімдерін қабылдау
процесі үшін де. Сондай-ақ Қазақстан Республикасының агроөнеркәсіптік
кешеніндегі жағдайды жақсарту бойынша ұсыныстар беру үшін де
пайдаланылуы мүмкін.

{\bfseries Түйін сөздер:} аграрлық саясат, ауыл шаруашылығы, инвестициялар,
аграрлық формация, тенденциялар, тренд моделі, өндіріс құралдары.
\begin{articleheader}

{\bfseries НЫНЕШНЕЕ  СОСТОЯНИЕ СЕЛЬХОЗФОРМИРОВАНИЙ В РЕСПУБЛИКЕ КАЗАХСТАН И
ТЕНДЕНЦИИ их РАЗВИТИЯ в современных условиях}

{\bfseries \textsuperscript{1}С.А. Сагинова\textsuperscript{\envelope },
\textsuperscript{2}Д.А. Сапарова, \textsuperscript{3}В.Ф. Стукач}
\end{articleheader}
\begin{affiliation}

\textsuperscript{1}Казахский университет технологии и бизнеса им. К.
Кулажанова, Астана, Казахстан,

\textsuperscript{2}Университет «Туран-Астана», Астана, Казахстан,

\textsuperscript{3}Омский государственный аграный университет им. П.А.
Столыпина, Омск, Россия,

e-mail: \href{mailto:saginova.s@gmail.com}{\nolinkurl{saginova.s@gmail.com}}
\end{affiliation}

Целью данной статьи является исследование современного состояния и
перспектив развития сельскохозяйственных формирований в Республике
Казахстан на основе прогнозов. В работе проводится анализ основных
аспектов сельского хозяйства, таких как производство, инвестиции,
развитие растениеводства и животноводства, и рассматриваются перспективы
их трансформации в современных условиях.

Применяемая методология для анализа развития сельскохозяйственных
формирований в Республике Казахстан охватывает комплексный подход,
который включает в себя как количественные, так и качественные методы
анализа, а также широкий спектр информационных источников для
обеспечения достоверности и объективности получаемых результатов.

Базой данной работы являются эконометрические вычисления на основе
анализа экономических показателей, влияющих на развитие
сельскохозяйственных формирований с различными формами собственности в
Республике Казахстан. Прогнозный анализ будущих трендов их развития
позволит совершенствовать планирование в организациях и предприятиях
АПК, а также в процессе принятия управленческих решений по улучшению
ситуаций в отраслях сельского хозяйства. Полученные результаты могут
быть использованы для обмена информацией и в процессе принятия
управленческих решений, а также для предоставления рекомендаций по
улучшению ситуации в отраслях АПК Республики Казахстан.

{\bfseries Ключевые слова:} аграрная политика, сельское хозяйство,
инвестиции, сельскохозяйственное формирование, тенденции, трендовая
модель, средства производства.
\begin{multicols}{2}

{\bfseries Introduction.} Kazakhstan's state agricultural policy
proactively engages with agricultural producers to develop an
agro-industrial complex. Proper communication between producers of
various forms of ownership is essential if we are to avoid gaps in
providing basic production resources, which, if implemented properly,
helps to reduce costs and ensure efficient delivery of products to the
consumer's table, ultimately contributing to the balanced development of
an entire agricultural sector and agro-industrial complex altogether
{[}1{]}.

However, ill-considered actions to quickly replace methods and processes
following innovations and global trends can make an already difficult
situation in industries even worse. Accordingly, we require a
comprehensive analysis, we need to involve both domestic and
international scientific experts to transform agro-industrial complex
industries into sustainable production (using the principles of a green
economy, circular economy, bioeconomy, and other new economic models).
Furthermore, the best international practices in these areas might prove
beneficial as well. In this context, it is useful to calculate
fundamental base of analytical and forecast data, including, but not
limited to, mathematical, statistical, and econometric models.

Statistics over the last decades show a period of intense integration,
redistribution of property, changes in land use, and restructuring of
production the agro-industrial sector had to undergo. This has led
abandoning activities requiring a significant amount of labor and
resource costs.

In the process of natural agriculture reform, affected by external and
internal factors, areas intended for agricultural crops suffered a
partial reduction. The rapid decrease in potential is due to poor
resource management, negative environmental footprint, and insufficient
modernization of equipment and production processes.

Another negative consequence lowering export potential of our
agricultural products is non-compliance with international quality
standards in general and standards of ``clean'' organic products in
particular. It is no secret that presently the world pays great
attention to the food product contents. Should certain goods meet the
parameters of organic products, oftentimes their supply volumes are not
enough for exporting. Meaning, conditions for generating large volumes
of high-quality agricultural products are still to be created.

One of the points that need urgent changes is creation of effective
agribusinesses for further scaling and intensification of production
{[}2{]}.

In light of this, studying issues and prospects for the development of
an agro-industrial complex in the Republic of Kazakhstan requires an
in-depth analysis of the activities of agricultural enterprises. This
will allow disclosure of hidden reserves and their use in new types of
activities and methods that will be used to organize labor in the
country's agricultural sector.

The international experience shows that transitioning to a sustainable
development in the long term helps to significantly improve the quality
of natural resources, agricultural raw materials, and finished products
of the agricultural sector, as well as increase labor productivity. A
preliminary regional analysis of the agro-industrial complex's
development problems and agricultural formations of the country is
required if we are to choose the most suitable and most effective way to
address them {[}3{]}.

{\bfseries Materials and Methods.} In order to analyse the prospects for
the development of the agro-industrial complex of the Republic of
Kazakhstan, this study has chosen a trend model, in particular the
linear trend model, which allows to effectively approximate historical
data and forecast future changes in key agricultural factors. The choice
of this method is due to several circumstances, including stability and
predictability of changes in the agricultural sphere, availability of
data and ease of interpretation of results, which makes the linear trend
an optimal tool for solving the set tasks. In the following, the
rationale for the choice of this model is given, as well as its
applicability in the context of analysing the factors affecting the
development of the agro-industrial complex of Kazakhstan.

The choice of the trend model, in particular the linear trend model, for
forecasting the development of factors of the agro-industrial complex
(AIC) of the Republic of Kazakhstan is justified by a number of factors
that reflect the specific conditions of agriculture and the data
available for analysis.

Firstly, the linear trend allows us to effectively approximate data
showing stable, predictable changes, such as long-term fluctuations in
yields, production volumes and prices characteristic of the agricultural
sector. In an agricultural economy where changes are gradual and subject
to certain patterns, a linear model makes it possible to forecast these
changes based on historical data with a reasonably high accuracy.

Secondly, the linear model is easy to apply and interpret, which is
especially important for making operational decisions in conditions of
limited data, which are often present in the agricultural sector. This
approach allows not only to make forecasts based on available
statistics, but also to assess long-term development trends without the
need for complex computational resources.

Third, the linear model works well when factors of change - such as the
economic situation, natural conditions or changes in government policy -
do not fluctuate significantly and abruptly, which is often the case in
Kazakhstan' s agriculture. Although more sophisticated
models can offer greater accuracy in the face of variable and unstable
factors, the linear trend represents the best means for an initial
assessment of the prospects for the development of the agro-industrial
complex on the basis of available data. Thus, the trend model, and in
particular the linear trend, is a reliable tool for building forecasts
in the agricultural sector, providing the necessary results with minimal
data requirements and complex calculations, which is especially
important when resources for more sophisticated methods are limited.

Kazakhstan's agro-industrial sector is presented by the following main
forms of management: large agricultural enterprises, medium-sized
farms/peasant farms, and small personal subsidiary farms. Large farms
are registered as legal entities while farms, in terms of organizational
and legal form, are individual entrepreneurships not bearing legal
entity status. Individual entrepreneurs or peasant farms cultivate
approximately 30\% of agricultural land. Farm households vary in scale
and can be large, medium, or small. Larger farms are most common up
north where the land is more than 5,000 hectares. Accordingly, medium
and small farms are mainly concentrated down south. Medium-sized farms
in the southern regions can vary between 3 and 500 hectares making about
a third of their area. Although we have excluded personal subsidiary
farms as a form of management from the point of view of economic
organization, they remain significant producers of agricultural
products, especially livestock. As a rule, personal subsidiary
households (hereinafter referred to as PSF) are small home farms keeping
one to three cows, sheep, and goats, sometimes poultry. They may also
keep a small vegetable garden between several hundred square meters and
0.25-1 ha.

Just as elsewhere in the world, in Kazakhstan's modern conditions, the
role of limited liability partnerships, agriholdings, etc. is showing a
growing trend. ``Green'' clusters' role is increasing just as well
{[}4{]}. Investing in Kazakhstan's agricultural sector will contribute
to development of small and medium-sized businesses in livestock and
crop production sectors. However, being competitive means producers must
go organic and switch over to ``green'' technologies {[}5{]}.

Over the past five years, agriculture has been enjoying significant
investments amounting to approx. 1.7 trillion tenge in subsidies. This
includes 486.7 billion tenge for livestock farming, 366.8 billion tenge
for crop production, and 729.3 billion tenge as financial instruments.

Table 1 shows Kazakhstan's main economic statistics for the past period
of time. Studying its trend allows us to get an idea of the
organizational form development in our agriculture.
\end{multicols}



\begin{longtable}[H]{|@{\,}%
  >{\raggedright\arraybackslash}p{(\columnwidth - 10\tabcolsep) * \real{0.2611}}|%
  >{\raggedright\arraybackslash}p{(\columnwidth - 10\tabcolsep) * \real{0.1382}}|%
  >{\raggedright\arraybackslash}p{(\columnwidth - 10\tabcolsep) * \real{0.1382}}|%
  >{\raggedright\arraybackslash}p{(\columnwidth - 10\tabcolsep) * \real{0.1382}}|%
  >{\raggedright\arraybackslash}p{(\columnwidth - 10\tabcolsep) * \real{0.1350}}|%
  >{\raggedright\arraybackslash}p{(\columnwidth - 10\tabcolsep) * \real{0.1894}}|@{\,}}
  \caption*{Table 1 - Main Indicators of the Republic of Kazakhstan's
  Agro-Industrial Complex Between 2020 and 2023}\\

  \hline
\textbf{Indicators} & \textbf{2020} & \textbf{2021} & \textbf{2022} & \textbf{2023} & \textbf{2023 to 2022 Change, Per Cent} \\
\hline
\endfirsthead
\hline
\textbf{Indicators} & \textbf{2020} & \textbf{2021} & \textbf{2022} & \textbf{2023} & \textbf{2023 to 2022 Change, Per Cent} \\
\hline
\endhead
\hline
\endfoot
\endlastfoot
Gross Output of Agricultural Products (Services), Billion Tenge & 6,364.0 & 7,549.8 & 9,521.0 & 7,625.2 & −19.93 \\
\hline
Gross Livestock Production in the RK, Million Tenge & 3,687,310.3 & 4,387,236.5 & 5,808,259.8 & 7,218,965.5 & 24.28 \\
\hline
Region-Wise, Million Tenge & & & & & \\
\hline
Akmola & 202,790.8 & 264,476.3 & 321,137.9 & 358,160.61 & 11.54 \\
\hline
Aktobe & 202,120.1 & 242,888.8 & 259,290.5 & 292,278.92 & 12.73 \\
\hline
Almaty & 430,331.6 & 475,467.6 & 376,980.4 & 444,495.00 & 17.92 \\
\hline
Atyrau & 48,764.2 & 70,076.1 & 81,324.7 & 89,457.86 & 10.00 \\
\hline
West Kazakhstan & 127,066.5 & 146,379.9 & 167,183.6 & 185,025.57 & 10.67 \\
\hline
Zhambyl & 161,919.3 & 175,418.4 & 215,045.4 & 229,674.79 & 6.80 \\
\hline
Karaganda & 215,670.5 & 275,683.8 & 225,605.3 & 279,172.76 & 23.75 \\
\hline
Kostanay & 160,750.3 & 188,280.0 & 207,066.0 & 231,400.73 & 11.77 \\
\hline
Kyzylorda & 52,888.5 & 60,882.3 & 68,282.2 & 75,249.63 & 10.19 \\
\hline
Mangystau & 15,417.8 & 18,057.3 & 23,208.8 & 24,105.43 & 3.86 \\
\hline
Pavlodar & 147,999.6 & 167,474.1 & 203,273.3 & 216,632.56 & 6.56 \\
\hline
North Kazakhstan & 178,087.7 & 219,863.2 & 263,024.8 & 280,949.95 & 6.80 \\
\hline
Turkestan & 304,785.1 & 362,230.9 & 401,717.6 & 434,228.12 & 8.09 \\
\hline
East Kazakhstan & 365,954.5 & 421,154.3 & 255,844.4 & 338,722.86 & 32.36 \\
\hline
The City of Astana & 151.7 & 148.9 & 157.8 & 140.26 & −11.13 \\
\hline
The City of Almaty & 826.3 & 658.0 & 705.4 & 267.25 & −62.12 \\
\hline
The City of Shymkent & 21,936.2 & 27,833.8 & 26,120.8 & 30,659.71 & 17.37 \\
\hline
& & & & & \\
\hline
Gross Crop Production in the RK, Million Tenge & 2,637,460.7 & 3,116,973.5 & 3,658,757.6 & 4,227,405.9 & 15.53 \\
\hline
& & & & & \\
\hline
Akmola & 468,740.5 & 475,525.0 & 770,299.9 & 808,414.53 & 4.95 \\
\hline
Aktobe & 123,040.1 & 132,008.1 & 196,735.8 & 204,241.09 & 3.81 \\
\hline
Almaty & 531,894.2 & 610,353.1 & 391,848.6 & 528,890.79 & 34.97 \\
\hline
Atyrau & 36,286.8 & 42,241.8 & 52,902.8 & 56,891.53 & 7.53 \\
\hline
West Kazakhstan & 69,650.8 & 94,765.2 & 131,714.4 & 141,130.49 & 7.15 \\
\hline
Zhambyl & 229,015.6 & 302,261.7 & 363,509.1 & 410,694.51 & 12.98 \\
\hline
Karaganda & 167,721.0 & 217,338.7 & 254,301.4 & 283,819.49 & 11.61 \\
\hline
Kostanay & 430,972.8 & 415,585.4 & 811,647.3 & 810,008.53 & −0.20 \\
\hline
Kyzylorda & 89,524.2 & 108,578.3 & 117,693.8 & 133,345.34 & 13.29 \\
\hline
Mangystau & 3,579.7 & 3,465.0 & 5,136.4 & 4,941.05 & −3.81 \\
\hline
Pavlodar & 154,089.4 & 260,633.1 & 314,082.3 & 356,741.13 & 13.60 \\
\hline
North Kazakhstan & 598,313.9 & 679,297.0 & 909,326.2 & 995,276.45 & 9.45 \\
\hline
Turkestan & 438,023.1 & 567,578.9 & 648,470.7 & 738,890.48 & 13.96 \\
\hline
East Kazakhstan & 325,022.8 & 454,045.6 & 288,634.0 & 407,674.73 & 41.28 \\
\hline
The City of Astana & 345.4 & 354.2 & 412.3 & 352.34 & −14.54 \\
\hline
The City of Almaty & 6,662.9 & 7,077.6 & 4,338.8 & 5,816.01 & 34.03 \\
\hline
The City of Shymkent & 14,426.9 & 16,127.9 & 17,572.4 & 17,259.90 & −1.78 \\
\hline
Fixed Investment, Billion Tenge & 12,270.1 & 13,242.2 & 15,251.1 & 17,649.3 & 15.72 \\
\hline
Registered Legal Entities of the RK by Economic Sectors (Agriculture, Forestry, and Fisheries), Units & 18,843 & 19,991 & 20,327 & 20,990 & 3.26 \\
\hline
\multicolumn{6}{|@{}>{\raggedright\arraybackslash}p{(\columnwidth - 10\tabcolsep) * \real{1.0000} + 10\tabcolsep}@{\,}|}{%
Note: Compiled by the authors based on data from the Committee on Statistics of the Ministry of National Economy of the Republic of Kazakhstan} \\
\hline
\end{longtable}




Table 1 shows that in 2023 compared to 2022, gross livestock output rose
by 24.3\% and gross crop output rose by 15.5\%. However, there is a
reduction in gross livestock production in rural areas of Akmola and
Almaty regions and a reduction in gross crop production in Akmola and
Shymkent regions.


\begin{longtable}[H]{|@{\,}%
  >{\raggedright\arraybackslash}p{(\columnwidth - 8\tabcolsep) * \real{0.3729}}|%
  >{\raggedright\arraybackslash}p{(\columnwidth - 8\tabcolsep) * \real{0.1505}}|%
  >{\raggedright\arraybackslash}p{(\columnwidth - 8\tabcolsep) * \real{0.1506}}|%
  >{\raggedright\arraybackslash}p{(\columnwidth - 8\tabcolsep) * \real{0.1350}}|%
  >{\raggedright\arraybackslash}p{(\columnwidth - 8\tabcolsep) * \real{0.1911}}|@{\,}}
  \caption*{Table 2 - Region Shares in the Total Gross Livestock Output
  Between 2021 and 2023}\\

  \hline
\textbf{Indicators} & \textbf{2021} & \textbf{2022} & \textbf{2023} & \textbf{2023 to 2021 Ratio, Per Cent} \\
\hline
\endfirsthead
\hline
\textbf{Indicators} & \textbf{2021} & \textbf{2022} & \textbf{2023} & \textbf{2023 to 2021 Ratio, Per Cent} \\
\hline
\endhead
\hline
\endfoot
\endlastfoot
Gross Livestock Output in the RK, Million Tenge & 2,411,486.7 & 3,687,310.3 & 5,808,259.8 & 58.48 \\
\hline
Region-Wise, Million Tenge & & & & \\
\hline
Akmola & 264,476.3 & 321,137.9 & 358,160.61 & 26.16 \\
\hline
Aktobe & 242,888.8 & 259,290.5 & 292,278.92 & 16.90 \\
\hline
Almaty & 475,467.6 & 376,980.4 & 444,495.00 & −6.97 \\
\hline
Atyrau & 70,076.1 & 81,324.7 & 89,457.86 & 21.67 \\
\hline
West Kazakhstan & 146,379.9 & 167,183.6 & 185,025.57 & 20.89 \\
\hline
Zhambyl & 175,418.4 & 215,045.4 & 229,674.79 & 23.62 \\
\hline
Karaganda & 275,683.8 & 225,605.3 & 279,172.76 & 1.25 \\
\hline
Kostanay & 188,280.0 & 207,066.0 & 231,400.73 & 18.63 \\
\hline
Kyzylorda & 60,882.3 & 68,282.2 & 75,249.63 & 19.09 \\
\hline
Mangystau & 18,057.3 & 23,208.8 & 24,105.43 & 25.09 \\
\hline
Pavlodar & 167,474.1 & 203,273.3 & 216,632.56 & 22.69 \\
\hline
North Kazakhstan & 219,863.2 & 263,024.8 & 280,949.95 & 21.74 \\
\hline
Turkestan & 362,230.9 & 401,717.6 & 434,228.12 & 16.58 \\
\hline
East Kazakhstan & 421,154.3 & 255,844.4 & 338,722.86 & −24.34 \\
\hline
The City of Astana & 148.9 & 157.8 & 140.26 & −6.16 \\
\hline
The City of Almaty & 658.0 & 705.4 & 267.25 & −146.21 \\
\hline
The City of Shymkent & 27,833.8 & 26,120.8 & 30,659.71 & 9.22 \\
\hline
\end{longtable}



Gross livestock output has shown changes in regional specific weights.
Case in point, 2023 compared to previous years (2021 and 2022), as
evidenced by the data in Table 2.



\begin{longtable}[H]{|@{\,}%
  >{\raggedright\arraybackslash}p{(\columnwidth - 8\tabcolsep) * \real{0.3646}}|%
  >{\raggedright\arraybackslash}p{(\columnwidth - 8\tabcolsep) * \real{0.1509}}|%
  >{\raggedright\arraybackslash}p{(\columnwidth - 8\tabcolsep) * \real{0.1509}}|%
  >{\raggedright\arraybackslash}p{(\columnwidth - 8\tabcolsep) * \real{0.1451}}|%
  >{\raggedright\arraybackslash}p{(\columnwidth - 8\tabcolsep) * \real{0.1885}}|@{\,}}
  \caption*{Table 3 - Region Shares in the Total Gross Crop Production
  Output Between 2021 and 2023}\\

  \hline
\textbf{Indicators} & \textbf{2021} & \textbf{2022} & \textbf{2023} & \textbf{2023 to 2021 Ratio, Per Cent} \\
\hline
\endfirsthead
\hline
\textbf{Indicators} & \textbf{2021} & \textbf{2022} & \textbf{2023} & \textbf{2023 to 2021 Ratio, Per Cent} \\
\hline
\endhead
\hline
\endfoot
\hline
\endlastfoot
Gross Crop Production Output in the RK, Million Tenge & 2,050,455.8 & 2,637,460.7 & 3,658,757.6 & 43.96 \\
\hline
Region-Wise, Million Tenge & & & & \\
\hline
Akmola & 475,525.0 & 770,299.9 & 808,414.53 & 41.18 \\
\hline
Aktobe & 132,008.1 & 196,735.8 & 204,241.09 & 35.37 \\
\hline
Almaty & 610,353.1 & 391,848.6 & 528,890.79 & −15.40 \\
\hline
Atyrau & 42,241.8 & 52,902.8 & 56,891.53 & 25.75 \\
\hline
West Kazakhstan & 94,765.2 & 131,714.4 & 141,130.49 & 32.85 \\
\hline
Zhambyl & 302,261.7 & 363,509.1 & 410,694.51 & 26.40 \\
\hline
Karaganda & 217,338.7 & 254,301.4 & 283,819.49 & 23.42 \\
\hline
Kostanay & 415,585.4 & 811,647.3 & 810,008.53 & 48.69 \\
\hline
Kyzylorda & 108,578.3 & 117,693.8 & 133,345.34 & 18.57 \\
\hline
Mangystau & 3,465.0 & 5,136.4 & 4,941.05 & 29.87 \\
\hline
Pavlodar & 260,633.1 & 314,082.3 & 356,741.13 & 26.94 \\
\hline
North Kazakhstan & 679,297.0 & 909,326.2 & 995,276.45 & 31.75 \\
\hline
Turkestan & 567,578.9 & 648,470.7 & 738,890.48 & 23.18 \\
\hline
East Kazakhstan & 454,045.6 & 288,634.0 & 407,674.73 & −11.37 \\
\hline
The City of Astana & 354.2 & 412.3 & 352.34 & −0.53 \\
\hline
The City of Almaty & 7,077.6 & 4,338.8 & 5,816.01 & −21.69 \\
\hline
The City of Shymkent & 16,127.9 & 17,572.4 & 17,259.90 & 6.56 \\
\hline
\end{longtable}

\begin{multicols}{2}


Table 3 data analysis also shows a reduction in the share of gross crop
production in the total volume in 2023 compared to previous years. This
affected heavily the Almaty, Zhambyl, and East Kazakhstan regions
(please refer to Table).

Table 1 shows an increase in the total number of large and medium-sized
farms that are legal entities, ultimately reaching 20,090 units in 2023,
which makes 4\% of the total number of registered enterprises and legal
entities operating in other sectors.

The growth in gross output, investment volume, and the number of
registered legal entities indicate a clear progress. These changes are
likely the result of integrated efforts in agriculture and increased
investor interest in the agro-industrial sector. Additionally, state
support can play an important role in these economic indicators' growth
since over the specified period, agricultural subsidies from the state
budget grew to reach 408.7 billion tenge in 2022 (226.2 billion tenge in
2018, 356.3 billion tenge in 2019, 384.8 billion tenge in 2020, and 450
billion tenge in 2021).

Let us analyze evolution of key quantitative indicators of legal entity
development in the Republic of Kazakhstan in the agricultural, forestry,
and fishery sector. We shall look at various types of ownership, too.
Based on the statistical data presented in Table 4, this will allow us
to understand development trends of agricultural structures in the
country and their contribution to the economy and food security.
\end{multicols}

\begin{longtable}[H]{|@{\,}%
  >{\raggedright\arraybackslash}p{(\columnwidth - 14\tabcolsep) * \real{0.2467}}|%
  >{\raggedright\arraybackslash}p{(\columnwidth - 14\tabcolsep) * \real{0.1068}}|%
  >{\raggedright\arraybackslash}p{(\columnwidth - 14\tabcolsep) * \real{0.1064}}|%
  >{\raggedright\arraybackslash}p{(\columnwidth - 14\tabcolsep) * \real{0.1064}}|%
  >{\raggedright\arraybackslash}p{(\columnwidth - 14\tabcolsep) * \real{0.1064}}|%
  >{\raggedright\arraybackslash}p{(\columnwidth - 14\tabcolsep) * \real{0.1086}}|%
  >{\raggedright\arraybackslash}p{(\columnwidth - 14\tabcolsep) * \real{0.0996}}|%
  >{\raggedright\arraybackslash}p{(\columnwidth - 14\tabcolsep) * \real{0.1191}}|@{\,}}
  \caption*{Table 4 - The number of the RK's Active Legal Entities of
  Various Types of Ownership by Economic Sectors (Agriculture, Forestry,
  and Fisheries) for the Period between 2018 and 2023}\\

  \hline
\textbf{Indicators} & \textbf{2018} & \textbf{2019} & \textbf{2020} & \textbf{2021} & \textbf{2022} & \textbf{2023} & \textbf{2023 in Per Cent to 2018} \\
\hline
\endfirsthead
\hline
\textbf{Indicators} & \textbf{2018} & \textbf{2019} & \textbf{2020} & \textbf{2021} & \textbf{2022} & \textbf{2023} & \textbf{2023 in Per Cent to 2018} \\
\hline
\endhead
\hline
\endfoot
\endlastfoot
Registered, privately owned, legal entities by sectors, units & 17,007 & 17,582 & 18,497 & 19,632 & 19,935 & 20,720 & 121.8 \\
\hline
Active, state owned, legal entities by sectors, units & 72 & 72 & 65 & 67 & 63 & 61 & 84.7 \\
\hline
Active, foreign owned, legal entities, units & 236 & 263 & 281 & 292 & 314 & 332 & 140.6 \\
\hline
Active small and medium-sized businesses, peasant, or farm enterprises, units & 231,312 & 252,264 & 260,781 & 261,071 & 275,776 & 285,561 & 123.4 \\
\hline
\multicolumn{8}{|@{}>{\raggedright\arraybackslash}p{(\columnwidth - 14\tabcolsep) * \real{1.0000} + 14\tabcolsep}|@{}}{%
Note: Compiled by the authors based on data from the Committee on Statistics of the Ministry of National Economy of the Republic of Kazakhstan} \\
\hline
\end{longtable}


\begin{multicols}{2}

The above table shows an increase in the number of registered
agribusinesses for all forms of ownership, except for the state-owned
ones.

Account must be taken of the fact that registered, private-owned, legal
entities in the Republic of Kazakhstan's agricultural sector include
agricultural cooperatives. There were 3,284 of those at the end of 2022,
employing over 7.3 thousand people {[}6{]}.

Quantitative forecasting analysis that uses a structured trend method
based on actual economic growth data will show promising areas for the
development of agribusinesses {[}7, 8{]}. In this case, trend
forecasting method was used in an attempt to calculate the trend in
changes in the number of Kazakhstan's agribusinesses until 2026.

We use statistical time series to analyze dynamics of economic
phenomena. Levels of these series are determined by various factors
affecting both long-term and short-term, including random effects.
Changes in the conditions for the development of these phenomena are
reflected in the level of phenomena in question over time {[}9, 10{]}.

To demonstrate application of the extrapolation method, we shall use
time series data reflecting the activities of legal entities operating
in agriculture, forestry, and fisheries in the Republic of Kazakhstan
for the period between 2018 and 2023 as shown in Table 4.

To start with, we determine parameters of the equation calculated by the
least squares method.

For the calculated data, the system of equations is as follows:

\[\left\{ \begin{array}{r}
6a + 21b = 113373 \\
21a + 91b = 410185
\end{array} \right.\ \]

From the first equation, we express a and substitute it into the second
equation. For the result, we get a = 764.543, b = 16,219.6.

Calculation table 5 shows quality assessment values of the equation
parameters.
\end{multicols}


\begin{longtable}[H]{|@{\,}%
  >{\raggedright\arraybackslash}p{(\columnwidth - 10\tabcolsep) * \real{0.1665}}|%
  >{\raggedright\arraybackslash}p{(\columnwidth - 10\tabcolsep) * \real{0.1665}}|%
  >{\raggedright\arraybackslash}p{(\columnwidth - 10\tabcolsep) * \real{0.1666}}|%
  >{\raggedright\arraybackslash}p{(\columnwidth - 10\tabcolsep) * \real{0.1668}}|%
  >{\raggedright\arraybackslash}p{(\columnwidth - 10\tabcolsep) * \real{0.1668}}|%
  >{\raggedright\arraybackslash}p{(\columnwidth - 10\tabcolsep) * \real{0.1666}}|@{\,}}
  \caption*{Table 5- Calculated quality assessment of the equation
  parameters}\\

  \hline
\textbf{t} & \textbf{y} & \textbf{y(t)} & \textbf{(yi-ycp)²} & \textbf{(yi-y(t))²} & \textbf{(t-tp)²} \\
\hline
\endfirsthead
\hline
\textbf{t} & \textbf{y} & \textbf{y(t)} & \textbf{(yi-ycp)²} & \textbf{(yi-y(t))²} & \textbf{(t-tp)²} \\
\hline
\endhead
\hline
\endfoot
\endlastfoot
1 & 17,007 & 16,984.143 & 3,566,432.25 & 522.449 & 6.25 \\
\hline
2 & 17,582 & 16,984.143 & 1,725,282.25 & 27,784.127 & 2.25 \\
\hline
3 & 18,497 & 18,513.229 & 158,802.25 & 263.367 & 0.25 \\
\hline
4 & 19,632 & 19,277.771 & 542,432.25 & 125,477.881 & 0.25 \\
\hline
5 & 19,935 & 20,042.314 & 1,080,560.25 & 11,516.356 & 2.25 \\
\hline
6 & 20,720 & 20,806.857 & 3,328,800.25 & 7,544.163 & 6.25 \\
\hline
 & & 113,373 & 10,402,309.5 & 173,108.343 & 17.5 \\
\hline
\multicolumn{6}{|@{}>{\raggedright\arraybackslash}p{(\columnwidth - 10\tabcolsep) * \real{1.0000} + 10\tabcolsep}|@{}}{%
Note: Compiled by the authors based on calculations} \\
\hline
\end{longtable}

\begin{multicols}{2}


Let us perform an evaluation check of the accuracy of the trend model
equation's calculated parameters and test the hypotheses regarding

\[R^{2} = 1 - \frac{\sum_{}^{}\left( y_{i} - y_{t} \right)^{2}}{\sum_{}^{}\left( y_{i} - \overline{y} \right)^{2}} = 0.9834\]

The calculated value characterizes 98.34\% of cases of
\emph{t}-influence on the change of the result factor \emph{y}.
Otherwise speaking, trend equation modeling accuracy is high.

{\bfseries Results and Discussions.} The study analyzed the relationship
between the indicator ``Number of Registered Legal Entities of the
Republic of Kazakhstan by Sectors of the Economy with Private
Ownership'' and the time factor. In the process of defining the model,
we selected a linear trend and analyzed its parameters using the least
squares method.

According to the research results, it was found that 98.34\% of the
total variability of the factor indicator value is associated with a
change in the time parameter. In addition, it was found that the model
parameters have statistical significance. It is possible to carry out an
economic interpretation of these parameters, which shows that the
average increase in the factor indicator value is 764.543 units with
each change in the time factor.

The model used for forecasting based on selected key factors with R²
probability levels clearly shows that if current development trends are
maintained, the forecast value is consistent with the calculated value
of the identified dynamics of change in indicators (Table 6).
\end{multicols}


\begin{longtable}[H]{|@{\,}%
  >{\raggedright\arraybackslash}p{(\columnwidth - 6\tabcolsep) * \real{0.5607}}|%
  >{\raggedright\arraybackslash}p{(\columnwidth - 6\tabcolsep) * \real{0.1518}}|%
  >{\raggedright\arraybackslash}p{(\columnwidth - 6\tabcolsep) * \real{0.1516}}|%
  >{\raggedright\arraybackslash}p{(\columnwidth - 6\tabcolsep) * \real{0.1359}}|@{\,}}
  \caption*{Table 6 - Forecast Values of the Indicators ``Number of
  Operating Legal Entities of Various Types of Ownership in the Republic
  of Kazakhstan by Economic Sectors (Agriculture, Forestry, and
  Fisheries)'' for the Period Between 2024 and 2026}\\

  \hline
\textbf{Indicators} & \textbf{2024} & \textbf{2025} & \textbf{2026} \\
\hline
\endfirsthead
\hline
\textbf{Indicators} & \textbf{2024} & \textbf{2025} & \textbf{2026} \\
\hline
\endhead
\hline
\endfoot
\endlastfoot
Registered, privately owned, legal entities by sectors, units & 21,571 & 22,336 & 23,100 \\
\hline
Active, state owned, legal entities by sectors, units & 59 & 56 & 54 \\
\hline
Active, foreign owned, legal entities, units & 351 & 369 & 388 \\
\hline
Active small and medium-sized businesses, peasant, or farm enterprises, units & 295,335 & 305,108 & 314,882 \\
\hline
\multicolumn{4}{|@{}>{\raggedright\arraybackslash}p{(\columnwidth - 6\tabcolsep) * \real{1.0000} + 6\tabcolsep}|@{\,}}{%
Note: Compiled by the authors based on calculations} \\
\hline
\end{longtable}


\begin{multicols}{2}

The study focused on quantitative data, which allowed us to formulate
accurate and tangible forecasts for Kazakhstan' s
agro-industrial complex. It should be emphasised that the impact of
qualitative factors, such as climate change or access to technology,
does not pose a significant risk to the accuracy of the forecasts in
this analysis. The influence of these factors, although important for
long-term development, does not currently have a significant effect on
general trends in agriculture, especially in the short term, which is
confirmed by the stability of historical data. Moreover, quantitative
data based on current trends allow for a fairly accurate forecast of the
main directions of the industry' s development, as many
of these qualitative factors have not yet shown sharp fluctuations that
could significantly change the dynamics. In the future, it will be
possible to take these factors into account in case of their more
pronounced influence, but at the moment their impact on the forecasted
indicators remains limited.

The results of forecasting economic processes using a model based on
data dynamics as shown in studies by various authors, turned out quite
convincing.

Based on the modeling results, we can formulate a comprehensive idea of
the current state and prospects for the development of agricultural
businesses in Kazakhstan. These findings may be useful for strategy
development, decision-making, and planning future activities in
agricultural sector.

The practical value of the results obtained is that the forecast data
can be used to optimise management decisions in the agro-industrial
complex of Kazakhstan. In particular, the following aspects of
management can be improved on the basis of the proposed forecasting
model:

- Production planning, due to the fact that forecasts based on
historical data can help agricultural enterprises to plan production
volumes more accurately and minimise the risks of overproduction or
shortage of products. This is important for efficient resource
utilisation and loss prevention.

- Building pricing policy and marketing, by forecasting the dynamics of
prices for agricultural products, which allows to optimise pricing
strategy and increase the competitiveness of products in domestic and
foreign markets. Companies will be able to calculate more accurately
when it is more profitable to sell products and when to hold them for
sale in the future.

- Resource and inventory management also needs predictive data to help
plan resource requirements (fertiliser, seed, fuel, etc.) more
accurately, and to optimise logistics and inventory management, which
will reduce procurement and storage costs.

- Investment decision-making relies directly on forecasts, in the
agricultural sector, which can help enterprises to better justify
long-term investments, for example, in expanding production capacity,
purchasing new machinery or introducing innovative technologies such as
precision farming systems.

- Using forecasts to assess possible risks associated with climate
change, global price fluctuations and other external factors will allow
enterprises to take early action to minimise risks and adjust long-term
strategies.

The chosen research method is unequivocally based on modeling economic
indicators. In this case, it is based on development trends in GDP data
for agriculture and quantitative data of the research object and
investments. It does not reflect qualitative indicators and other
factors affecting the current situation and further development of
agribusinesses in the country. However, as a basis for an in-depth
subject study of this topic, we consider the results of this study both
relevant and applicable.

Recognizing positive trends in the development of agribusinesses, we
feel important to note that at the moment, agriculture's development
level and its contribution to the national GDP remain low compared to
previous decades.

We believe that transitioning to an economic model focused on the
effective development of all forms of agricultural enterprises analyzed
in this study will effectively address issues of developing the
agro-industrial complex, growing agricultural production, which will
ultimately contribute to ensuring food security and improving living
standards.

{\bfseries Conclusions.} As a result of the study aimed at analysing the
development of the agro-industrial complex of Kazakhstan, it was
revealed that the transition to an economic model focused on the
effective development of all forms of agrarian enterprises is a key step
in solving the existing problems of agriculture. This, in turn, will
have a positive impact on the growth of agricultural production, which
contributes to improving food security and improving the quality of life
of the country' s population.

Based on the findings and analysis of the current situation in the
agro-industrial complex, several important recommendations for public
policy and private business to support sustainable agricultural
development can be identified.

Recommendations for public policy:

1. Development of agricultural infrastructure: To improve the conditions
for agriculture in remote areas of Kazakhstan, it is necessary to
increase investment in infrastructure, including transport networks,
product storage, water supply and electricity. Creation of modern
logistics chains will help to reduce costs and increase the
competitiveness of agro-producers.

2. Support for the introduction of innovative technologies: It is
necessary to develop programmes that stimulate the introduction of new
technologies in agriculture. The inclusion of subsidies for the purchase
of machinery, the creation of educational centres to train farmers in
modern agricultural methods, and tax incentives for those using
innovative and environmentally friendly technologies can help to
increase the productivity and sustainability of the sector.

3. Ensuring access to finance for small and medium-sized agricultural
producers: Creating more accessible financial instruments for small and
medium-sized agricultural producers, such as soft loans, state support
programmes and loan rate subsidies, will increase investment in the
development of small agribusinesses and improve their competitiveness.

4. State support for green technologies: State support mechanisms should
be developed and implemented to promote sustainable and environmentally
friendly technologies such as organic farming, water conservation and
carbon reduction technologies. This could include both direct financial
support and certification programmes for environmentally friendly
products. Recommendations for private business:

1. Invest in green technologies: Private agricultural enterprises should
focus on long-term investments in environmentally friendly technologies.
The introduction of organic farming, energy efficient irrigation
methods, use of renewable energy sources (solar panels, biogas) will not
only improve environmental sustainability, but also create a competitive
advantage in international markets where consumers are increasingly
oriented towards environmentally friendly products.

2 Optimisation of business processes: It is recommended to optimise
internal processes to improve efficiency and reduce costs. This includes
implementing inventory management systems, improving logistics and
automating business processes. Developing and implementing digital
platforms for trade and marketing of agricultural products will help
entrepreneurs enter new markets and provide more accurate control over
processes.

3 Training and professional development of employees: To improve the
quality of work and competitiveness in agribusiness, it is necessary to
organise courses and trainings for employees aimed at teaching modern
farming methods, use of new technologies and innovative approaches in
agro-production. This will help to create a more skilled labour force
and increase productivity.

4. Co-operation with scientific institutions and public authorities:
Private business should actively co-operate with scientific institutions
to develop and implement innovative solutions, as well as with public
authorities to obtain affordable financial support and implement
subsidies. This co-operation can help to improve product quality and
introduce advanced agricultural technologies.
\end{multicols}


{\bfseries References}

1. Moldashev A.B., Nikitina G.A., Guseva G.YA. Povyshenie roli
agropromyshlennogo proizvodstva Kazahstana na obshchem rynke
stran-uchastnic EAES: rekomendacii.- Almaty, 2017. - 35 s. {[}in
Russian{]}

2 Saparova G.K., Sultanova G.T. Problemy i perspektivy razvitiya
agrarnogo proizvodstva v RK v sovremennyh usloviyah: monografiya.
-Atyrau: red.izd. centr Atyrauskij universitet nefti i gaza imeni
S.Utebaeva, 2020. -218 s. {[}in Russian{]}

3. Ishfaq, M., Wang, Y., Xu, J. et al. Improvement of nutritional
quality of food crops with fertilizer: a global meta-analysis //Agron.
Sustain.-2023. -Vol. 43(74).
\href{https://doi.org/10.1007/s13593-023-00923-7}{DOI
10.1007/s13593-023-00923-7}

4. LavrikovaYU.G., Malysh E.V. Zelenaya ekonomika v klasternom razvitii
// Regional' naya ekonomika: teoriya i praktika. -2014.
-№ 36(363). -S. 48-58.

URL:
\url{https://cyberleninka.ru/article/n/zelenaya-ekonomika-v-klasternom-razvitii}
{[}in Russian{]}

5. YAnovskaya O.A., Saginova S.A. Prodovol' stvennaya
bezopasnost' Kazahstana v usloviyah integracii: problemy i perspektivy:
Monografiya. -- Karaganda: TOO «Tengri ltd». 2020. - 210 s. ISBN
978-601-7950-91-0 {[}in Russian{]}

6. ElDala -- novosti i analitika agrarnogo bizneza Kazahstana.- 2023. -
URL:
\url{https://eldala.kz/novosti/kazahstan/15357-v-kazahstane-vyroslo-kolichestvo-selhozkooperativov}
(Extracted on May 26th, 2024) {[}in Russian{]}

7. Babich, T.N. Prognozirovanie i planirovanie v usloviyah rynka:
Uchebnoe posobie / T.N. Babich, I.A. Koz'eva, YU.V. Vertakova, E.N.
Kuz'bozhev. - M.: NIC INFRA-M, 2013.- 336 c. {[}in Russian{]}

8. Zamkov O.O., Tolstopyatenko A.V., CHeremnyh YU.N. Matematicheskie
metody v ekonomike.-- M.: Delo i Servis, 2001.- 368 s. ISBN
5-86509-054-2 {[}in Russian{]}

9. Malyhin V.I. Matematika v ekonomike. -- M.: INFRA-M, 1999.- 355 s.
ISBN 5-86225-867-1. {[}in Russian{]}

10. Garmash A. N. Ekonomiko-matematicheskie metody i prikladnye modeli:
uchebnik dlya bakalavriata i magistratury / A. N. Garmash, I. V. Orlova,
V. V. Fedoseev; pod redakciej V. V. Fedoseeva. - 4-e izd., pererab. i
dop.- Moskva: Izdatel'stvo YUrajt, 2022. - 328 s.

ISBN 978-5-9916-3698-8 {[}in Russian{]}

\emph{{\bfseries Information about the authors}}

Saginova S.- PhD, Associate Professor,K. Kulazhanov Kazakh University of
Technology and Business, Astana, Kazakhstan, e-mail:
saginova.s@gmail.com;

Saparova D.- PhD, Turan-Astana University, Astana, Kazakhstan, e-mail:
saparova.ok@ mail.ru;

Stukach V. - Doctor of Economics, Professor, P. A. Stolypin Omsk State
Agrarian University, Omsk, Russian Federation, e-mail: vic.econ@mail.ru.

\emph{{\bfseries Сведения об авторах}}

Сагинова С.А. - доктор PhD, ассоциированный профессор,Казахский
университет технологии и бизнеса им. К. Кулажанова, Астана, Казахстан,
e-mail: saginova.s@gmail.com;

Сапарова Д.А. - докторант PhD, «Туран-Астана», Астана, Казахстан,
e-mail: saparova.ok@mail.ru;

Стукач В.Ф. -- доктор экономических наук, профессор, Омский
государственный аграрный университет им. П.А. Столыпина, Омск,
Российская Федерация, e-mail: vic.econ@mail.ru.
