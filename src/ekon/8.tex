

\id{IRSTI 06.77.02}

{\bfseries MEDICAL REHABILITATION PROGRAMS IN INSURANCE:}

{\bfseries CASE STUDY AND NEW REALITIES}

{\bfseries A.M. Kurmanov}, {\bfseries A.B. Bekmagambetov, A.Ye.
Sabidullina\textsuperscript{\envelope },} {\bfseries L.I. Yedilbayeva}

Republican Research Institute for Labor Protection of the Ministry of
Labor and Social Protection of the Population of the Republic of
Kazakhstan, Astana, Kazakhstan

{\bfseries \textsuperscript{\envelope }}Corresponding author:
\href{mailto:sabidullina96@inbox.ru}{\nolinkurl{sabidullina96@inbox.ru}}

The purpose of this study is to analyze the impact of social protection
measures for employees in the Republic of Kazakhstan based on the social
insurance system, with a focus on coverage of necessary medical
services. This article will provide a broad overview of current
literature on social protection measures for workers, justifying the
choice of research methods. The results will help to see the path of
funds movement, as well as reflect current market prices for an
approximate minimum package of necessary services and funds for the
affected person. These data reflect the sufficiency of funds to cover a
certain amount of necessary services and funds within the framework of
current legislation. The following methods will be used:
economic-statistical, evaluative-comparative, and logical and
analytical. The current legislative framework for protecting the
interests of workers in Kazakhstan will be identified. It is important
to note that social protection for employees is an essential component
of the economy, and reforms in this area must be effective in the long
run. Kazakhstan is making efforts to improve its legislative framework
in this regard. Cases that reflect modern market realities and the
transformation of social protection will be presented. Social insurance
for employees is an essential component of the overall social protection
system for the population. This insurance provides financial support to
employees in times of illness, injury, or temporary disability, as well
as in the event of the loss of a primary provider for their family. This
ensures the stability and security of working individuals and their
relatives.

{\bfseries Keywords:} employee health insurance, social protection of
employees, Occupational safety and health, industrial accidents,
Compensation for harm, medical rehabilitation measures.

{\bfseries САҚТАНДЫРУДАҒЫ МЕДИЦИНАЛЫҚ ОҢАЛТУ БАҒДАРЛАМАЛАРЫ:}

{\bfseries ІС ӘДІСІ ЖӘНЕ ЖАҢА ШЫНДЫҚТАР}

{\bfseries А.М. Курманов, А.Б. Бекмагамбетов, А.Е.
Сабидуллина\textsuperscript{\envelope }}, {\bfseries Л.И.} {\bfseries Едильбаева}

Қазақстан Республикасы Еңбек және халықты әлеуметтік қорғау
министрлігінің Еңбекті қорғау жөніндегі республикалық ғылыми-зерттеу
институты, Астана, Қазақстан,

e-mail: sabidullina96@inbox.ru

Бұл зерттеудің мақсаты қажетті медициналық қызметтермен қамтуға баса
назар аудара отырып, әлеуметтік сақтандыру жүйесіне негізделген
Қазақстан Республикасындағы қызметкерлерді әлеуметтік қорғау шараларының
әсерін талдау болып табылады. Бұл мақалада зерттеу әдістерін таңдауды
негіздейтін қызметкерлерді әлеуметтік қорғау шаралары туралы заманауи
әдебиеттерге кең шолу жасалады. Нәтижелер қаражаттың қозғалу жолын
көруге көмектеседі, сонымен қатар зардап шеккен адамға қажетті қызметтер
мен қаражаттың минималды пакетінің ағымдағы нарықтық бағаларын
көрсетеді. Бұл деректер қолданыстағы заңнама шеңберінде қажетті
қызметтер мен қаражаттың белгілі бір көлемін жабу үшін қолма-қол ақшаның
жеткіліктілігін көрсетеді. Келесі әдістер қолданылады:
экономикалық-статистикалық, бағалау-салыстырмалы және
логикалық-аналитикалық. Қазақстандағы қызметкерлердің мүдделерін қорғау
үшін қолданыстағы заңнамалық база айқындалатын болады. Қызметкерлерді
әлеуметтік қорғау экономиканың маңызды құрамдас бөлігі болып табылатынын
және бұл саладағы реформалар ұзақ мерзімді перспективада тиімді болуы
керек екенін атап өткен жөн. Қазақстан осыған байланысты өзінің
заңнамалық базасын жетілдіруге күш салуда. Қазіргі заманғы нарықтық
шындықты және әлеуметтік қорғау жүйесінің трансформациясын көрсететін
кейстер ұсынылатын болады. Қызметкерлерді әлеуметтік сақтандыру халықты
әлеуметтік қорғаудың жалпы жүйесінің маңызды құрамдас бөлігі болып
табылады. Бұл сақтандыру қызметкерлерге ауырған, жарақат алған немесе
уақытша еңбекке жарамсыз болған жағдайда және олардың отбасылары үшін
негізгі асыраушысынан айырылған жағдайда қаржылық қолдау көрсетеді. Бұл
жұмыс істейтін адамдар мен олардың жақындарының тұрақтылығы мен
қауіпсіздігін қамтамасыз етеді.

{\bfseries Түйін сөздер:} қызметкерлерді медициналық сақтандыру,
қызметкерлерді әлеуметтік қорғау, Еңбекті қорғау, өндірістегі жазатайым
оқиғалар, зиянды өтеу, медициналық оңалту жөніндегі іс-шаралар.

{\bfseries ПРОГРАММЫ МЕДИЦИНСКОЙ РЕАБИЛИТАЦИИ В СТРАХОВАНИИ:}

{\bfseries КЕЙС МЕТОД И НОВЫЕ РЕАЛИИ}

{\bfseries А.М. Курманов, А.Б. Бекмагамбетов, А.Е.
Сабидуллина\textsuperscript{\envelope },} {\bfseries Л.И. Едильбаева}\\
Республиканский научно-исследовательский институт по охране труда
Министерства труда и социальной защиты населения Республики Казахстан,
Астана, Казахстан,

e-mail:
\href{mailto:sabidullina96@inbox.ru}{\nolinkurl{sabidullina96@inbox.ru}}

Целью данного исследования является анализ влияния мер социальной защиты
работников в Республике Казахстан, основанных на системе социального
страхования, с акцентом на охват необходимыми медицинскими услугами. В
этой статье будет представлен широкий обзор современной литературы о
мерах социальной защиты работников, обосновывающий выбор методов
исследования. Результаты помогут увидеть путь движения средств, а также
отражает актуальные рыночные цены на приближенно минимальный пакет
необходимых услуг и средств для пострадавшего лица. Эти данные отражают
достаточность денежных средств для покрытия определенного объема
необходимых услуг и средств в рамках действующего законодательства.
Будут использованы следующие методы: экономико-статистический,
оценочно-сравнительный и логико-аналитический. Будет определена
действующая законодательная база для защиты интересов работников в
Казахстане. Важно отметить, что социальная защита работников является
важным компонентом экономики, и реформы в этой области должны быть
эффективными в долгосрочной перспективе. Казахстан прилагает усилия для
совершенствования своей законодательной базы в этом отношении. Будут
представлены кейсы, отражающие современные рыночные реалии и
трансформацию системы социальной защиты. Социальное страхование
работников является важным компонентом общей системы социальной защиты
населения. Это страхование обеспечивает финансовую поддержку работникам
в случае болезни, травмы или временной нетрудоспособности, а также в
случае потери основного кормильца для их семей. Это обеспечивает
стабильность и безопасность работающих людей и их близких.

{\bfseries Ключевые слова:} медицинское страхование работников, социальная
защита работников, охрана труда, несчастные случаи на производстве,
возмещение вреда, мероприятия по медицинской реабилитации.

{\bfseries Introduction.} Employee health insurance is part of a compulsory
insurance system that aims to protect employees from financial risk
associated with various social risks, including illness, disability, job
loss, and old age. The purpose of this insurance is to provide social
protection for employees and their families through the provision of
benefits and compensation in case of an insured event. Occupational
safety and health (OSH) are an essential part of the Decent Work program
of the International Labour Organization (ILO). The ILO defines "decent
work" as the right to productive employment in conditions of freedom,
equality, security, and human dignity, and states that work can only be
considered decent if it is safe and healthy {[}1{]}.

However, according to the Global Monitoring Report published by the
World Health Organization (WHO) and the ILO, during 2016, 1.9 million
people worldwide died from occupational diseases and injuries. Most of
these deaths were due to respiratory and cardiovascular conditions
{[}2{]}. Occupational accidents have a significant impact on
people' s well-being and can lead to high costs for
social health and insurance systems in any country, disrupting the
sustainability of production systems and working life. To address these
challenges, it is essential to implement effective measures that take
full advantage of advances in safe, healthy, and decent work. These
measures should aim to maintain a sustainable production system while
ensuring the well-being of workers. Work-related injuries are a growing
concern in the workforce, as they can have a significant impact on
organizations, especially financially. However, researchers disagree on
the definition of indirect costs associated with workplace accidents.
This study uses simulated scenarios based on current market conditions
to calculate costs.

The functions of providing medical rehabilitation for victims at work
are not necessarily solved within the framework of the accident
insurance system. In different EU countries, the situation with medical
rehabilitation differs significantly from one another. In some
countries, the NS and PP insurance system has responsibilities for all
types of rehabilitation (including professional and social), in others -
only vocational rehabilitation, in others -- only medical rehabilitation
{[}3{]}.

Various cases were considered in the development of these scenarios,
which were built upon examples from different industries. Despite the
growing number of well-designed studies confirming the effectiveness of
comprehensive rehabilitation programs from an evidence-based medicine
perspective, access to them in the Republic of Kazakhstan remains
somewhat difficult due to both the imperfections of the regulatory
framework and the lack of insurance coverage for such programs.

In the last 2-3 years, there have been changes to the regulatory
framework for medical rehabilitation, and therefore, it is of great
practical interest to analyze the existing regulatory framework and its
impact on simulated cases. The aim is to evaluate the effectiveness of
the changes in the regulatory framework governing medical rehabilitation
in Kazakhstan using case studies as an example.

This study expands the understanding, interrelation and interaction in
the totality of understanding of medical rehabilitation and insurance,
and reflects modern market realities in the framework of the
implementation of current legislative acts in this area based on the
given cases based on market prices in Kazakhstan.

{\bfseries Materials and methods.} The investigation relies on a
comprehensive analysis of international and national statistical
reports, as well as official publications issued by prominent
international financial institutions, serving as the foundation for our
research. The main research methods for the development are: a method of
systematization and generalization for a comprehensive review of the
mechanisms of state assistance to the development of the medical
rehabilitation, a method of analyzing documents for studying legislation
in the field under study and a method of logical generalization for the
development of conclusions. All the legislative acts and regulations
specified in this article in the field under study are valid documents
and are relevant.

In order to evaluate the state of infrastructure development within the
Republic of Kazakhstan regarding medical rehabilitation services
provided within the context of insurance coverage, it is crucial to
assess the efficacy of resource allocation within the rehabilitation
process. When a country maintains a conducive regulatory environment
characterized by low levels of corruption, coupled with the
implementation of effective laws, governmental programs, and strategic
development plans, it significantly impacts the behavior of subjects.

Employee insurance is an essential component of the social protection
system for the population. It provides financial protection for
employees in case of illness, injury, temporary disability, or loss of a
breadwinner. This helps ensure stability and security for workers and
their families.

The basic principles of employee insurance include mandatory
participation in the system for all working citizens, solidarity in
covering risks, and fair distribution of financing between employees and
employers.

Employee social security systems can vary from country to country
depending on legislation and socio-economic conditions, but their goal
is always to provide social protection for workers. In general, employee
social insurance plays an important role in ensuring the stability and
well-being of employees and their families. It provides them with
protection against financial risks associated with work.

However, the amounts allocated for medical rehabilitation of injured
employees may not always be sufficient to cover all their needs. In
order to determine the level of coverage in modern Kazakhstan, we have
studied the legislative framework and built case models.

{\bfseries Results and discussion.} Industrial accidents and occupational
illnesses not only cause harm to individuals and their families, but
also significantly impact society' s economy. The
International Social Security Association estimates that the annual cost
of non-fatal workplace accidents alone amounts to approximately 4\% of
the global gross domestic product {[}4{]}.

In recent years, there has been a trend towards a decrease in workplace
accidents {[}5{]}. This can be attributed to preventive measures and
initiatives implemented by companies and government agencies, as well as
the increasing proportion of the workforce employed in sectors with
lower accident rates, such as services {[}6{]}. The reintegration of
workers into the workforce through the support provided by state
occupational accident insurance is a critical aspect that may
significantly impact the effectiveness of occupational rehabilitation
services and individuals'{} utilization of medical
services.

The specific composition of the workforce, psychological factors, and
the level of healthcare provided in the event of industrial accidents
all play a crucial role in determining whether individuals who have
sustained work-related injuries to their neck, back, or shoulders will
return to the labour market or resume their original employment
{[}7-9{]}.

For instance, several demographic variables such as gender and income,
as well as psychosocial factors such as an individual' s
confidence level, can significantly influence the likelihood of patients
returning to their pre-injury employment. Furthermore,
rehabilitation-related factors, such as the successful completion of
rehabilitation programmes, also play a pivotal role in facilitating the
reintegration process {[}9{]}. Industrial accidents and their
repercussions have emerged as a significant concern in contemporary
society. The construction industry, in particular, bears a notable
responsibility, accounting for a staggering 21.5\% of fatalities and
12.7\% of injuries {[}10{]}. Beyond the incalculable loss of human lives
and societal well-being, these occurrences have a direct impact on the
organizational structure and operational efficiency of companies,
resulting in a decline in productivity and profit margins. In the
legislative framework of the Russian Federation, medical rehabilitation
is defined as a comprehensive set of medical and psychological
interventions aimed at either the full or partial restoration, and in
some cases, compensation for impaired or lost functions of an affected
organ or system. This process also involves maintaining bodily functions
during acute pathological processes or exacerbations of chronic
conditions. Furthermore, it encompasses preventive measures, early
detection, and correction of potential functional impairments in damaged
organs or systems, aiming to prevent disability, enhance quality of
life, maintain employment capacity, and promote social reintegration of
patients into society {[}11-12{]}.

In the context of Kazakhstan, the government is actively working to
expand and enhance the infrastructure of rehabilitation centers across
the country, with a focus on providing comprehensive healthcare services
for the rehabilitation of individuals. Pursuant to the regulations on
medical rehabilitation, as stipulated in Decree No. 21381 dated October
9, 2020 {[}13{]}, the process of medical rehabilitation constitutes a
comprehensive array of medical services designed to preserve, partially
or fully restore, and (or) replace impaired and/or lost functions of an
individual' s body.

In the year 2013, Order No. 759, issued by the Ministry of Healthcare of
the Republic of Kazakhstan on December 27th, 2013 {[}14{]}, formalized a
standard for the provision of medical rehabilitation services to the
population within the country, mandating that such rehabilitation be
delivered through a multidisciplinary approach involving medical
professionals from various specialties, with a phased approach to
restoring patients'{} health status. On January 31, 2024,
the Ministry of Health of the Republic of Kazakhstan issued Order No. 20
{[}15{]}, which approved the Rules for Reimbursement of Expenses for
Preventive and/or Rehabilitation Measures under Compulsory Insurance for
Employees in the Performance of Official Duties. These Rules establish
the procedure for the reimbursement of expenses related to preventive
and/or rehabilitation measures. Within the framework of this document,
rehabilitation measures are defined as a set of actions aimed at
restoring a worker' s professional capacity and reducing
their level of disability.

The term "preventive measures", in turn, refers to initiatives designed
to establish and enhance safe working conditions and provide mechanisms
for their compensation.

A comprehensive list of preventive measures has been established,
providing guidance on the types of measures that policyholders can seek
reimbursement for from their insurers under the provisions of the "Law
of the Republic of Kazakhstan on Compulsory Insurance of Employees
against Accidents in the Performance of Labor (Official)
Duties"{[}16{]}.

The mechanism of harm caused by injury to health is explored in greater
detail below. Should the employee have been asked a question at the time
of an accident or occupational injury, and his presence at the scene
could be explained bym the fulfilment of his work obligations, he is
entitled to receive social compensation, including reimbursement for
expenses incurred by employees in the course of their labour or other
duties as prescribed by the laws of the Republic of Kazakhstan.

1. Compensation for damage caused by damage to health from the
employer' s funds.

• Compulsory social benefits for disability.

In accordance with Article 133 of the Labor Code of the Republic of
Kazakhstan {[}17{]}, in case of an occupational injury or injury to
health, the employer is obliged, at his own expense, to pay employees
social benefits for temporary disability in the amount of one hundred
percent of the average salary from the first day of disability. The
basis for the payment of social benefits for temporary disability are
disability certificates.

• Compensation for damage caused by damage to health

According to paragraph 3 of Article 122 of the Labor Code of the
Republic of Kazakhstan {[}17{]}, when harm is caused to an employee
related to the establishment of the degree of loss of professional
ability from five to twenty-nine percent inclusive, the employer is
obliged to reimburse the employee for lost earnings and expenses caused
by damage to his health. The amount of expenses caused by damage to
health reimbursed by the employer during the period of determining the
degree of disability may not exceed two hundred and fifty monthly
calculation indices established for the corresponding financial year by
the law on the republican budget at the time of payment.

A guaranteed amount of free medical care at the expense of budgetary
funds is provided to citizens regardless of the status of insurance:

 ambulance services, including with the involvement of medical aviation
in certain cases

 Primary health care services (PHC)

Specialized outpatient medical care: services for injuries, poisoning or
other urgent conditions;

 Medical rehabilitation:

\begin{itemize}
\item
  in the treatment of the underlying disease;
\item
  for tuberculosis patients;
\end{itemize}

Palliative care.

Citizens who regularly pay contributions to the CSHI and have the status
of "INSURED" can receive a wider range of medical services without
paying them additionally. The list of compulsory social health insurance
also includes medical rehabilitation.

The employer must pay social benefits for temporary disability either
until the person fully recovers and goes to work, or until the employee
is examined and the medical and social examination (ITU) determines the
disability and the degree of loss of his professional ability to work.

• Voluntary health insurance and voluntary social benefits

Kazakhstan maintains a high level of private spending on medical care.

Collective agreements may provide for one-time payments by the employer
for burial and loss of a breadwinner, depending on the composition of
the family, if death occurred as a result of an occupational injury or
occupational disease, as well as benefits for various disability groups.

2. Compensation for damage caused by damage to health from an insurance
company (with which the employer is obliged to conclude a compulsory
accident insurance contract for the employee).

A) In accordance with paragraph 2 of Article 19 of the Law of the
Republic of Kazakhstan "On compulsory insurance of an employee against
accidents in the performance of his/her labor (official) duties"
{[}16{]} compensation for additional expenses caused by damage to the
employee' s health in case of establishing the degree of
loss of professional ability from thirty to one hundred percent is
carried out by the insurer on the basis of documents confirming these
expenses are presented by the employee or the person who incurred these
expenses. At the same time, expenses for medical care provided within
the guaranteed volume of free medical care and in the system of
compulsory social health insurance are not subject to reimbursement.

The total amount of insurance payments for reimbursement of additional
expenses caused by damage to health may not exceed the following amounts
(in monthly calculation indices established for the corresponding
financial year by the law on the republican budget):

 when determining the degree of loss of professional ability to work
from thirty to fifty -- nine percent inclusive - 500;

 when determining the degree of loss of professional ability to work
from sixty to eighty -- nine percent inclusive - 750;

 when determining the degree of loss of professional ability to work
from ninety to one hundred percent inclusive -- 1,000.

B) In accordance with Chapter 2 of the Rules for Reimbursement of Costs
for Preventive Measures and (or) Rehabilitation Measures {[}15{]}, the
insurer, in addition to reimbursing the additional costs specified in
paragraph A, reimburses the policyholder and (or) the beneficiary for
the costs actually incurred (part of the costs) for preventive measures
(according to the list) within the limits the sum insured provided for
in the employee' s compulsory accident insurance
contract. The maximum amount of reimbursement of the costs of the
policyholder and (or) the beneficiary may not exceed 6 (six) percent of
the paid insurance premium calculated on the expiration date of the
compulsory employee accident insurance contract concluded between the
policyholder and the insurer.

C) In accordance with Chapter 3 of the Rules for Reimbursement of Costs
for Preventive Measures and (or) Rehabilitation Measures, the insurer,
in addition to reimbursing the additional costs specified in paragraph
A, reimburses the policyholder and (or) the beneficiary for the costs
actually incurred (part of the costs) for social and (or) vocational
rehabilitation in accordance with Appendix 3 (see Table 1) to The Rules
and the MR for the IPAR of persons with disabilities (not included in
the guaranteed amount of free medical care and compulsory social health
insurance). The maximum amount of reimbursement of the costs of the
policyholder and (or) the beneficiary may not exceed 6 (six) percent of
the paid insurance premium calculated on the expiration date of the
compulsory employee accident insurance contract concluded between the
policyholder and the insurer.

{\bfseries Table 1 - Appendix 3 to the Rules for Reimbursement of costs for
preventive (or) rehabilitation measures. Measures for social and (or)
vocational rehabilitation}

% \begin{longtable}[]{@{}
%   >{\raggedright\arraybackslash}p{(\columnwidth - 2\tabcolsep) * \real{0.3333}}
%   >{\raggedright\arraybackslash}p{(\columnwidth - 2\tabcolsep) * \real{0.6667}}@{}}
% \toprule\noalign{}
% \multirow{3}{=}{\begin{minipage}[b]{\linewidth}\raggedright
% 1. Social rehabilitation
% \end{minipage}} & \begin{minipage}[b]{\linewidth}\raggedright
% psychological support, assistance and correction services;
% \end{minipage} \\
% & \begin{minipage}[b]{\linewidth}\raggedright
% medical and psychological consultations;
% \end{minipage} \\
% & \begin{minipage}[b]{\linewidth}\raggedright
% consultations on rehabilitation and a healthy lifestyle.
% \end{minipage} \\
% \midrule\noalign{}
% \endhead
% \bottomrule\noalign{}
% \endlastfoot
% \multirow{4}{=}{2. Professional rehabilitation} & workplace
% modernization to improve work processes and employee adaptation; \\
% & training in professional retraining and advanced training courses; \\
% & provision of therapeutic and preventive nutrition for medical
% reasons; \\
% & auto-correction (use of bandages, orthoses, corsets) depending on the
% working conditions and the production process. \\
% \multirow{4}{=}{3. Restorative and reconstructive rehabilitation} &
% rehabilitation and rehabilitation therapy (medication, physical,
% occupational therapy, kinesiotherapy, manual therapy, spa
% recreation); \\
% & reconstructive surgery: services for restoring the integrity of the
% human body systems responsible for movement, restoring the biological
% functions of the skin, maximizing functional abilities and recovery,
% minimizing the consequences caused by an industrial accident; \\
% & prosthetic and orthopedic care (selection and use of means of
% movement, orthoses, orthopedic shoes); \\
% & development and training of programs using step-by-step tasks and
% actions as a prerequisite for involvement in functional training in
% elementary self-care activities (including self-care). \\
% \multirow{2}{=}{4. Speech therapy and language rehabilitation} &
% clinical and/or instrumental examinations, diagnosis, treatment and
% management of speech, voice, language, fluency and swallowing disorders
% that affect the ability to communicate; \\
% & acquisition of communication systems and devices for persons with
% disabilities in verbal communication. \\
% \end{longtable}

\emph{Source: compiled from the source {[}15{]}}

An important point for citizens of the Republic of Kazakhstan is that if
the list of documents necessary for reimbursement of costs is available
in electronic form in databases or information systems of state bodies,
it is not required to be provided by the policyholder or beneficiary.
The insurer can access this information through the organization
responsible for maintaining the database.

To calculate the amount of compensation, the insurer obtains the
necessary documents from the database or information system with the
consent of the beneficiary through the same organization. The insurer is
still responsible for reimbursing costs associated with rehabilitation
measures for events that occur during the validity period of the
employee' s mandatory accident insurance policy.After the
initial determination of the degree of loss of ability to work due to an
injury, the injured employee is entitled to receive compensation for the
costs of one sanatorium treatment, regardless of their individual
rehabilitation program. The reimbursement of these expenses is carried
out up to a maximum of 100 times the monthly index established by law
for the corresponding financial year in the republican budget. This
compensation is based on documents confirming the expenses incurred.

1. State benefits for disability due to labor injury or occupational
disease.

In accordance with Article 937 of the Civil Code of the Republic of
Kazakhstan {[}18{]}, when a citizen is injured or otherwise damaged, the
victim' s lost earnings (income), which he had or
definitely could have had in connection with the establishment of his
degree of loss of professional ability to work in the performance of his
labor (official) duties, is subject to compensation for the entire
period of disability . This type of compensation payments belongs to the
compulsory social insurance system and is regulated by the Law of the
Republic of Kazakhstan "On Compulsory Social Insurance" {[}19{]}. The
State Social Insurance Fund (Fund) is formed on the basis of social
contributions that the employer is obliged to make. In this context, the
employer is a payer of social contributions. The employer makes monthly
social contributions for its employees in the form of 5\% of the salary
of each employee. At the same time, an independent employed person, for
example, an individual entrepreneur, can also be a payer. The recipient
of the social payment is a person for whom payments were made, or who
paid social contributions to the State Social Insurance Fund (SSIF).
SSIF is established by the state as an off---budget organization and is
not included in the state budget system.

In case of compensation for earnings or part of it, the disability
pension assigned to the victim in connection with an occupational
injury, as well as other pensions assigned both before and after an
occupational injury, are not counted towards compensation. Also, the
earnings received by the victim after the injury are not counted towards
the compensation for harm.

In accordance with paragraph 3 of Article 248 of the Social Code of the
Republic of Kazakhstan {[}20{]}, persons with disabilities of groups I
and II are exempt from paying mandatory pension contributions
(hereinafter -- OPV) if the disability is established indefinitely.
These categories can pay OPV upon application (voluntarily).

In accordance with paragraph 6 of Article 248 of the Social Code of the
Republic of Kazakhstan, agents for persons with disabilities of groups I
and II are exempt from paying mandatory pension contributions from the
employer (hereinafter -- OPVR), if the disability is established
indefinitely.

In accordance with article 26 of the Law "On Compulsory Social Health
Insurance", contributions for persons with disabilities (regardless of
the group and the validity period of the disability certificate) are
paid by the State. Employers are exempt from paying CSHI deductions for
employees with disabilities in accordance with paragraph 3 of article 27
of the CSHI Law.

In accordance with paragraph 5 of the Social Code of the Republic of
Kazakhstan, people with disabilities receive the state social disability
allowance for the entire period of disability (ITU), which depends on
the subsistence minimum (PM) established for the current year: disabled
people of group I -2.20 PM, disabled people of group II - 1.83 PM,
disabled people of group III - 1.61 PM.

The case method made it possible to apply theoretical knowledge to
solving practical problems regarding the monetary coverage of the needs
of the injured person during work. This approach compensates for an
exclusively scientific approach and provides a broader understanding of
the business and market processes.

Case 1. Injury at the production site (see Table 2).

An employee (engineer) of the airline, while performing his official
duties at a technical warehouse, was hit on the shoulder as a result of
the departure of the mechanism parts.

First aid was provided at the scene by paramedics (a health center
paramedic, as the injury was sustained on the territory of the
enterprise). Next, the victim was transported to a medical facility. As
a result of the accident, according to the Rehabilitation Routing Scale,
the patient was assigned 3 (Severe dysfunction and disability), ICD 10
S42 {[}21{]}.

The treatment was carried out in inpatient conditions, providing
round-the-clock medical supervision, treatment, care, as well as the
provision of a bed with meals.

After the transfer of the patient from the specialized department to the
rehabilitation department within the same medical organization,
physiotherapy and massage, as well as consultations with other
specialists, are prescribed according to indications. All of the above
expenses were covered under the CSHI.

At the company, at the scene of the accident, occupational safety and
health specialists took measures to organize an investigation of the
accident and prepare investigation materials, and notifications of the
incident were sent to a number of authorities. The accident was
classified by the commission conducting the investigation as an
industrial accident, as a result of which an industrial accident act was
issued. One copy of the act on the investigation of a group industrial
accident, together with copies of the investigation materials, was sent
to the executive body of the insurer (within three days). Upon
notification of an insured event that occurred during the period of
insurance coverage under the employee' s compulsory
accident insurance contract, the insurer immediately registered it and
submitted information on this insured event to the organization for the
formation and maintenance of a database in accordance with the
regulatory legal act of the authorized body for regulation, control and
supervision of the financial market and financial organizations.

During the medical examination, the degree of loss of professional
ability to work was determined from thirty to fifty-nine percent,
inclusive, for 6 months. The victim required additional expenses caused
by damage to his health, which are not included in the costs of medical
care provided within the guaranteed volume of free medical care and in
the compulsory social health insurance system. These expenses are
covered by the insurance company in the amount of 500 MCI for 2024
(3,692 tenge).

In accordance with Chapter 2 of the Rules for Reimbursement of Costs for
Preventive Measures and (or) Rehabilitation Measures, the insurer, in
addition to reimbursing the additional costs specified above, reimburses
the policyholder and (or) the beneficiary for the costs actually
incurred (part of the costs) for preventive measures (according to the
list) within the insurance amount provided for by the compulsory
insurance contract an employee from accidents. The maximum amount of
reimbursement of the costs of the policyholder and (or) the beneficiary
may not exceed 6 (six) percent of the paid insurance premium calculated
on the expiration date of the compulsory employee accident insurance
contract concluded between the policyholder and the insurer.

Reimbursement was carried out by the insurer to the current account of
the policyholder and (or) the beneficiary, opened in a second-tier bank
and indicated in the application, within 7 (seven) business days from
the date of receipt of the application and all documents.

{\bfseries Table 2 - Calculation according to case No. 1 on medical
rehabilitation (loss of professional ability to work from 30 to 59\%)}

% \begin{longtable}[]{@{}
%   >{\raggedright\arraybackslash}p{(\columnwidth - 2\tabcolsep) * \real{0.7814}}
%   >{\raggedright\arraybackslash}p{(\columnwidth - 2\tabcolsep) * \real{0.2186}}@{}}
% \toprule\noalign{}
% \begin{minipage}[b]{\linewidth}\raggedright
% {\bfseries Sources of payment}
% \end{minipage} & \begin{minipage}[b]{\linewidth}\raggedright
% {\bfseries Tenge}
% \end{minipage} \\
% \midrule\noalign{}
% \endhead
% \bottomrule\noalign{}
% \endlastfoot
% 1. {\bfseries Guaranteed volume of free medical care}, including
% {[}22-24{]}: & {\bfseries 49 080} \\
% 1.1 Emergency medical care (transportation to a medical institution) &
% \emph{18 000} \\
% 1.2.MRI of the shoulder joint & \emph{22 000} \\
% \begin{minipage}[t]{\linewidth}\raggedright
% \begin{enumerate}
% \def\labelenumi{\arabic{enumi}.}
% \setcounter{enumi}{2}
% \item
%   1.3. Examination by a doctor (primary) - surgeon
% \end{enumerate}
% \end{minipage} & \emph{2 700} \\
% \begin{minipage}[t]{\linewidth}\raggedright
% \begin{enumerate}
% \def\labelenumi{\arabic{enumi}.}
% \setcounter{enumi}{3}
% \item
%   1.4. Primary surgical treatment of the wound
% \end{enumerate}
% \end{minipage} & \emph{3 700} \\
% \begin{minipage}[t]{\linewidth}\raggedright
% \begin{enumerate}
% \def\labelenumi{\arabic{enumi}.}
% \setcounter{enumi}{4}
% \item
%   1.5. Novocaine blockade
% \end{enumerate}
% \end{minipage} & \emph{1 300} \\
% 1.6. Dressing & \emph{1 380} \\
% 2. {\bfseries Compulsory social health insurance}, including {[}25-31{]}: &
% {\bfseries 223 000} \\
% 2.1. Round-the-clock medical supervision, treatment, care, 10 days
% (12,000 tenge for 1 day, observation and care of a patient in a
% hospital) & \emph{120 000,00} \\
% 2.2. Provision of a bed with meals, 10 days
% 
% (4,200 for 1 day, stay in an inpatient ward) & \emph{42 000,00} \\
% 2.3. Physiotherapy (1,500 tenge per session) & \emph{15 000} \\
% 2.4. Massage (2500 tenge per session) & \emph{25 000} \\
% 2.5. Consultations with other specialists (Doctor' s
% consultation (first category) - 4200 tenge per 1 appointment) & \emph{21
% 000} \\
% {\bfseries 3. Сompulsory insurance of employees against accidents in the
% performance of their work (official) duties} &
% \begin{minipage}[t]{\linewidth}\raggedright
% \begin{enumerate}
% \def\labelenumi{\arabic{enumi}.}
% \setcounter{enumi}{1}
% \item
%   {\bfseries 053 000}
% \end{enumerate}
% \end{minipage} \\
% 3.3. Additional expenses caused by damage to the
% employee' s health (up to 500 MCI, where the MCI is 3,692
% tenge in 2024)
% 
% 3.4. Reimbursement of preventive measures to the policyholder (6\% of
% the insurance premium)3450000x6\%=207,000 tenge & \emph{1~846 000}
% 
% \emph{207 000} \\
% {\bfseries Total}: & {\bfseries 2 325 080} \\
% \end{longtable}

\emph{Source: compiled by the authors}

Case 2. Injury at the production site (covered by the loss of earnings
for 6 months by the insurance company, see Table 3).

An employee of a transport company, while performing his official duties
at the workplace (warehouse of goods), suffered a head injury as a
result of a fall of incorrectly fixed loads (goods / parts) of vehicles.

First aid was provided at the scene, which was provided by medical
workers (a health center paramedic, since the injury occurred on the
territory of the enterprise). Next, the victim was transported to a
medical facility. As a result of the accident, according to the
Rehabilitation Routing Scale, the patient was assigned 4 (Gross
dysfunction and disability), ICD 10 S01.9 {[}20{]}. Treatment was
provided in inpatient conditions, providing round-the-clock medical
supervision, treatment, care, as well as the provision of a bed with
meals.

After the transfer of the patient from the specialized department to the
rehabilitation department within the same medical organization,
physiotherapy and massage, as well as consultations with other
specialists, are prescribed according to indications. All these expenses
were covered within the framework of the CSHI.

At the company, at the scene of the accident, occupational safety and
health specialists took measures to organize an investigation of the
accident and prepare investigation materials, and notifications of the
incident were sent to a number of authorities. The accident was
classified by the commission conducting the investigation as an
industrial accident, as a result of which an industrial accident act was
issued. One copy of the act on the investigation of an industrial
accident, together with copies of the investigation materials, has been
sent to the executive body of the insurer (within three days). Upon
notification of an insured event that occurred during the period of
insurance coverage under the employee' s compulsory
accident insurance contract, the insurer immediately registered it and
submitted information on this insured event to the organization for the
formation and maintenance of a database in accordance with the
regulatory legal act of the authorized body for regulation, control and
supervision of the financial market and financial organizations.

As a result, a monthly insurance payment was assigned to the injured
employee as compensation for damage related to the loss of earnings
(income) by the employee in connection with the establishment of the
degree of loss of professional ability from 60 to 89 percent inclusive,
which is carried out by the insurer for a period of 6 months (where it
is necessary to take into account that the amount of average monthly
earnings (income), taken into account for the calculation to be
reimbursed for lost earnings (income) does not exceed ten times the
minimum wage, established for the relevant financial year by the law on
the republican budget, on the date of conclusion of the contract of
compulsory insurance of an employee against accidents.)

Compensation was carried out by the insurer to the current account of
the policyholder and (or) the beneficiary, opened in a second-tier bank
and indicated in the application after the fact of confirmation of the
insured event.

{\bfseries Table 3 - Calculation according to case No. 2 for medical
rehabilitation (taking into account the insurance
company' s coverage of the loss of earnings for 6
months)}

% \begin{longtable}[]{@{}
%   >{\raggedright\arraybackslash}p{(\columnwidth - 2\tabcolsep) * \real{0.7427}}
%   >{\raggedright\arraybackslash}p{(\columnwidth - 2\tabcolsep) * \real{0.2573}}@{}}
% \toprule\noalign{}
% \begin{minipage}[b]{\linewidth}\raggedright
% {\bfseries Sources of payment}
% \end{minipage} & \begin{minipage}[b]{\linewidth}\raggedright
% {\bfseries Tenge}
% \end{minipage} \\
% \midrule\noalign{}
% \endhead
% \bottomrule\noalign{}
% \endlastfoot
% 1. {\bfseries Guaranteed volume of free medical care}, including
% {[}22-24{]}: & {\bfseries 41 700} \\
% 1.1 Emergency medical care (transportation to a medical institution) &
% \emph{18 000} \\
% 1.2.MRI of the head & \emph{21 000} \\
% 1.3. Examination by a doctor (primary) -surgeon & \emph{2 700} \\
% 1.4. Primary surgical treatment of the wound & \emph{3 700} \\
% 1.5. Novocaine blockade & \emph{1 300} \\
% 1.6. Dressing & \emph{1 380} \\
% 2. {\bfseries Compulsory social health insurance}, including {[}25-31{]}: &
% {\bfseries 474 300} \\
% 2.1. Round-the-clock medical supervision, treatment, care, 24 days
% (12,000 tenge for 1 day, observation and care of a patient in a
% hospital) & \emph{288 000,00} \\
% 2.2. Provision of a bed with meals, 24 days (4,200 for 1 day, stay in
% the ward of the inpatient department) & \emph{100 800,00} \\
% 2.3. Physiotherapy (1,500 tenge per session) & \emph{22 500,00} \\
% 2.4. Massage (2500 tenge per session) & \emph{37 500,00} \\
% 2.5. Consultations with other specialists (Doctor' s
% consultation (highest category) - 5000 tenge for 1 appointment) &
% \emph{25 000,00} \\
% 3. Basic income (MCI 3,692 tenge in 2024; minimum wage 85,000 tenge in
% 2024, the legal limit is 850 000 tenge) : & {\bfseries 3~236 000} \\
% 4.1. Compensation for lost earnings for 6 months\textsuperscript{*}
% \textsuperscript{Approximately}
% 
% 4.2. Monthly insurance payment to an employee as compensation for damage
% at work for 6 months\textsuperscript{*} \textsuperscript{Approximately}
% 
% 4.3. Additional expenses caused by damage to the
% employee' s health (up to 750 MCI, where the MCI is 3,692
% tenge in 2024)
% 
% 4.4. Reimbursement of preventive measures to the policyholder (6\% of
% the insurance premium)3450000x6\%=207,000 tenge & \emph{510 000}
% 
% \emph{70 833}
% 
% \emph{2~769 000}
% 
% \emph{207 000} \\
% {\bfseries Total}: & {\bfseries 3~992 500} \\
% \end{longtable}

\emph{Source: compiled by the authors}

{\bfseries Conclusion.} Based on the findings of the study, several
conclusions can be drawn. There is a substantial body of research
dedicated to the specific aspects of industrial accidents and
occupational diseases in various countries, including an analysis of the
state' s role in this context. To examine the
state' s involvement, the legal framework of the Republic
of Kazakhstan is presented, along with case studies used for analysis.

Within the scope of this research, the constructed scenarios by the
authors revealed that the allocated amounts within the legally
established limits are insufficient to meet all the needs of those
affected. The developed set of scenarios allowed for a comparison of
market prices in Kazakhstan, considering the legally defined resources.

Overall, Kazakhstan demonstrates a commendable approach in its
policy-making, grounded on-The right to compensation is designed to
safeguard the rights and interests of individuals who have suffered harm
in the course of their employment. However, in order to identify
additional incentives for allocating financial resources towards
supporting victims, expanding infrastructure, and enhancing the quality
of rehabilitation centers across the country, it is essential to examine
the experiences of various nations. These measures should be aligned
with the most pressing issues in the realm of social security.

The role of the state in the development of medical rehabilitation in
Kazakhstan is one of the central, determining ones. First of all, we are
talking about the creation of favorable economic and political
conditions, which researchers attribute to the number of determining
factors in the development of medical rehabilitation services provided
within the framework of insurance coverage. In the context of economic
conditions, we are talking about the degree of state regulation of the
industry. The rules for reimbursement of expenses for preventive (or)
rehabilitation measures, social and (or) professional rehabilitation
measures include a fairly wide range of services, which today are quite
expensive services and goods on the market of Kazakhstan. According to
the researchers, the examples given in the cases are the minimum package
of what the victim will be able to receive and do not confirm sufficient
data of services for the recovery of the sick person and his successful
return to society. This study is only the initial stage, reflecting
market realities, and the data for a comparative analysis of the level
of adequate insurance coverage should still be studied in subsequent
studies.

\emph{{\bfseries Financing:} The scientific results were obtained within
the framework of program-targeted funding by the Ministry of Labor and
Social Protection of the Population of the Republic of Kazakhstan
(scientific and technical program No. BR22182673 «Transformation of the
state mechanism of social guarantees in respect of persons employed in
harmful working conditions in the modern context».}

{\bfseries References}

\begin{enumerate}
\def\labelenumi{\arabic{enumi}.}
\item
  Forastieri, V. Improving health in the workplace:
  ILO' s framework for action. -2014. URL:
  https://www.ilo.org/publications/improving-health-workplace-ilos-framework-action
\item
  WHO/ILO joint estimates of the work-related burden of disease and
  injury. 2000-2016: global monitoring report. -2021. URL:
  https://www.who.int/news/item/16-09-2021-who-ilo-almost-2-
  million-people-die-from-work-related-causes-each-year.
\item
  YU.N.Pronin, V.G.Prisenko. Medicinskaya reabilitaciya postradavshih na
  proizvodstve. -2013. URL:
  https://cyberleninka.ru/article/n/meditsinskaya-reabilitatsiya-postra-davshih-na-proizvodstve/viewer
  (data obrashhenija: 30.07.2024) {[}in Russian{]}
\item
  Abdalla S., Apramian S.S., Cantley L.F., Cullen M.R. Occupation and
  risk for injuries / In: Mock C.N., Nugent R., Kobusingye O., Smith
  K.R. (Eds.) // Injury Prevention And Environmental Health, 3rd edition
  The International Bank for Reconstruction and Development/The World
  Bank, Washington (DC). -2017. DOI 10.1596/978-1-4648-0522-6\_ch6
\item
  Rusli Bin N. Rising trend of work-related commuting accidents, deaths,
  injuries and disabilities in developing countries: a case study of
  Malaysia // Ind. Health. -Vol. 52 (4). -P. 275-277.
\end{enumerate}

\href{https://doi.org/10.2486/indhealth.52-275}{DOI
10.2486/indhealth.52-275}. 2014

\begin{enumerate}
\def\labelenumi{\arabic{enumi}.}
\setcounter{enumi}{5}
\item
  Vega-Calderón L., Almendra R., Fdez-Arroyabe P., Zarrabeitia M.T.,
  Santurtún A. Air pollution and occupational accidents in the Community
  of Madrid, Spain // Int. J. Biometeorol. -2021. -Vol. 65 (3). -P.
  429-436. DOI 10.1007/s00484-020-02027-3 Mar
\item
  Campolieti M, Gunderson MK, Smith JA. The effect of vocational
  rehabilitation on the employment outcomes of disability insurance
  beneficiaries: new evidence from Canada // IZA J Labor Pol. -2014.
  -Vol. 3(10). DOI
  \href{http://dx.doi.org/10.1186/2193-9004-3-10}{10.1186/2193-9004-3-10}
\item
  Fadyl JK, McPherson KM, Schlüter PJ, Turner-Stokes L. Factors
  contributing to work-ability for injured workers: literature review
  and comparison with available measures // Disabil Rehabil.
  -2010.-Vol.32(14). P.1173-1183. DOI 10.3109/09638281003653302
\item
  Selander J, Marnetoft S-U, Bergroth A, Ekholm J. Return to work
  following vocational rehabilitation for neck, back and shoulder
  problems: risk factors reviewed // Disabil Rehabil. -2002.-Vol.
  24(14). -P. 704-12. DOI 10.1080/09638280210124284
\item
  Eurostat (European Statistical Office. Accidents at work statistics1
  (ESAW). In: Fatal and non-fatal accidents at work, by sex, age groups,
  injury groups and NACE Rev. 2 economic sectors {[}hsw\_mi07{]}. 2022.
  URL:
  https://ec.europa.eu/eurostat/web/main/search/-/search/estatsearchportlet\_WAR\_estatsearchportlet\_INSTANCE\_bHVzuvn1SZ8J?\_estatsearchportlet\_WAR\_estatsearchportlet\_INSTANCE\_bHVzuvn1SZ8J\_pageNumber=1\&\_estatsearchportlet\_WAR\_estatsearchportlet\_INSTANCE\_bHVzuvn1SZ8J\_pageSize=11\&text=fatal+and+non-fatal+accidents+at+work\%2C+by+sex\%2C+age+groups\%2C+injury+
  (data obrashhenija: 30.03.2024).
\item
  Federal' nyj zakon ot 21.11.2011 № 323-FZ «Ob osnovah
  ohrany zdorov' ja grazhdan v Rossijskoj Federacii».
  2011. URL: http://www. consultant.ru/document/cons\_doc\_LAW\_121895/
  (data obrashhenija 04.05.2024). {[}in Russian{]}
\item
  Prikaz Ministerstva zdravoohranenija RF ot 31.07.2020 № 788n «Ob
  utverzhdenii Porjadka organizacii medicinskoj reabilitacii vzroslyh».
  2020. URL:
  https://normativ.kontur.ru/document?moduleId=1\&documentId=438476
  (data obrashhenija 04.05.2024) {[}in Russian{]}
\item
  Prikaz Ministra zdravoohranenija Respubliki Kazahstan ot 7 oktjabrja
  2020 goda № ҚR DSM-116/2020. Zaregistrirovan v Ministerstve justicii
  Respubliki Kazahstan 9 oktjabrja 2020 goda № 21381. 2020. URL:
  https://adilet.zan.kz/rus/docs/V2000021381. (data obrashhenija
  04.05.2024). {[}in Russian{]}
\item
  Prikaz Ministra zdravoohranenija Respubliki Kazahstan ot 7 aprelja
  2023 goda № 65. Zaregistrirovan v Ministerstve justicii Respubliki
  Kazahstan 10 aprelja 2023 goda № 32263. 2023. URL:
  https://adilet.zan.kz/rus/docs/V2300032263 . (data obrashhenija
  10.05.2024). {[}in Russian{]}
\item
  Prikaz Ministra truda i social' noj zashhity naselenija
  Respubliki Kazahstan ot 31 janvarja 2024 goda № 20. Zaregistrirovan v
  Ministerstve justicii Respubliki Kazahstan 2 fevralja 2024 goda №
  33965. 2024. URL: https://adilet.zan.kz/rus/docs/V2400033965. (data
  obrashhenija 10.05.2024). {[}in Russian{]}
\item
  Zakon respubliki Kazahstan. Ob objazatel' nom
  strahovanii rabotnika ot neschastnyh sluchaev pri ispolnenii im
  trudovyh (sluzhebnyh) objazannostej. 2024. URL:
  https://online.zakon.kz/Document/?doc\_id=1052939. (data obrashhenija
  11.05.2024). {[}in Russian{]}
\item
  Trudovoj kodeks Respubliki Kazahstan. 2024. URL:
  https://online.zakon.kz/Document/?doc\_id=38910832. (data obrashhenija
  12.05.2024). {[}in Russian{]}
\item
  Grazhdanskij kodeks Respubliki Kazahstan (osobennaja
  chast'). 2024. URL:
  https://online.zakon.kz/Document/?doc\_id=1013880. (data obrashhenija
  12.05.2024). {[}in Russian{]}
\item
  Zakon Respubliki Kazahstan Ob objazatel' nom
  social' nom strahovanii. 2023. URL:
  https://online.zakon.kz/Document/?doc\_id=32615593. (data obrashhenija
  12.05.2024). {[}in Russian{]}
\item
  Social' nyj kodeks Respubliki Kazahstan. 2024. URL:
  https://online.zakon.kz/Document/?doc\_id=36492598. (data obrashhenija
  12.05.2024). {[}in Russian{]}
\item
  Mezhdunarodnaja klassifikacija boleznej 10-go peresmotra (MKB-10).
  URL: https://mkb-10.com/ . (data obrashhenija 12.05.2024). {[}in
  Russian{]}
\item
  A. Ivanilova. Skol' ko stoit odin vyzov «skoroj
  pomoshhi» k pacientu v Kazahstane. 2017. URL:
  https://mk-kz.kz/articles/2017/08/19/skolko-stoit-odin-vyzov-skoroy-pomoshhi-k-pacientu-v-kazakhstane.html.
  (data obrashhenija 12.05.2024). {[}in Russian{]}
\item
  DOQ.kz. 2024. URL:
  \url{https://doq.kz/services/almaty/categories/mrt?gad_source=1&gclid=CjwKCAjw5ImwBhBtEiwAFHDZx1jdiACtpccbd7w4T7jNd9z0quYofYTaqJgjJZ7mmaOK3zG6MKI_xoCh2QQAvD_BwE}.
  (data obrashhenija 16.05.2024).
\item
  Gorodskaja polikltnika nomer 25. 2024. URL: https://www.gkp25.kz/plat.
  (data obrashhenija 12.04.2024). {[}in Russian{]}
\item
  KGP na PHV «Vostochno -- Kazahstanskij oblastnoj reabilitacionnyj
  centr». 2024. URL: https://www.vkorc.kz/ru/gobmp.html. (data
  obrashhenija 12.04.2024). {[}in Russian{]}
\item
  Ambulatornyj centr v g. Ust'-Kamenogorsk. 2024. URL:
  https://ambulatory.kz/napravleniya/fiziokabinet/. (data obrashhenija
  12.04.2024). {[}in Russian{]}
\item
  Reabilitacionnyj centr «Saqtasyn (Saktasyn)». 2024. URL:
  https://saqtasyn.103.kz/rubric/reabilitacionnye-centry/. (data
  obrashhenija 12.03.2024). {[}in Russian{]}
\item
  Reabilitacionnyj centr «Dos Medicus». 2024. URL:
  https://dosmedicus.kz/prejskurant-cen. (data obrashhenija 03.03.2024).
  {[}in Russian{]}
\item
  Gorodskaja klinicheskaja bol' nica №1. 2024. URL:
  https://pervaya.kz/ru/price. (data obrashhenija 03.03.2024). {[}in
  Russian{]}
\item
  Gorodskaja klinicheskaja bol' nica №4 Almaty. 2024.
  URL: https://gkb4.kz/ru/prejskurant-tsen-na-platnye-uslugi/ . (data
  obrashhenija 03.03.2024). {[}in Russian{]}
\item
  Tomas.kz. 2024. URL:
  https://tomas.kz/t/kostyli-10003/?ysclid=lu7be52wh8945261009. (data
  obrashhenija 03.03.2024).
\end{enumerate}

\emph{{\bfseries Information about the authors}}

Kurmanov A.M.-Candidate of Economic Sciences, CEO of «Republican
Research Institute for Occupational Safety and Health of the Ministry of
Labor and Social Protection of the Republic of Kazakhstan»,Astana,
Kazakhstan, e-mail:
\href{mailto:rniiot@rniiot.kz}{\nolinkurl{rniiot@rniiot.kz}};

Bekmagambetov A.B. - Candidate of Legal Sciences, Associate Professor,
Deputy Director General for Research of «Republican Research Institute
for Occupational Safety and Health of the Ministry of Labor and Social
Protection of the Republic of Kazakhstan», Astana, Kazakhstan, e-mail:
\href{mailto:adilet1979@mail.ru}{\nolinkurl{adilet1979@mail.ru}};

Sabidullina А.Е., Doctoral Student, Senior Researcher, Branch «South» of
RSE on the REM «Republican Research Institute for Occupational Safety
and Health of the Ministry of Labor and Social Protection of the
Republic of Kazakhstan», Almaty, Kazakhstan, e-mail:
sabidullina96@inbox.ru;

Yedilbayeva L.I., Candidate of Medicine, Leading Researcher, Branch
«South» of RSE on the REM «Republican Research Institute for
Occupational Safety and Health of the Ministry of Labor and Social
Protection of the Republic of Kazakhstan», Almaty, Kazakhstan, e-mail:
laura.ibragimovna@gmail.com

\emph{{\bfseries Сведения об авторах}}

Курманов А.М. -кандидат экономических наук, генеральный директор РГП на
ПХВ «Республиканский научно-исследовательский институт по охране труда
Министерства труда и социальной защиты населения Республики Казахстан»,
Астана, Казахстан, e-mail:
\href{mailto:rniiot@rniiot.kz}{\nolinkurl{rniiot@rniiot.kz}};

Бекмагамбетов А.Б., кандидат юридических наук, ассоциированный
профессор, заместитель генерального директора по научной работе РГП на
ПХВ «Республиканский научно-исследовательский институт по охране труда
Министерства труда и социальной защиты населения Республики Казахстан»,
Астана, Казахстан, e-mail:
\href{mailto:adilet1979@mail.ru}{\nolinkurl{adilet1979@mail.ru}};

Сабидуллина А.Е. - докторант, старший научный сотрудник филиала «Южный»
РГП на ПХВ «Республиканский научно-исследовательский институт по охране
труда Министерства труда и социальной защиты населения Республики
Казахстан», Алматы, Казахстан, e-mail: sabidullina96@inbox.ru;

Едильбаева Л.И. - кандидат медицинских наук, ведущий научный сотрудник
филиала «Южный» РГП на ПХВ «Республиканский научно-исследовательский
институт по охране труда Министерства труда и социальной защиты
населения Республики Казахстан», Алматы, Казахстан, e-mail:
laura.ibragimovna@gmail.com