\id{ҒТАМР 06.56.21}{https://doi.org/10.58805/kazutb.v.4.25-553}

\begin{articleheader}
\sectionwithauthors{Т.Б.Мукушев,Б.А.Жуматаева,Б.С. Сапарова,М.А.Алтынбеков, К.Д. Кожабергенова}{КӘСІПОРЫНДАРДАҒЫ БАСҚАРУ ЕСЕБІ МЕН БАҚЫЛАУ ӘДІСТЕМЕСІН ДАМЫТУ}

{\bfseries \textsuperscript{1}Т.Б.Мукушев\textsuperscript{\envelope },\textsuperscript{2}Б.А.Жуматаева, \textsuperscript{1}Б.С.
Сапарова, \textsuperscript{3}М.А.Алтынбеков,

\textsuperscript{2}К.Д. Кожабергенова }
\end{articleheader}
\begin{affiliation}

\textsuperscript{1}Л.Н.Гумилев атындағы Еуразия Ұлттық Университеті,
Астана, Қазақстан,

\textsuperscript{2}Қ.Құлажанова атындағы Қазақ технология және бизнес
университеті,. Астана, Қазақстан,

\textsuperscript{3}«ESIL University» Мекемесі, Астана, Қазақстан

\raggedright{\bfseries \textsuperscript{\envelope }}Корреспондент-автор: tolegen1986@mail.ru
\end{affiliation}

Бұл зерттеудің мақсаты коммерциялық ұйымдардағы басқарушылық есеп пен
бақылаудың әдістемесін әзірлеу болып табылады. Зерттеу жұмысында
басқарушылық есептің заманауи тәсілдері зерттеледі, қолданыстағы бақылау
әдістері мен олардың нарықтық экономикадағы тиімділігі
талданады.Әдістеме әдеби дереккөздерді жан-жақты талдауды, әртүрлі
коммерциялық ұйымдардың менеджерлерімен сауалнамалар мен сұхбаттар
түріндегі эмпирикалық зерттеулерді, сондай-ақ деректерді өңдеудің
статистикалық әдістерін қолдануды қамтиды.Зерттеудің негізгі нәтижелері
тиімді басқарушылық есеп пен бақылау бухгалтерлік есепті автоматтандыру
жүйелері және деректерді талдау сияқты инновациялық технологияларды
біріктіруді талап ететінін көрсетеді. Заманауи ақпараттық жүйелерді
пайдалану бухгалтерлік есептің ашықтығы мен дәлдігін айтарлықтай
арттыратыны анықталды, бұл өз кезегінде негізделген басқару шешімдерін
қабылдауды жеңілдетеді.Зерттеу сонымен қатар жүйелі аудиттер мен ішкі
шолуларды қоса алғанда, кешенді ішкі бақылау тәсілдерін енгізу
тәуекелдерді азайтуға және ұйымның қаржылық тұрақтылығын жақсартуға
көмектесетінін көрсетеді. Нәтижелердің тәжірибелік маңыздылығы оларды
коммерциялық ұйымдарда ресурстарды басқаруды жақсарту, шығындарды
оңтайландыру және бизнестің жалпы тиімділігін арттыру үшін қолдану
мүмкіндігінде жатыр. Әзірленген әдістеме әртүрлі салалардың
ерекшеліктеріне бейімделген неғұрлым жетілдірілген басқарушылық есеп пен
бақылау жүйелерін құруға негіз бола алады.Осылайша, бұл зерттеу
коммерциялық ұйымдардың бәсекеге қабілеттілігін арттыруға көмектесетін
жаңа тәсілдер мен құралдарды ұсына отырып, басқарушылық есептің теориясы
мен тәжірибесін дамытуға елеулі үлес қосады.

{\bfseries Түйін сөздер:} басқару есебі, бақылау, қаржылы есеп, есептілік,
бизнес, ақпарат, бизнес процес, стратегиялық тәсіл, операция, жоба.
\begin{articleheader}

{\bfseries РАЗВИТИЕ МЕТОДИКИ УПРАВЛЕНЧЕСКОГО УЧЕТА И КОНТРОЛЯ НА
ПРЕДПРИЯТИЯХ}

{\bfseries \textsuperscript{1}Т.Б.Мукушев\textsuperscript{\envelope },
\textsuperscript{2}Б.А.Жуматаева,\textsuperscript{1}Б.С. Сапарова, 
\textsuperscript{3}М.А.Алтынбеков,

\textsuperscript{2}К.Д. Кожабергенова}
\end{articleheader}

\begin{affiliation}

\textsuperscript{1}Евразийский Национальный Университет им. Л.Н.Гумилев,
Астана, Казахстан,

\textsuperscript{2}Казахский университет технологии и бизнеса им.
К.Кулажанова, Астана, Казахстан

\textsuperscript{3}Esil University, Астана, Казахстан,

e-mail:
\href{mailto:tolegen1986@mail.ru}{\nolinkurl{tolegen1986@mail.ru}}
\end{affiliation}

Целью данного исследования является разработка методологии
управленческого учета и контроля в коммерческих организациях. В
исследовательской работе изучаются современные методы управленческого
учета, анализируются существующие методы контроля и их эффективность в
условиях рыночной экономики.Методика включает комплексный анализ
литературных источников, эмпирические исследования в форме опросов и
интервью с руководителями различных коммерческих организаций, а также
использование методов статистической обработки данных.Основные
результаты исследования показывают, что эффективный управленческий учет
и контроль требуют интеграции инновационных технологий, таких как
системы автоматизации учета и анализа данных. Установлено, что
использование современных информационных систем существенно повышает
прозрачность и точность бухгалтерского учета, что, в свою очередь,
способствует принятию обоснованных управленческих решений.Исследование
также показывает, что внедрение комплексного подхода к внутреннему
контролю, включая регулярные аудиты и внутренние проверки, может помочь
снизить риски и улучшить финансовую стабильность
организации.Практическая значимость результатов заключается в
возможности их применения в коммерческих организациях для улучшения
управления ресурсами, оптимизации затрат и повышения общей эффективности
бизнеса. Разработанная методология может служить основой для создания
более совершенных систем управленческого учета и контроля,
адаптированных к специфике различных отраслей промышленности. Таким
образом, данное исследование вносит существенный вклад в развитие теории
и практики управленческого учета, предлагая новые подходы и инструменты,
способствующие повышению конкурентоспособности коммерческих организаций.

{\bfseries Ключевые слова:} управленческий учет, контроль, финансовый учет,
отчетность, бизнес, информация, бизнес процесс, стратегический подход,
операция, проект.
\begin{articleheader}

{\bfseries DEVELOPMENT OF METHODOLOGY OF MANAGEMENT ACCOUNTING AND CONTROL
AT ENTERPRISES}

{\bfseries \textsuperscript{1}T.B.Mukushev\textsuperscript{\envelope },
\textsuperscript{2}B.A.Zhumatayeva, \textsuperscript{1}B.S.Saparova,
\textsuperscript{3}M.A.Altynbekov, \textsuperscript{2}K. Kozhabergenova}
\end{articleheader}
\begin{affiliation}

\textsuperscript{1}Eurasian National University. L.N.Gumilyov, Astana,
Kazakhstan,

\textsuperscript{2}K.Kulazhanov Kazakh university of technology and
business, Astana, Kazakhstan,

\textsuperscript{3}Esil University, Astana, Kazakhstan,

e-mail:
\href{mailto:tolegen1986@mail.ru}{\nolinkurl{tolegen1986@mail.ru}}
\end{affiliation}

The purpose of this study is to develop the methodology of management
accounting and control in commercial organizations. Modern methods of
management accounting are studied in the research work, existing control
methods and their effectiveness in the market economy are analyzed.The
methodology includes a comprehensive analysis of literary sources,
empirical research in the form of surveys and interviews with managers
of various commercial organizations, as well as the use of statistical
data processing methods.The main findings of the study show that
effective management accounting and control require the integration of
innovative technologies such as accounting automation systems and data
analysis. It was found that the use of modern information systems
significantly increases the transparency and accuracy of accounting,
which in turn facilitates the adoption of informed management decisions.
The study also shows that implementing a comprehensive internal control
approach, including regular audits and internal reviews, can help reduce
risks and improve an organization' s financial stability.
The practical importance of the results lies in the possibility of their
application in commercial organizations to improve resource management,
optimize costs and increase overall business efficiency. The developed
methodology can serve as a basis for creating more advanced management
accounting and control systems adapted to the specifics of various
industries. Thus, this study makes a significant contribution to the
development of the theory and practice of management accounting,
offering new approaches and tools that help to improve the
competitiveness of commercial organizations.

{\bfseries Keywords:} management accounting, control, financial accounting,
reporting, business, information, business process, strategic approach,
operation, project.
\begin{multicols}{2}

{\bfseries Кіріспе.} Жаһандық және ішкі бәсекелестіктің күшеюі, жылдам
технологиялық прогресс, кәсіпкерлік қызметті әртараптандыру және
бизнес-процестердің күрделенуі жағдайында коммерциялық кәсіпорындарды
басқару айтарлықтай қайта құрылуда. Бұл өзгерістер бухгалтерлік есеп пен
бақылау жүйесін түбегейлі қайта қарауды және бейімдеуді қажет етеді.

Бәсекелестік ортада тиімді ұйымдастырылған басқарушылық есеп өте маңызды
болып табылады, өйткені ол басшылықты негізделген басқару шешімдерін
қабылдау үшін қажетті сенімді және өзекті ақпаратпен қамтамасыз етеді.
Бұл есеп жүйесі дәстүрлі қаржылық есептің кемшіліктерін жояды.

Көптеген елдердің заңнамалық базасында бағалы қағаздар эмитенттері үшін
ішкі бақылау комитеттерін құру міндетті талапқа айналды.

Тиімді басқарушылық бақылау қаржылық есеп берудің тұтастығын және
активтерді қорғауды қамтамасыз етуде ғана емес, сонымен қатар
бизнес-кәсіпорындардың табысты қызмет етуінде де маңызды рөл атқарады.
Ол ықтимал қателер мен тәуекелдерді анықтауға және жоюға, ресурстарды
пайдалануды оңтайландыруға және бизнес-процестердің тиімділігін
арттыруға бағытталған.

Сондықтан басқарушылық есеп пен бақылау әдістерін жетілдіру бойынша
теориялық негіздер мен практикалық ұсыныстарды әзірлеу және енгізу
корпоративтік басқарудың ең маңызды аспектілері болып табылады. Бұл
басқарудағы ашықтықты жақсартуға, акционерлер мен инвесторлар сияқты
мүдделі тараптарға жауапкершілікті арттыруға мүмкіндік береді және ұзақ
мерзімді перспективада кәсіпорынның тұрақты өсуі мен дамуын қамтамасыз
етеді. {[}1{]}.

Қазіргі динамикалық экономикалық жағдайда басқару есебі мен бақылау
әдістерін жетілдіру қажеттігі көп қырлы міндет болып табылады, ол мұқият
зерделеуді және өзгермелі нарық конъюнктурасына, технологиялық
жаңалықтарға бейімделуді және ашықтық пен басқару тиімділігіне қойылатын
талаптарды арттыруды талап етеді. Сауда кәсіпорнын басқаруды
оңтайландыру бүгінгі күні қатаң ішкі стандарттау және
бухгалтерлік-аналитикалық қолдау шеңберінде қалыптасатын сенімді
ақпаратты пайдаланумен тығыз байланысты. Қаржылық есептілікті бақылау
мен реттеуді және бухгалтерлік есеп жүйесінің қажеттіліктеріне
бейімдеуді қамтиды. Заманауи жағдайларда басқарушылық есеп пен ішкі
бақылау басқарудың барлық деңгейлерінде іргелі рөл атқарады, ол
негізделген шешімдердің жедел қабылдануын ғана емес, сонымен қатар
процестерді, ресурстарды және ұйымның ұзақ мерзімді перспективада
бәсекеге қабілеттілігін оңтайландыруды қамтамасыз етеді.

Зерттеудің мақсаты басқарудың әртүрлі қажеттіліктерін, экономикалық
қызметтің ерекшеліктерін және аналитикалық есептілікке қойылатын
талаптарды ескеретін басқарушылық есеп пен бақылау жүйесін құрудың
әдістемелік тәсілдерін әзірлеу болды.

Осы мақсатқа жету үшін келесі зерттеу міндеттері анықталды:

- тұжырымдамалық аппаратты нақтылау және осы процестердің құқықтық
реттелуіне терең баға беру үшін басқарушылық есеп пен бақылаудың мәнін
жан-жақты талдау.

- ұйымдық мақсаттарға қол жеткізуге ықпал ететін негізгі әдістер мен
құралдарды анықтай отырып, тиімді басқару үшін басқарушылық есеп пен
бақылаудың практикалық қолданылуын зерттеу.

{\bfseries Материалдар мен әдістер.} Зерттеу нәтижесінде кәсіпорынның
басқарушылық есеп пен бақылау жүйесін ұйымдастыру кезінде кездесетін
негізгі проблемалары мен шектеулері анықталды.

Негізгі мәселе - нақты салалық және ұйымдастырушылық сипаттамаларға
бейімделген әдістерді әзірлеуге бірыңғай көзқарастың жоқтығы.

Басқарушылық есептің көбінесе компанияның стратегиялық мақсаттарын
есепке алмайтындығына әкеледі, бұл оның басқару шешімдерін қабылдау үшін
практикалық құндылығын төмендетеді. Қазіргі таңда басқару есебі мен
бақылау әдістерін әзірлеу озық халықаралық тәжірибелерді талдауға және
оларды отандық кәсіпорындардың жағдайына бейімдеуге негізделді.

\emph{Әдебиеттерге шолу.} Заманауи басқару динамикасында басқарушылық
есеп пен ішкі бақылау басқарудың барлық деңгейлерінде басты рөл
атқарады, ол ұйымның тиімді жұмыс істеуі құрылатын негіз болып табылады.
Бухгалтерлік есеп, мониторинг, талдау, жоспарлау және реттеу функциялары
жалпы басқару жүйесіне біріктірілген. Сонымен қатар, басқарушылық есеп
пен бақылау барлық басқару функцияларын табысты жүзеге асырудың ажырамас
құралы болып табылады. Басқару есебі мен бақылаудың маңыздылығын түсіну
олардың экономикалық мәні мен ұйымды басқарудың стратегиялық және
операциялық аспектілеріне әсерін терең түсінуден басталады.

Қ.Т. Тайғашинованың зерттеулерінде Басқару есебі келесідей
түсіндіріледі: басқару есебі -- қаржылық ақпаратты анықтауды, өлшеуді,
жинауды, егжей-тегжейлі талдауды, құрылымдауды, түсіндіруді және
кейіннен ұсынуды қамтитын жүйелі процесс. Бұл ақпарат ұйымның жоғары
басшылығына тиімді жоспарлауды жүзеге асыру, ағымдағы нәтижелерді
бағалау және стратегиялық және операциялық мақсаттардың орындалуын
бақылау үшін қажет. {[}2{]}.

Ж.Қ. Нұрғазинаның айтуынша басқару есебі кәсіпорынның ақпараттық
жүйесімен интегралды түрде әрекеттеседі, тек осы жүйенің бөлігі ретінде
ғана емес, сонымен қатар негізгі қызмет ретінде де әрекет етеді. Оның
негізгі мақсаты -- ұйым басшылығын стратегияларды әзірлеу, жедел
басқару, оның қызметінің барлық аспектілерінің тиімділігін бақылау және
бағалау үшін қажетті ақпаратпен қамтамасыз ету. {[}3{]}.

В. М. Родионова и В. Я. Шлейников пікірлері бойынша ішкі бақылау --
шаруашылық жүргізуші субъекті немесе басқару органы сыртқы әсерге
қарамастан өз мүддесі үшін өз жұмысын өз бетінше тексеру және бағалау
үшін атқаратын ішкі функция. {[}4{]}.

В.Е. Керимов бойынша кәсіпорынның ішкі бақылау негізгі параметрлер мен
бақылау объектілерін анықтау, қателер, бұрмалаулар және басқа да
жағымсыз салдарлардың ықтималдығы ең жоғары «сыни» нүктелерді анықтау
жүзеге асырылатын жүйе ретінде қабылданады. {[}5{]}.

Біздің пікірінше, басқарудың ішкі бақылауы -- ықтималдық көзқарасқа
негізделген және ашықтық, күрделілік және қатаң реттеудің жоғары
дәрежесімен сипатталатын күрделі және динамикалық жүйе. Бұл жүйе бақылау
ортасы, тәуекелді бағалау процесі, ақпараттық жүйе, бақылау қызметі,
бақылау, мониторинг, есепке алу, талдау, бақылау процедуралары және
жоспарлау сияқты көп функциялы элементтерді қамтиды.

{\bfseries Нәтижелер мен талқылау.} Қаржылық және басқарушылық есеп,
сондай-ақ оларды бақылау әдістері заңнамалық реттеудің бөлігі болып
табылады. Қазіргі уақытта осы процестерге арналған стандарттар мен
ережелерді белгілеуге жалпы қабылданған көзқарас жоқ. Коммерциялық
ұйымның стратегиясын табысты жүзеге асыру басқарушылық есеп жүйесі
шеңберінде құрылған сыртқы және ішкі пайдаланушылардың қажеттіліктерін
ескеретін нақты ақпаратсыз мүмкін емес.

Басқару есебі қаржылық есепке тән шектеулерді еңсеру, негізделген
басқару шешімдерін қабылдау үшін қажетті ақпаратты қалыптастырудың
негізгі құралы болып табылады.

Қазіргі шаруашылық жағдайында басқару есебінің әдіснамасын жетілдіру өте
күрделі және көп деңгейлі процесске айналады, оған әртүрлі сыртқы және
ішкі факторлар әсер етеді және жан-жақты және терең талдауды қажет
етеді. Қазіргі жағдайда бухгалтерлік есеп пен аналитикалық жүйеге
біріктірілген есеп түрлерінің әртүрлілігі есеп саясатын жасау міндетін
қояды. Осыған байланысты басқару есебін ұйымдастыру және жүргізу
әдістерінде елеулі айырмашылықтар туындайды. Сондықтан әртүрлі
ғалымдардың теориялық көзқарастары мен эмпирикалық деректерін терең
талдауға негізделген басқару есептерін құрудың әдіснамалық тәсілдерін
зерттеу тиімді басқаруды және стратегиялық жоспарлауды қамтамасыз ету
үшін ерекше мәнге және өзектілікке ие болады (кесте 1).
\end{multicols}


\begin{longtable}[H]{|@{} 
    >{\raggedright\arraybackslash}p{(\columnwidth - 2\tabcolsep) * \real{0.3116}}| 
    >{\raggedright\arraybackslash}p{(\columnwidth - 2\tabcolsep) * \real{0.6884}}|@{}}
    \caption*{1-кесте. Әр түрлі ғалымдар ұсынған басқару есебін құру
    кезеңдері}\\
    \hline
  \begin{minipage}[b]{\linewidth}\raggedright
  {\bfseries Ғалым}
  \end{minipage} & \begin{minipage}[b]{\linewidth}\raggedright
  {\bfseries Басқару есебінің кезеңдері}
  \end{minipage} \\ 
  \endhead
  \hline
  \endfoot
  \hline
  1 & 2 \\
  Роберт Каплан и Дэвид Нортон & 1. Стратегиялық мақсаттарды анықтау \\
  & 2. Көрсеткіштер жүйесін жасау \\
  & 3. Жүйені жүзеге асыру \\
  & 4. Бақылау және бақылау \\
  \hline
  Анри Файоль & 1. Жоспарлау \\
  & 2. Ұйымдастыру \\
  & 3. Бақылау \\
  & 4. Үйлестіру \\
  & 5. Қарым-қатынас \\
  \hline
  Джон Шанк и Вижай Говиндараджан & 1. Мәліметтерді жинау \\
  & 2. Мәліметтерді өңдеу және талдау \\
  & 3. Ақпаратты көрсету \\
  & 4. Шешім қабылдау \\
  \hline
  Майкл Портер & 1. Құн тізбегін талдау \\
  & 2. Негізгі факторларды анықтау \\
  & 3. Процесті оңтайландыру \\
  \hline
  Гордон Шиллинглоу & 1. Бухгалтерлік есеп объектілерін анықтау \\
  & 2. Өлшеу және мәліметтерді жинау \\
  & 3. Мәліметтерді өңдеу және талдау \\
  & 4. Шешім қабылдау \\
  & 5. Бақылау және реттеу \\
  \hline
  Чарльз Хорнгрен & 1. Мәліметтерді жинау және классификациялау \\
  & 2. Мәліметтерді өңдеу \\
  & 3. Талдау және түсіндіру \\
  & 4. Есепті дайындау \\
  & 5. Шешім қабылдау \\
  \hline
  Джеймс Х. Донован & 1. Жоспарлау \\
  & 2. Шығындарды басқару \\
  & 3. Тиімділікті бағалау \\
  & 4. Бақылау және есеп беру \\
  \hline
  Эрик Кохен & 1. Мәліметтерді жинау \\
  & 2. Мәліметтерді өңдеу және сақтау \\
  & 3. Деректерді талдау \\
  & 4. Есепті дайындау \\
  & 5. Басқару үшін ақпаратты пайдалану \\
  \hline
  \end{longtable}
  
  
\begin{multicols}{2}

Бұл кезеңдер әртүрлі ғалымдардың әдіснамалық тәсілдеріне байланысты
әртүрлі болуы мүмкін, бірақ олардың барлығы басқару есебін
оңтайландыруға және алынған ақпарат негізінде шешім қабылдау процесін
жетілдіруге бағытталған.

Талдау басқарудың ақпараттық қажеттіліктерін, ұйымдастырылған шаруашылық
қызметінің ерекшеліктерін және реляциялық аналитикалық жүйелердің
құрылымын ескеретін біркелкі әдістердің жоқтығын анықтады. Авторлар
қарастырған әдістерді қолданылатын тәсілдер мен әдістемелерге байланысты
негізгі үш топқа бөлуге болады.

Әдістердің бірінші тобы стратегиялық және тактикалық деңгейде басқару
шешімдерін қабылдауға бағытталған стратегиялық көзқарасқа негізделген.
Ол стратегиялық аудиттер жүргізу және компанияның миссиясын анықтау
арқылы жалпы бизнес өнімділігін жақсартуға көмектеседі.

Әдістердің екінші тобы тактикалық басқаруға бағытталған және әдетте
компанияның стратегиясын егжей-тегжейлі шолуды немесе оны түзетуді
қамтымайды.

Көптеген аудиторлық және консалтингтік компаниялар басқару есебіне
бағытталған өз қызметінің бөлігі ретінде әдістердің үшінші тобын
белсенді пайдаланады. Мұндай жобалардың негізгі мақсаты -- уақыт, еңбек,
қаржылық мүмкіндіктер және т.б. сияқты ресурстық шектеулерді ескере
отырып, алдын ала белгіленген нәтижелерге қол жеткізу.

Басқару есебін енгізу орындау стратегиясын әзірлеуді, тәуекелдерді және
жоспарлардан ауытқуларды азайтуды және өзгерістерді тиімді басқаруды
қамтиды. Тәжірибеде ұйымды басқаруда оңтайлы нәтижелерге қол жеткізу
үшін жобалау әдістерін технологиялық тәсілмен біріктіру жиі қолданылады.

Ішкі бақылау жүйесін құрудың әдістері мен тәсілдерін зерттеу ұйымды
басқарудың тиімділігін қамтамасыз етудің маңызды кезеңі болып табылады.
Қазіргі бизнес жағдайында ішкі бақылау нақты қаржылық есептілікті
қамтамасыз етуде, активтерді қорғауда, нормативтік талаптарды сақтауда
және тәуекелді басқаруда маңызды рөл атқарады.

Ішкі бақылау жүйесін құрудың негізгі әдістеріне стратегиялық және
тактикалық тәсілдер жатады {[}14{]}.

Стратегиялық тәсіл ұзақ мерзімді жоспарлауға және ұйымның мақсаттарына
жетуге бағытталған саясаттар мен процедураларды әзірлеуге бағытталған.
Ол стратегиялық мақсаттар мен тәуекелдерді анықтауды, басқару жүйесінің
құрылымын әзірлеуді және басқару процестерін реттеу және жақсарту үшін
кері байланыс жүйесін құруды қамтиды.

Басқарушылық есеп жүйесін дамыту тиімді ішкі бақылау жүйесін құрумен
байланысты мәселелермен ажырамас байланысты.

Ішкі бақылауды басқару тәсілдерін талдау екі негізгі бағытты --
стратегиялық және тактикалық бағытты анықтады.

Бұл кесте тиімді басқаруды және нормативтік талаптардың сақталуын
қамтамасыз ету үшін ұйымдар қолдана алатын ішкі бақылау жүйесін құрудың
негізгі тәсілдерін ұсынады. Басқару есебі мен бақылау жүйесін құру
операциялық, функционалдық және стратегиялық басқару деңгейлерін қамтуы
керек (2 кесте).
\end{multicols}

\begin{longtable}[]{|@{}
    >{\raggedright\arraybackslash}p{(\columnwidth - 2\tabcolsep) * \real{0.2823}}|
    >{\raggedright\arraybackslash}p{(\columnwidth - 2\tabcolsep) * \real{0.7177}}|@{}}
    \caption*{2 - кесте. Ішкі бақылау жүйесін құру әдістері мен тәсілдеріне
    шолу}\\
    \hline
  \begin{minipage}[b]{\linewidth}\raggedright
  Тәсіл
  \end{minipage} & \begin{minipage}[b]{\linewidth}\raggedright
  Сипаттама
  \end{minipage} \\ \hline
  \endhead
  \hline
  \endfoot
  Стратегиялық & Ұзақ мерзімді стратегиялар мен саясаттарды әзірлеуге
  бағытталған. Ұйымның стратегиялық мақсаттарын анықтауды, тәуекелдерді
  бағалауды және тәуекелдерді басқару жүйесін әзірлеуді қамтиды. Басқару
  жүйесінің құрылымын құруды, саясаттар мен процедураларды әзірлеуді және
  басқару процестерін реттеу және жақсарту үшін кері байланыс жүйесін
  қамтамасыз етуді қамтиды. \\ \hline
  Тактикалық & Ұйымның күнделікті қызметінде стратегиялық шешімдерді
  жүзеге асыруға бағытталған нақты әдістер мен процедураларға назар
  аударады. Пайдалану процедураларын әзірлеуді, бақылау әрекеттерін,
  өнімділікті талдауды және сәйкессіздіктерді шешуді қамтиды. \\ \hline
  Кешенді & Теңдестірілген тәуекелдерді басқаруға, операциялық тиімділікке
  және ұйымның стратегиялық мақсаттарына қол жеткізу үшін стратегиялық
  және тактикалық тәсілдерді біріктіру. \\ 
  \end{longtable}
  
  \begin{multicols}{2}

Басқару үш деңгейге бөлінеді -- оперативтік, тактикалық және
стратегиялық, олардың әрқайсысы басқарушылық есеп пен ішкі бақылау
жүйесінде өзіндік рөл атқарады.

Жедел басқару ұйымның күнделікті міндеттері мен ағымдағы мақсаттарына
назар аударады. Ол жоспарлардың орындалуын, ағымдағы процестерді
бақылауды және ағымдағы деректер негізінде жедел шешім қабылдауды
қамтамасыз етеді.

Тактикалық басқару орта мерзімді стратегияларды әзірлеумен және жүзеге
асырумен айналысады. Бұл өндірістік және қаржылық нәтижелерді талдау
және бағалау, нарықтық өзгерістерге бейімделу және ресурстарды
оңтайландыру деңгейі.

Стратегиялық басқару ұзақ мерзімді мақсаттарға бағытталған және
компанияның миссиясы мен даму бағыттарын анықтайды. Ол қаржылық
жоспарлауды, ресурстарды басқаруды және болашаққа тәуекелді бағалауды
қамтиды.

Басқару есебі мен ішкі бақылау жүйесі контекстінде оперативтік,
тактикалық және стратегиялық басқаруды талдауды жалғастыру олардың өзара
байланысын және ұйымның тиімділігіне әсерін тереңірек түсінуге мүмкіндік
береді.

Операциялық басқару -- ағымдағы деректер мен жедел ақпарат негізінде
басқару шешімдері қабылданатын бастапқы деңгей. Басқарудың бұл деңгейі
ұйымның күнделікті жұмыс істеуіне бағытталған және өндіріске, сатуға,
сатып алуға және т.б. байланысты күнделікті операцияларды қамтиды.
Басқару есебі контекстінде операцияларды басқару ағымдағы нәтижелерді
бақылау және дереу түзетулер енгізу үшін деректерді пайдаланады.

Тактикалық басқару ұйым қызметінің неғұрлым стратегиялық аспектілеріне
назар аударады. Бұл деңгей нақты мақсаттарға жетуге бағытталған орта
мерзімді жоспарлар мен тактикаларды әзірлеу мен жүзеге асыруды қамтиды.
Басқару есебі контекстінде тактикалық менеджмент орта мерзімді даму
стратегиясы туралы шешім қабылдау, нарықтық өзгерістерге бейімделу және
ресурстарды оңтайландыру үшін деректерді талдау мен түсіндіруге
бағытталған.

Стратегиялық менеджмент ұйымның ұзақ мерзімді дамуының бағыттарымен және
жалпы стратегияны қалыптастырумен айналысады. Басқарудың бұл деңгейі
компанияның миссиясын, құндылықтарын және ұзақ мерзімді мақсаттарын
анықтайды, сонымен қатар жаңа нарықтарға, инновацияларға және дамудың
жалпы бағытына шығудың стратегияларын әзірлейді. Басқару есебінде
стратегиялық менеджмент ұзақ мерзімді жоспарлау, қаржылық тұрақтылықты
бағалау, болашақ нәтижелерді болжау және ұзақ мерзімді мақсаттарға жету
үшін стратегияларды әзірлеу үшін деректерді пайдаланады.

Басқарудың осы деңгейлерінің әрқайсысында ішкі бақылау маңызды рөл
атқарады. Операциялық деңгейде ол операциялық қызметтің дәлдігі мен
сенімділігін қамтамасыз етуге, тактикалық деңгейде стратегиялық
жоспарлардың сақталуын және ресурстарды пайдалану тиімділігін бақылауға,
ал стратегиялық деңгейде тәуекелдерді барынша азайтуға және ұзақ
мерзімді талаптардың сақталуын қамтамасыз етуге бағытталған.

Осылайша, ұйымның тұрақты дамуын қамтамасыз ету, оның мақсаттарына қол
жеткізу және бәсекеге қабілеттілікті арттыру үшін барлық деңгейдегі
тиімді басқару және ішкі бақылау - жедел, тактикалық және стратегиялық
маңызды.

{\bfseries Қорытынды.} Қорытындылай келе, басқарушылық есеп пен ішкі
бақылау ұйымды барлық деңгейде: жедел, тактикалық және стратегиялық
деңгейде тиімді басқаруда шешуші рөл атқаратынын атап өтуге болады.
Операцияларды басқару ағымдағы операцияларға назар аударады және
операциялық шешімдер қабылдау үшін дәл және уақтылы деректерді талап
етеді. Тактикалық басқару стратегиялық мақсаттарға жету үшін деректерді
талдау негізінде орта мерзімді жоспарлар мен тактикаларды әзірлеумен
айналысады. Стратегиялық менеджмент компанияның ұзақ мерзімді
перспективаға жалпы стратегиясы мен даму бағытын анықтайды. Ішкі
бақылау, өз кезегінде, барлық деңгейдегі басқару функцияларын дұрыс
орындауды, тәуекелдерді азайтуды және белгіленген стандарттар мен
саясаттардың сақталуын қамтамасыз етеді. Ол басқару шешімдері
қабылданатын ақпараттың сенімділігі мен дұрыстығын қамтамасыз ететін
басқару жүйесінің құрамдас бөлігі болып табылады.

Осылайша, басқарудың барлық деңгейлерінде басқарушылық есеп пен ішкі
бақылауды біріктіру ұйымға өз ресурстарын тиімді пайдалануға, және ұзақ
мерзімді перспективада тұрақты дамуға ықпал ете отырып, стратегиялық
мақсаттарға жетуге мүмкіндік береді.

Басқару есебі мен бақылау үшін ең жақсы тәсіл жобаның элементтерін және
ұйымдағы басқару мен бақылаудың тұтас көрінісін қамтамасыз ететін
стратегиялық тәсілдерді біріктіретін біріктірілген әдіс болып табылады.

Коммерциялық ұйымдардағы басқарушылық есеп пен бақылаудың әртүрлі
аспектілерін талдау барысында бұл процестер тиімді басқаруды және
стратегиялық шешімдерді қабылдауды қамтамасыз етуде шешуші рөл
атқаратыны анықталды. Басқару есебі мен бақылау әдіснамасын әзірлеу тек
стратегиялық жоспарлауды ғана емес, операциялық тәуекелдерді басқаруды,
қаржылық ақпаратты талдауды және ішкі бақылаудың жүйелі аудитін қамтитын
кешенді тәсілді қажет етеді.

Зерттеу негізінде келесі ұсыныстарды жасауға болады:

- әдістеме әзірлеу кезінде ұйымның нақты қажеттіліктеріне бейімделген
есеп пен бақылаудың әртүрлі әдістерін ескеру қажет. Тиімді басқару
мақсаттарына жету үшін басқарудың стратегиялық, тактикалық және
операциялық деңгейлерін біріктіру маңызды;

- басқару есебі мен бақылаудың заманауи ақпараттық технологияларын
енгізу жинақтау, талдау және есеп беру процесін жетілдіруге мүмкіндік
береді. Басқару шешімдерін қабылдаудың дәлдігі мен тиімділігін арттыру
үшін басқарудың ақпараттық жүйелерін пайдалану ұсынылады;

- әдістемені тиімді енгізу үшін сауатты және білікті кадрлар қажет.
Қызметкерлерді басқарушылық есеп пен бақылау мәселелері бойынша тұрақты
оқытуды қамтамасыз ету, сондай-ақ осы процестерге инвестициялау
қажеттілігі туралы басшылықтың хабардарлығын арттыру ұсынылады.

- әдістемені үздіксіз жетілдіру және ұйымның стратегиялық мақсаттарына
қол жеткізу үшін басқарушылық есеп пен бақылаудың тиімділігін бағалау
жүйесін енгізу маңызды.

Осылайша, басқару есебі мен бақылау әдіснамасын әзірлеу және табысты
енгізу коммерциялық ұйымдарды қазіргі заманғы басқарудың маңызды
элементі болып табылады. Кәсіпкерлік қызметтің ерекшеліктерін ескере
отырып, кешенді тәсілді қолдану және заманауи технологияларды енгізу
динамикалық өзгеретін экономикалық жағдайда ұйымның тиімді жұмыс істеуін
қамтамасыз етуге көмектеседі.
\end{multicols}

\begin{center}
	{\bfseries Әдебиеттер}
	\end{center}
	
	\begin{references}

1. М.А. Алтынбеков, Г.М. Сагиндыкова, Г.С. Түсібаева Кәсіпорында басқару
есебін ұйымдастыру: оқулық.- Алматы: LEM баспасы, 2023.- 304б. ISBN:
978-601-239-742-0

2. Тайгашинова~К.Т.~ Проблемы формирования и развития экологического
учета и аудита. Теория и практика / монография.~К.Т.Тайгашинова~--
Алматы: ТОО «Жания-Полиграф», 2022. -- 379 с. ISBN 978-601-269-203-7

3. Байдақов А., Алтынбеков М.А. Басқару есебіндегі шығындар есебінің
мәселелері және оларды топтастыру // Вестник «КазУЭФМТ». -- Нур --
Султан -- 2023 --№ 1(50) -- С.~72-79

4. Сембиева Л.М. Введение~в~финансы~: учебное пособие / Л.М. Сембиева,
С.Б. Макыш, А.О. Жагыпарова. - Алматы : Эпиграф, 2020. ISBN
978-601-342-204-6.

5. Керимов В. Э. Бухгалтерский управленческий учет: Практикум для
бакалавров. --- 11-е изд., перераб. / В.Э. Керимов. - Москва: Дашков и
К, 2021. - 96 с. ISBN 978-5-394-04113-6.~

6. Стратегический управленческий~учет~: учебное пособие / Б.Ж. Акимова,
А.О. Махамбетова, Л.Ж. Айтхожина, А.А. Кажмухаметова. - Алматы :
TechSmith, 2023. - 252, {[}1{]} с. : ил., табл. - Библиогр.: с. 246-252.
ISBN 978-601-352-296-8.~

7. Мочалова Л.А. Стратегический анализ и планирование: учебник / Л.А.
Мочалова, В.И. Власов; Уральский государственный горный университет. -
2-е изд. - Москва : Ай Пи Ар Медиа , 2024. - 167, {[}1{]} с. : ил.,
табл. - Библиогр.: с. 153-155. ISBN 978-5-4497-1853-2.

8. Джолдасбаева Г.К. Управление~затратами~на предприятии: учебное
пособие / Г.К. Джолдасбаева. - 2-е изд., перераб. и доп. - Караганда :
АҚНҰР, 2017. - 167 c. ISBN 978-601-7053-43-7.

9. Клишевич Н.Б. Финансы организаций:~менеджмент~и анализ : учебное
пособие для студентов, обучающихся по специальностям "Финансы и кредит",
"Бухгалтерский учет, анализ и аудит" / Н.Б. Клишевич. - Москва: КНОРУС,
2016. - 303 с. ISBN 978-5-406-04851-1

10 Акимова Б.Ж. Стратегический~управленческий~учет: учебное пособие /
Б.Ж. Акимова, А.О. Махамбетова, Л.Ж. Айтхожина, А.А. Кажмухаметова. -
Алматы : New book, 2021. - 252с. ISBN 978-601-352-296-8.

11Хорнгрен~Ч.Т. Бухгалтерский учет: управленческий аспект /
Ч.Т.~Хорнгрен, Фостер Дж.; пер. с англ. под ред. Я.В.Соколова. - Москва
: Финансы и статистика, 2001. - 416 с. - (Серия по бухгалтерскому учету
и аудиту). ISBN 5-279-01212-2.

12. Мархаева Б.А. Управленческий~учет~2: учебное пособие / Б.А.
Мархаева, А.Д. Каршалова; Министерство образования и науки Республики
Казахстан, УО "Алматы менеджмент университет". - Алматы : Балауса, 2017.
- 198 с. ISBN 978-601-7470-77-7.

13. Мархаева Б.А. Управленческий~учет~: учебное пособие : с
интеллект-картами / Б.А. Мархаева; НОУ "Алматы Менеджмент Университет".
- 2-е изд., доп. и перераб. - Алматы : Алматы Менеджмент Университет,
2015. - 343 с. ISBN 978-601-7529-29-1.~

14. Мизиковский И. Е. Управленческий учет и защита учетной информации:
тесты: Учебное пособие / Мизиковский И. Е., Милосердова А. Н., Ясенев В.
Н. - М.: Магистр, НИЦ ИНФРА-М, 2016. - 112 с. ISBN 978-5-9776-0303-4

15. Нургазина Ж.К. Управленческий~учет: учебник / Ж.К. Нургазина;
Министерство образования и науки Республики Казахстан. - Алматы:
Ассоциация вузов РК, 2014. - 411 с. ISBN 978-601-289-126-3.

\end{references}

\begin{center}
{\bfseries References}
\end{center}

\begin{references}

1. M.A. Altynbekov, G.M. Sagindykova, G.S. Tүsіbaeva Kәsіporynda basқaru
esebіn ұjymdastyru: oқulyқ.- Almaty: LEM baspasy, 2023.- 304b. ISBN:
978-601-239-742-0 {[}in Kazakh{]}.

2. Tajgashinova K.T. Problemy formirovaniya i razvitiya ekologicheskogo
ucheta i audita. Teoriya i praktika / monografiya. K.T.Tajgashinova --
Almaty: TOO «ZHaniya-Poligraf», 2022. -- 379 s. ISBN 978-601-269-203-7

3. Bajdaқov A., Altynbekov M.A. Basқaru esebіndegі shyғyndar esebіnің
mәselelerі zhәne olardy toptastyru // Vestnik «KazUEFMT». -- Nur --
Sultan -- 2023 --№ 1(50) -- S. 72-79 {[}in Kazakh{]}.

4. Sembieva L.M. Vvedenie v finansy : uchebnoe posobie / L.M. Sembieva,
S.B. Makysh, A.O. ZHagyparova. - Almaty : Epigraf, 2020. - ISBN
978-601-342-204-6. {[}in Russian{]}.

5. Kerimov V. E. Buhgalterskij upravlencheskij uchet: Praktikum dlya
bakalavrov. --- 11-e izd., pererab. / V.E. Kerimov. - Moskva: Dashkov i
K, 2021. - 96 s. ISBN 978-5-394-04113-6 {[}in Russian{]}.

6. Strategicheskij upravlencheskij uchet : uchebnoe posobie / B.ZH.
Akimova, A.O. Mahambetova, L.ZH. Ajthozhina, A.A. Kazhmuhametova. -
Almaty : TechSmith, 2023. - 252, {[}1{]} s. : il., tabl. - Bibliogr.: s.
246-252. ISBN 978-601-352-296-8 {[}in Russian{]}.

7. Mochalova L.A. Strategicheskij analiz i planirovanie: uchebnik / L.A.
Mochalova, V.I. Vlasov; Ural' skij gosudarstvennyj gornyj
universitet. - 2-e izd. - Moskva : Aj Pi Ar Media , 2024. - 167, {[}1{]}
s. : il., tabl. - Bibliogr.: s. 153-155. ISBN 978-5-4497-1853-2 {[}in
Russian{]}.

8. Dzholdasbaeva G.K. Upravlenie zatratami na predpriyatii: uchebnoe
posobie / G.K. Dzholdasbaeva. - 2-e izd., pererab. i dop. - Karaganda :
AҚNҰR, 2017. - 167 c. ISBN 978-601-7053-43-7 {[}in Russian{]}.

9. Klishevich N.B. Finansy organizacij: menedzhment i analiz : uchebnoe
posobie dlya studentov, obuchayushchihsya po
special' nostyam "Finansy i kredit", "Buhgalterskij
uchet, analiz i audit" / N.B. Klishevich. - Moskva: KNORUS, 2016. - 303
s. ISBN 978-5-406-04851-1 {[}in Russian{]}.

10 Akimova B.ZH. Strategicheskij upravlencheskij uchet: uchebnoe posobie
/ B.ZH. Akimova, A.O. Mahambetova, L.ZH. Ajthozhina, A.A.
Kazhmuhametova. - Almaty : New book, 2021. - 252s. ISBN
978-601-352-296-8 {[}in Russian{]}.

11Horngren CH.T. Buhgalterskij uchet: upravlencheskij aspekt / CH.T.
Horngren, Foster Dzh.; per. s angl. pod red. YA.V.Sokolova. - Moskva :
Finansy i statistika, 2001. - 416 s. - (Seriya po buhgalterskomu uchetu
i auditu). ISBN 5-279-01212-2 {[}in Russian{]}.

12. Marhaeva B.A. Upravlencheskij uchet 2: uchebnoe posobie / B.A.
Marhaeva, A.D. Karshalova; Ministerstvo obrazovaniya i nauki Respubliki
Kazahstan, UO "Almaty menedzhment universitet". - Almaty : Balausa,
2017. - 198 s. ISBN 978-601-7470-77-7 {[}in Russian{]}.

13. Marhaeva B.A. Upravlencheskij uchet : uchebnoe posobie: s
intellekt-kartami / B.A. Marhaeva; NOU "Almaty Menedzhment Universitet".
- 2-e izd., dop. i pererab. - Almaty : Almaty Menedzhment Universitet,
2015. - 343 s. ISBN 978-601-7529-29-1 {[}in Russian{]}.

14. Mizikovskij I. E. Upravlencheskij uchet i zashchita uchetnoj
informacii: testy: Uchebnoe posobie / Mizikovskij I. E., Miloserdova A.
N., YAsenev V. N. - M.: Magistr, NIC INFRA-M, 2016. - 112 s ISBN
978-5-9776-0303-4 {[}in Russian{]}.

15. Nurgazina ZH.K. Upravlencheskij uchet: uchebnik / ZH.K. Nurgazina;
Ministerstvo obrazovaniya i nauki Respubliki Kazahstan. - Almaty:
Associaciya vuzov RK, 2014. - 411 s. ISBN 978-601-289-126-3 {[}in
Russian{]}.
\end{references}

\begin{authorinfo}
\hspace{1em}\emph{{\bfseries Авторлар туралы мәлімет}}

Мукушев Т.Б. {\bfseries -} «Мемлекеттік ауддит» кафедрасының докторанты,
Л.Н.Гумилев ат Еуразия Ұлттық Университеті, Астана, Қазақстан, e-mail:
\href{mailto:Tolegen1986@mail.ru}{\nolinkurl{Tolegen1986@mail.ru}};

Жуматаева Б.А.{\bfseries -} PhD, қауымдастырылған профессор, Қ.Құлажанова
атындағы Қазақ технология және бизнес университеті, Астана, Қазақстан,
e-mail:
\href{mailto:bahyt_jumataeva@mail.ru}{\nolinkurl{bahyt\_jumataeva@mail.ru}};

Сапарова Б.С.- PhD, қауымдастырылған профессор, Л.Н.Гумилев ат Еуразия
Ұлттық Университеті, Астана, Қазақстан\emph{,} e-mail:
\href{mailto:mtbb1986@gmail.com}{\nolinkurl{mtbb1986@gmail.com}};

Алтынбеков М.А.PhD, қауымдастырылған профессор, «ESIL University»
Мекемесі, Астана, Қазақстан, e-mail:
\href{mailto:Everest-astana@mail.ru}{\nolinkurl{Everest-astana@mail.ru}}.
Кожабергенова К.Д. - Техника ғылымдарының кандидаты, қауымдастырылған профессор, Қ.Құлажанова  атындағы Қазақ технология және бизнес университеті,
Астана, Қазақстан, e-mail: Kala 08@listl.ru.

\hspace{1em}\emph{{\bfseries Information about the authors}}

Mukushev T.B. - Doctoral student of the ``State Audit" Department, L.N.
Gumilyov Eurasian National University, Astana, Kazakhstan,
e-mail:\href{mailto:Tolegen1986@mail.ru}{\nolinkurl{Tolegen1986@mail.ru}};

Zhumatayeva B.A. - PhD, Associate professor, Kazakh University of
Technology and Business named after K.Kulazhanov, Astana, Kazakhstan,
e-mail:
\href{mailto:bahyt_jumataeva@mail.ru}{\nolinkurl{bahyt\_jumataeva@mail.ru}};

Saparova B. S. - PhD, Associate professor, L.N. Gumilyov Eurasian
National University, Astana. Kazakhstan, e-mail:
\href{mailto:mtbb1986@gmail.com}{\nolinkurl{mtbb1986@gmail.com}};

Altynbekov M.A. - PhD, Associate professor, ESIL University, Astana.
Kazakhstan, e-mail:
\href{mailto:Everest-astana@mail.ru}{\nolinkurl{Everest-astana@mail.ru}}.

Kozhabergenova K. - Candidate of Technical Sciences, Associate Professor, K.Kulazhanov Kazakh University of Technology and Business,
 Astana, Kazakhstan, e-mail: Kala 08@listl.ru.
\end{authorinfo}

