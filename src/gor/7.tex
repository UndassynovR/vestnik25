
\newpage
{\bfseries ҒТАМР 52.47.15}

{\bfseries КҮРДЕЛІ ЖАҒДАЙЛАРДА ҰҢҒЫМАЛАРДЫ БҰРҒЫЛАУҒА АРНАЛҒАН ҚАЛДЫҚТАРҒА
НЕГІЗДЕЛГЕН БҰРҒЫЛАУ ЕРІТІНДІЛЕРІ}

{\bfseries \textsuperscript{1}С.М. Ерменов, Г.М.
\textsuperscript{2}Эфендиев,
\textsuperscript{1}А.С.Садырбаева\textsuperscript{\envelope },
\textsuperscript{1}М.К. Жантасов, \textsuperscript{1}С.Е. Байботаева}

\textsuperscript{1}М.Әуезов атындағы Оңтүстік Қазақстан университеті,
Шымкент, Қазақстан,

\textsuperscript{2}Әзірбайжан Республикасының Мұнай және газ институты,
Баку,Әзірбайжан,

{\bfseries \textsuperscript{\envelope }}Корреспондент-автор: a.sadyrbaeva@mail.ru

Бұрғылау жұмыстарының тиімділігі мен сапасын арттыру мәселесі жаңа
мұнай-газ аумақтарын ашуды және белгілі аудандарда үлкен тереңдікті
игеруді көздейді. Үлкен тереңдікке ұңғымаларды бұрғылау, әдетте, тау-кен
геологиялық жағдайларының күрделенуінен туындаған айтарлықтай
қиындықтарды тудырады, бұл бірінші кезекте жоғары және төмен сұйықтық
(флюид) қысымы бар аралықтарды қамтиды. Ұңғымаларды күрделі жағдайларда
бұрғылау тиісті бұрғылау ерітінділерін қолдануды талап етеді. Заманауи
бұрғылау ерітінділері әртүрлі қымбат химиялық реагенттер мен
материалдардан тұратын қымбат, көп компонентті жүйелер болып табылады.

Ғылыми жарияланымдарда көрсетілген мәліметтер ұңғымаларды бұрғылауда
өнеркәсіптік қалдықтардың технологиялық функцияларында үнемділікті,
экологиялылықты сақтауға мүмкіндік беретін және географиялық тұрғыдан
тиімді болатын түрлерін қолдануға байланысты мәселелердің жеткіліксіз
зерттелгенін көрсетеді. Бұл мәселені әдеби тұрғыдан талқылаудың
жеткіліксіздігі зерттеуді тұжырымдау мен жүзеге асыруда белгілі бір
қиындықтар туғызады. Осыған сүйене отырып, бұл мақала әртүрлі
қалдықтарды бұрғылау ерітінділеріне қоспалар ретінде пайдалану бойынша
зерттеулерге шолу жасауға арналған.

{\bfseries Түйін сөздер:} ұңғымаларды бұрғылау, бұрғылау ерітіндісі,
реологиялық қасиеттері, ұңғыма оқпанының тұрақтылығы, сазды ерітінді,
соапсток.

{\bfseries БУРОВЫЕ РАСТВОРЫ НА ОСНОВЕ ОТХОДОВ ДЛЯ БУРЕНИЯ СКВАЖИН В
ОСЛОЖНЕННЫХ УСЛОВИЯХ}

{\bfseries \textsuperscript{1}С.М. Ерменов, \textsuperscript{2}Г.М.
Эфендиев, \textsuperscript{1}А.С.Садырбаева\textsuperscript{\envelope },
\textsuperscript{1}М.К. Жантасов, \textsuperscript{1}С.Е.Байботаева}

\textsuperscript{1}Южно-Казахстанский университет им.М.Ауэзова, Шымкент,
Казахстан,

\textsuperscript{2}Институт нефти и газа Азербайджанской
Республики,Баку, Азербайджан,

e-mail: a.sadyrbaeva@mail.ru

Проблема повышения эффективности и качества буровых работ предполагает
открытие новых нефтегазоносных территорий и освоение все больших глубин
в известных районах. Бурение скважин на большие глубины, как правило,
сопряжено с существенными трудностями, вызванными усложнением
горно-геологических условий, к которым, в первую очередь, относятся
интервалы с аномально высокими (АВПД) и низкими давлениями (АНПД)
флюидов. Бурение скважин в осложненных условиях требует использования
соответствующих буровых растворов. Современные буровые растворы
представляют собой дорогостоящие, многокомпонентные системы с большим
содержанием различных дорогостоящих химических реагентов и материалов.

Сведения, отраженные в научных публикациях, свидетельствуют о
недостаточной проработке вопросов, связанных с применением в бурении
скважин промышленных отходов, которые позволили бы при их
технологических функциях сохранить экономичность, экологичность и были
бы выгодными с географической точки зрения. Недостаточность литературной
проработки данного вопроса создает определенные трудности при постановке
и реализации исследований. Исходя из этого, настоящая статья посвящена
обзору исследований различных отходов на предмет использования их в
качестве добавок к буровым растворам.

{\bfseries Ключевые слова:} бурение скважин, буровой раствор; реологические
свойства, устойчивости ствола скважины, глинистый раствор, соапсток.

{\bfseries WASTE-BASED DRILLING FLUIDS FOR DRILLING WELLS IN DIFFICULT
CONDITIONS}

{\bfseries S.M. Yermenov, G.M. Afandiyev,
A.S.Sadyrbayeva\textsuperscript{\envelope }, M.K. Zhantasov, S.E. Baibotaeva}

\textsuperscript{1}M.Auezov South Kazakhstan University, Shymkent,
Kazakhstan,

\textsuperscript{2}Institute of oil and gas of the Republic of
Azerbaijan, Baku, Azerbaijan,

e-mail:
\href{mailto:a.sadyrbaeva@mail.ru}{\nolinkurl{a.sadyrbaeva@mail.ru}}

The problem of improving the efficiency and quality of drilling
operations involves the discovery of new oil and gas-bearing territories
and the development of ever greater depths in known areas. Drilling
wells to great depths, as a rule, is associated with significant
difficulties caused by the complication of mining and geological
conditions, which, first of all, include intervals with abnormally high
and low fluid pressures. Drilling wells in complicated conditions
requires the use of appropriate drilling fluids. Modern drilling fluids
are expensive, multicomponent systems with a high content of various
expensive chemicals and materials.

The information reflected in scientific publications indicates
insufficient elaboration of issues related to the use of industrial
waste in drilling wells, which would allow their technological functions
to preserve efficiency, environmental friendliness and would be
beneficial from a geographical point of view. The lack of literary
elaboration of this issue creates certain difficulties in the
formulation and implementation of research. Based on this, this article
is devoted to a review of studies of various wastes for their use as
additives to drilling fluids.

{\bfseries Keywords:} drilling of wells, drilling mud; rheological
properties, stability of the borehole, clay solution, soapstock.

{\bfseries Кіріспе.} Бұрғылау жұмыстарының тиімділігі мен сапасын арттыру
мәселесі жаңа мұнай-газ аумақтарын ашуды және белгілі аудандарда үлкен
тереңдікті игеруді көздейді. Үлкен тереңдікке ұңғымаларды бұрғылау,
әдетте, тау-кен геологиялық жағдайларының күрделенуінен туындаған
айтарлықтай қиындықтарды тудырады, бұл бірінші кезекте жоғары және төмен
сұйықтық (флюид) қысымы бар аралықтарды қамтиды. Күрделі жағдайларда
ұңғымаларды бұрғылау тиісті бұрғылау ерітінділерін қолдануды талап
етеді. Заманауи бұрғылау ерітінділері әртүрлі қымбат химиялық реагенттер
мен материалдардан тұратын қымбат, көп компонентті жүйелер болып
табылады.

Егерде тапшы шикізат болып табылатын 1 тонна қымбат реагенттердің құнын
ескеретін болсақ, оларды арзан реагентпен алмастыру бұрғылау
ерітінділерін дайындау шығындарын, сондай-ақ, бұрғылау құнын едәуір
төмендетеді, нәтижесінде ұңғымаларды салудың техникалық-экономикалық
тиімділігін арттырады. Осыған байланысты қалдықтар негізінде бұрғылау
ерітінділерінің жаңа құрамдарын жасау өзекті мәселе болып табылады.

Өнеркәсіптік қалдықтарға негізделген бұрғылау ерітінділерін пайдалану
мәселесі осы уақытқа дейін жеткілікті түрде өңделмеген. Күрделі
жағдайларда ұңғымаларды бұрғылау тиісті бұрғылау ерітінділерін қолдануды
талап ететіндігі белгілі. Заманауи бұрғылау ерітінділері әртүрлі құнды
химиялық реагенттер мен материалдардан тұратын қымбат, көп компонентті
жүйелер болып табылады. Егер тапшы шикізат болып табылатын 1 тонна
қымбат реагенттердің құнын ескеретін болсақ, оларды арзан реагентпен
ішінара алмастыру бұрғылау ерітінділерін дайындау шығындарын, сондай-ақ
бұрғылау құнын едәуір төмендетеді, нәтижесінде ұңғымаларды бұрғылаудың
техникалық-экономикалық тиімділігін арттырады.

Қол жетімді, арзан, экологиялық таза бұрғылау ерітінділері негізінде
ерітінділердің жаңа құрамдарын әзірлеу қажет. Осы уақытқа дейін
жинақталған зерттеулер өнеркәсіптік қалдықтарды қолдану мүмкіндіктерін
бағалау бойынша әрекеттер жасалынуда. Алайда, бұл бағыттағы әдеби
ақпараттың жеткіліксіздігі оларды бұрғылау ерітінділеріне қоспалар
ретінде сәтті қолдану мүмкіндігін шектейтінін атап өткен жөн. Сондықтан,
белгілі бір қалдықтарды зерттеуге кіріспес бұрын, ғылыми әдебиеттерде
көрсетілген қолдану нәтижелерімен танысу және талдау қажет. Жарияланған
жұмыстарға шолу кейінгі зерттеу жұмыстарының тұжырымы мен бағытын
негіздеуге мүмкіндік береді.

Ұсынылып отырған мақалада жоғарыда келтірілгендерге сүйене отырып, әдеби
материалдар мен оларда баяндалған ережелерді талдау негізінде терең
ұңғымаларды бұрғылау үшін қалдықтарды шикізат ретінде қолдануға
негізделген бұрғылау ерітінділерінің құрамын таңдау мәселесінің қазіргі
жағдайын бағалау негізгі мақсат болып табылады. Мақала шолу түрінде
болғандықтан, оның негізгі міндеті -- күрделі жағдайларда ұңғымаларды
сәтті бұрғылау үшін бұрғылау ерітінділерінің құрамы мен түрін таңдаудағы
ғылыми мәселенің зерттелу деңгейін көрсету, қарастырылып отырған мәселе
бойынша осы уақытқа дейін жинақталған зерттеулерге сыни баға беру және
осы жұмыстарға негізделген қорытындылар жасау.

Ұңғымаларды бұрғылау кезіндегі қиындықтар. Бұрғылау кезінде мамандардың
алдында тұрған негізгі мәселелердің бірі - бұрғылау ерітінділерін
сіңіру, ұңғыма оқпанының тұрақтылығын жоғалту. Сондықтан бұрғылау
ерітінділерінің құрамы мен параметрлерін таңдағанда, алдымен ұңғымаларды
бұрғылау процесі жүзеге асырылатын геологиялық жағдайлар бағаланады.

{\bfseries Материалдар мен әдістер.} Ұңғымада болатын механизмдер мен
процестер тұрғысынан ұңғыма қабырғаларының тұрақтылығын жоғалтуға
байланысты қиындықтар магистральдық аймақтың жыныстарында шекті
жағдайдың пайда болуы нәтижесінде болады. Геологиялық жағдайларды және
олардың теріс процестерге әсерін шамамен былайша жіктеуге болады:
ашылатын қабаттардың жоғары өткізгіштігі, табиғи қабаттардың ашылуы және
жаңа жарықтардың пайда болуы, кавернаның пайда болуы, ұңғыма оқпанының
тарылуы, жыныстардың пайда болуы, төгілуі, құлауы болып табылады {[}1,
2, 3{]}. Зерттеушілер бұл құбылыстардың себептерін көбінесе механикалық
деп бөледі (мысалы, жоғары кернеулерге, тау жыныстарының беріктігінің
төмендігіне немесе бұрғылаудың дұрыс емес тәжірибесіне байланысты
ұңғыманың айналасындағы тау жыныстарының бұзылуы), тау жыныстары, әдетте
саздар мен бұрғылау ерітіндісі арасындағы бұзылатын өзара әрекеттесу
нәтижесінде пайда болатын химиялық әсерлер. Көбінесе тұрақсыздық
жағдайлары химиялық және механикалық факторлардың жиынтығының нәтижесі
болып табылады. Бұл мәселе ұңғымада ауыр қиындықтарды тудырады және
кейбір жағдайларда қымбат салдарға әкелуі мүмкін. Кен орындарын игеруді
жоспарлау кезеңінде ұңғыма оқпанының тұрақтылығын талдауға деген
қажеттіліктің артуы экономикалық ойларға және үлкен ауытқуы бар
ұңғымаларды, сондай-ақ, көлденең ұңғымаларды кеңінен қолдануға
байланысты болады. Бірқатар жұмыстарда ұңғыма оқпанының тұрақсыздығының
жіктелуі, себептері, көрсеткіштері және диагностикасы, сондай-ақ, ұңғыма
оқпанының кернеулі күйінің моделі келтірілген {[}4, 5, 6, 7{]}.

Аталған себептерге байланысты әртүрлі қиындықтар бұрғылау жылдамдығының
және жалпы техникалық-экономикалық көрсеткіштердің төмендеуіне әкеліп
соқтырады, олардың салдарын жоюға көп қаражат жұмсауды талап етеді,
сондықтан бұл мәселе әрдайым зерттеушілердің назарында болады.
Қиындықтарды болжау мәселесінің шешімдерінің сенімділігі ұңғыманың
бөлінуі туралы геологиялық ақпараттың толықтығы мен сенімділігіне де,
ықтимал қиындықтардың пайда болуын болжауды әдістемелік қамтамасыз ету
деңгейіне де байланысты болады.

Тау жыныстары зерттеу нысаны ретінде ұңғыманы бұрғылау басталғанға дейін
тау және қабат қысымынан туындаған күрделі шиеленісті күйде болады.
Ұңғыманың тау жыныстарын ашуы оны қоршап тұрған массивтегі кернеулердің
өзгеруімен қатар жүреді. Сонымен қатар, ең үлкен өзгерістер ұңғыманың
қабырғасын құрайтын тау жыныстарында байқалады. Уақыт бойынша тұрақты
табиғи кернеулер бұрғылау ерітіндісінің ауыспалы қысымымен ауыстырылады,
оның температурасы ұңғыманы қоршап тұрған тау жыныстарының
температурасына тең емес және уақыт бойынша да өзгереді. Сонымен қатар,
бұрғылау ерітіндісінің қабырға жыныстарымен және қанықтыратын
сұйықтықпен физикалық-химиялық әрекеттесуі болады. Мұның барлығы
ұңғыманың қабырғаларын құрайтын тау жыныстарының механикалық
қасиеттерінің уақыт өте келе өзгеруіне әкеледі. Мысалы, тау жыныстарының
беріктік сипаттамалары ұзақ мерзімді беріктіктің өзгеру заңдылықтарына
сәйкес өзгереді.

Тау жыныстары біртекті еместігімен ерекшеленеді, бұл оның барлық
сипаттамаларының кең өзгеруіне әкеледі. Сондықтан оңтайлы технологиялық
шешімдерді іздеуді тек тау жыныстарының қасиеттерін қолдана отырып
жүргізуге болады.

Бұрғылау кезіндегі қиындықтармен шектелген қима аралықтары, әдетте,
кеуекті жыныстардан тұрады. Тау жыныстарының қабаттық (кеуекті) қысымы
мен кеуектілігінің ұңғыманы ашу кезіндегі әрекет ету тәсілдеріне әсері
айқын және көптеген зерттеушілер мойындайды, бірақ олардың әсерін есепке
алу әдістемелік тұрғыдан пысықталмаған. Көптеген жұмыстар ұңғымалардың
тұрақтылығын зерттеу мәселелеріне арналған. Атап айтқанда, жұмыстарда
{[}7,8{]} ұңғыманың геомеханикалық сипаттамалары мен тұрақтылығын
есептеудің математикалық негіздемесі келтірілген. Бұрғылау мәселелерін
шешу үшін геомеханикалық модель құру процесі жұмыста сипатталған. Тау
жыныстарының қасиеттері көбінесе бұрғылау кезінде проблемалардың
себептерінің бірі болып табылады, бұл уақыттың тиімсіз шығындарына,
қаражаттың жоғалуына, кейде тіпті ұңғыманы жоюға әкеледі.

Жалпы, жарияланған жұмыстарға негізделген тұрақтылықтың жоғалуына
әкелетін себептерді келесідей жүйелеуге және қорытындылауға болады
{[}4{]}.

Келесі факторлар тобы ұңғыма қабырғаларының тұрақтылығының бұзылуына
әкеледі: геологиялық, физика-химиялық және техникалық-технологиялық.
Геологиялық факторларға мыналар жатады: кернеудің күйі, құрылымы және
литологиясы, қабаттағы сұйықтықтардың болуы, қабаттық (кеуекті) қысым,
тау жыныстарының физикалық-механикалық қасиеттері мен пайда болу
жағдайлары (қабаттасу бұрышы); физика-химиялық осмостық және капиллярлық
ылғал тасымалдау, қосымша сыну қысымдарының дамуымен бірге жүретін тау
жыныстарының беткі ылғалдануы және ақырында техникалық-технологиялық:
бұрғылау ерітіндісінің құрамы мен параметрлері, бұрғылау әдісі және
бұрғылау режимінің параметрлері, ерітіндінің көтерілу жылдамдығы,
зениттік және азимуттық бұрыштардың мәндері, ұңғыманың иілу
қарқындылығы, тұрақсыз жыныстардың болу ұзақтығы. Белгілі болғандай,
ұңғыма қабырғаларының тұрақтылығына әсер ететін негізгі факторлардың
бірі - бұрғылау ерітіндісінің параметрлері болып табылады. Кейбір
жағдайларда ұңғымаларды жоюдың себебі болып табылатын бұрғылау
процесіндегі ауыр қиындықтар, ұлттық экономикаға айтарлықтай материалдық
зиян келтіретін барлық дерлік бұзылулар бұрғылау ерітінділерінің
сапасының төмендігіне, жеткіліксіздігіне, ал кейбір жағдайларда
қасиеттерді басқарудың сенімді әдістері мен құралдарының болмауына
байланысты болуы мүмкін.

Бұл ретте тығыздық қабат қысымы мен гидрожару қысымының градиенттерімен
салыстырғанда тығыздық мәндеріндегі шектеулерді сақтау шарттарына сүйене
отырып есептелуі тиіс; осылайша, ерітіндінің шамадан тыс тығыздығы
ұңғыманың механикалық жылдамдығын төмендетуге, гидрожаруға және т.б.
ықпал етуі мүмкін.

Әр түрлі қалдықтарға негізделген бұрғылау ерітінділерін қолдану
тәжірибесі.

Ұңғымаларды бұрғылау кезіндегі маңызды міндеттердің бірі - ұңғыманы салу
циклі бойына бұрғылау ерітінділерінің құрамы мен қасиеттерін бақылау
болып табылады. Бұл, атап айтқанда, әртүрлі қиындықтардың, мұнай-газ
көріністерінің, тау жыныстарының гидравликалық жарылуының, дұрыс емес
химиялық өңдеумен жұтылудың, бұрғылау ерітіндісінің тығыздығын дұрыс
бағаламаудың қаупімен байланысты.

Мұнай-газ өнеркәсібі қоршаған ортаны ластау көзі болып табылатын
бұрғылау жұмыстарында пайдаланылған бұрғылау ерітіндісін, қабат суын
және жинақталған бұрғылау шламын көп мөлшерде өндіреді. Экологиялық
қауіпсіздіктің заманауи мәселелері балама көп функциялы биологиялық
ыдырайтын және экологиялық таза бұрғылау ерітіндісі қоспаларын зерттеу
мен пайдалануды ынталандырады. Осы мәселелерді айналып өту үшін
биоөнімдерді қолдану қолға алынған.

Бұл шолу түріндегі мақала зерттеушілерге және мұнай-газ өнеркәсібіне
қолғабыс көрсету үшін бұрғылау ерітіндісінде кейбір «жасыл»
биоөнімдердің қолданылуын көрсету үшін ұсынылған. Орындалған жұмыстарда
бұл биоөнімдердің үнемді бола отырып, бұрғылау ерітіндісінің қасиеттерін
жақсартуға ықпалын көрсетеді. {[}9{]} сәйкес, бұрғылау ерітіндісін
дайындамаудан алдын, биоөнімдер кептіріліп, ұнтақ күйіне дейін
ұсақталған. Бұрғылау ерітіндісінің дұрыс қоспасы қажетті мақсатқа жету
үшін бұрғылау ерітіндісінің қасиеттерін алмастыра алатын қабілеттілікке
ие, яғни оңтайлы пластикалық тұтқырлыққа және гельдің беріктігі мен
аққыштығының жақсы аралығына ие.

Сонымен қатар, осы {[}9{]} жұмыста айтылғандай, құрма дәнектері, шөп
және шөп күлі бұрғылау ерітіндісінің реологиялық қасиеттерін басқаруға
мүмкіндік беретін тамаша модификаторлар болып табылады. Авторлар
{[}10{]} қолданатын концентрациялар 350 мл судағы 22,5 г бентониттен су
негізіндегі бұрғылау ерітіндісіндегі әрбір қоспаның 0,25, 1,0, 1,5 және
2,0 ppb құрайды. Мақалада бингамның пластикалық моделін дәлелдейтін
құрма, құрамында шөп және шөп күлі бар бентонитті бұрғылау ерітіндісі
консистенциясының қисығы берілген. Құрма тұқымдары аққыштық шегін
арттырмай, гельдің пластикалық тұтқырлығы мен беріктігін арттыруға
көмектеседі.

Осыған байланысты мақалада {[}1{]} мақта гудронына негізделген
ингибиторлық реагенттің бұрғылау ерітіндісінің реологиялық қасиеттеріне
әр түрлі пайыздық әсерін анықтау бойынша зерттеулер келтірілген. Мақта
гудроны негізіндегі реагент концентрациясының бұрғылау ерітіндісінің
реологиялық көрсеткіштеріне (динамикалық ығысу кернеуі және шартты
тұтқырлық) әсерін зерттеу жүргізілді, олардың талдауы массаның 4\%-на
тең реагенттің оңтайлы құрамын анықтауға мүмкіндік берді және бұрғылау
ерітіндісіндегі үлкен концентрацияда айналым жүйесіндегі гидравликалық
кедергілері артып, бұрғылаудың механикалық жылдамдығы төмендейтіні
көрсетілген.

Зерттелетін жаңа реагенттің концентрациясынан саз ерітінділерінің
реологиялық параметрлерінің өзгеруі, авторлар атап өткендей, сызықтық
тәуелділікке ие. Алайда, бұл нәтижелер алдын ала бола отырып, қосымша
зерттеулерді қажет етеді. Бұл мақта гудронына негізделген реагенттің
сипатын анықтауға мүмкіндік береді. Белгіленген жұмыста осы негіздегі
сұйықтықтар псевдопластикалық сипатта болады деп алдын ала айтылған.
Ұқсас сұйықтықтар сәйкес {[}2{]} ұңғыманың оқпанын жарылған жыныстардан
тиімді тазартуды және ұңғымадағы қысымның минималды жоғалуын қамтамасыз
етеді. Жұмыста {[}6{]} бұрғылау ерітінділерін дайындау үшін авторлар
өздері жасаған композициялар мен күрделі реагент алу технологиясын
ұсынады.

Майлы дақылдарды өңдеу процесінде (мақта, мақсары, соя, күнбағыс, балық
өнеркәсібінің қалдықтары) тазарту сатысында, авторлар атап өткендей,
сұйық қалдықтардың едәуір мөлшері -- соапстоктар түзіледі. Олар май
өндірісінің біріккен ағынға төгіледі, бұл әр түрлі салалар үшін жанама
шикізат ресурстарына айналуы мүмкін болатын құнды компоненттердің едәуір
мөлшерін жоғалтуға әкеледі. Кейбір жағдайларда сабын өндіру үшін май
қышқылдарын алу үшін соапстоктар дистилляцияланады. Сонымен қатар, май
қышқылдарын дистилляциялаудың қалдықтары немесе гудрондар (мақта майын
өңдеу жағдайында -- госсиполды шайыры) қалады. Бұл қалдықтарды кәдеге
жаратудың мүмкін бағыттарының бірі -- оларды бұрғылау жуу сұйықтықтарын
дайындау үшін кешенді әсер ететін химиялық реагенттерді алу үшін негізгі
шикізат ретінде пайдалану. Қазіргі уақытта Қазақстанда күрделі
геологиялық жағдайларда ұңғымаларды бұрғылау үшін қолданылатын химиялық
реагенттер тиімсіз және өте қымбат. Бұрғылау жуу сұйықтықтарының
қасиеттерін жақсарту үшін негізінен улы, экологиялық зиянды және оларды
қолдану шарттарын толық ескермейтін минералды шыққан реагенттер
қолданылады. Жергілікті шикізат пен өндіріс қалдықтары негізінде
бұрғылау сұйықтықтарын дайындау үшін тиімділігі жоғары, импортты
алмастыратын және арзан композициялық химиялық реагенттерді алу
технологияларын игеру өзекті мәселе болып табылады.

Тиімділігі жоғары химиялық реагенттерді пайдалану бұрғылау жуу
сұйықтықтарын дайындау және олардың қасиеттерін реттеу шығындарының
төмендеуін қамтамасыз етеді, құрылымдық-реологиялық, сүзу, коррозияға
қарсы және трибологиялық қасиеттерін жақсартады, сондай-ақ, мұнай және
газ ұңғымалары мен қатты пайдалы қазбаларды бұрғылау ұңғымаларын
бұрғылау кезіндегі қиындықтардың алдын алу мәселелерін шешеді. Айта кету
керек, зерттеу процесінде зерттеушілердің назары майларды қоспалар
ретінде алу кезінде қалдықтарды қолдану, сондай-ақ, техникалық жартылай
фабрикаттарды тазарту сатысында балық майын өндіру нәтижесінде пайда
болды {[}7{]}. Бұл жағдайда сұйық қалдықтардың едәуір мөлшері ‒
соапстоктар түзіледі. Осы уақытқа дейін оларды кәдеге жарату мәселесі
шешілген жоқ, ол түпкілікті шешілген деп саналмайды және қазіргі
уақытта, бұл өз кезегінде {[}2{]} атап өткендей, балық өнеркәсібі үшін
жанама материалдық ресурстарға айналуы мүмкін май қалдықтарының құнды
компоненттерінің едәуір мөлшерін жоғалтуға әкеледі.

Дайын өнімді пайдаланудың ықтимал бағыттарын айқындай отырып, құрамында
майы бар қалдықтарды өңдеудің жаңа экологиялық қауіпсіз және
экономикалық тиімді технологияларының болуын анықтау, құрамдарын
жетілдіру немесе әзірлеу мақсатында зерттеулер мен пысықтаулар қажет.

Балық өңдеу кәсіпорындарының (майлыкөбіктімасса және соапсток) құрамында
майы бар қалдықтарын өңдеудің және пайдаланудың ықтимал бағыттарын
айқындау мақсатында осы уақытқа дейін жинақталған зерттеулер олардың
химиялық құрамын, сондай-ақ, липидтердің фракциялық және май қышқылдық
құрамдарын зерттеу нәтижелерін куәландырады.

Зерттеу нәтижелерін талдау майлыкөбіктімасса мен соапстоктың негізін су
(орта есеппен 35-тен 78\%-ға дейін), липидтер (орта есеппен 7-ден
56\%-ға дейін) және сабын (орта есеппен 7-ден 13\%-ға дейін) құрайтынын
көрсетті {[}2{]}. Сонымен қатар, бұл компоненттердің мазмұны өте кең
ауқымда өзгереді және өңделетін шикізатқа, тазартуға түсетін ағындардың
сипатына, сондай-ақ, тазарту қондырғыларының техникалық мүмкіндіктеріне
байланысты.

Майлы көбікті масса мен соапстокта бос май қышқылдарының едәуір
мөлшерінің болуы (липидтер құрамының 30\% дейін) және сабын аталған
объектілер мен олардың туындыларын үйкеліске қарсы композициялардың
майлау компоненті ретінде пайдалану мүмкіндігін көрсетеді. Сонымен
қатар, майлы көбікті масса және соапсток липидтерінің май қышқылдық
құрамының жоғары шексіздігі (полиқанықпаған май қышқылдарының қосындысы
шамамен 38\%) үйкеліс беттері арасында берік шекаралық қабаттардың пайда
болуына ықпал етуі мүмкін, бұл майлаудың тиімділігін едәуір арттырады
{[}1, 2, 7{]}.

Бұрғылаудың үнемі өсіп келе жатқан көлеміне байланысты мұнай-газ
өнеркәсібі бұрғылау ерітінділері үшін экологиялық таза майлау
материалдарының тапшылығын сезінуде. Соңғы жылдары арнайы мақсаттағы
қосалқы заттар тобынан бұрғылау ерітінділеріне арналған майлау
қоспалары, олар бұрын жіктелгендей, негізгі реагенттердің құрамына
сенімді түрде ауысады. Біріншіден, бұл көлбеу, қатты қисық және көлденең
ұңғымаларды бұрғылау үшін, құбырлар бағанының ұңғыманың қабырғаларына
үйкеліс күшін жеңу үшін энергия шығыны жоғары болғандықтан, осыған
байланысты әдебиетте мамандар бұрғылау ерітінділерінің майлау қабілетіне
үлкен мән береді.

Сонымен қатар, жоғарыда айтылғандай, бұрғылауда, әсіресе теңіз кен
орындарын игеру кезінде қолданылатын материалдардың экологиялық
қауіпсіздігінің жоғары талаптары табиғи заттар ‒ өсімдік майлары,
жануарлар майлары, құрамында майлары бар қалдықтар негізіндегі майлау
қоспаларына көбірек сәйкес келеді. Айта кету керек, экологиялық таза
майлаушы қоспаларын тұтыну көлемі қазіргі уақытта үнемі өсіп келеді және
оларды өндіру үшін шикізат базасын кеңейтуді талап етеді.

Құрамында бос май қышқылдары мен сабындары бар майлыкөбіктімасса мен
соапсток бұрғылау ерітіндісінің құрамындағы майлаушы компонент ретінде
пайдаланылуы мүмкін, бұл қосымша зерттеулердің тақырыбы болып табылады.

Бұрғылау ерітіндісінің сулы ортасында май қышқылдарының оңтайлы таралуы
үшін оларды сабын жасау үшін бейтараптандыру қажет. Май қышқылдарын
бейтараптандыру дәрежесі неғұрлым жоғары болса, олар ерітіндіде оңай
таралады (эмульсияланады), бірақ олардың тиімділігі соғұрлым төмен
болады, өйткені толық бейтараптандырылған май қышқылдарына негізделген
пленканың майлау қабілеті төмен. Композициядағы май құрамдас бөлігі мен
сабындандырғыш заттың оңтайлы арақатынасын анықтау керек, бұл оның
жоғары майлау қабілетін және сонымен бірге сулы ерітінділерде жеткілікті
эмульсиялануын қамтамасыз етеді.

Қазіргі уақытта М. Әуезов атындағы Оңтүстік Қазақстан университетінде
«Мұнай-газ өнеркәсібі қалдықтарынан мұнай-газ саласы үшін жаңа тиімді
материалдар алу технологиясын әзірлеу» тақырыбы бойынша мақсатты
қаржыландыру бағдарламасы мұнай-газ ұңғымаларын бұрғылау үшін бірқатар
өзекті мәселелерді шешуді көздейді.

Әртүрлі геологиялық жағдайларда қалдықтардың салыстырмалы тиімділігін
зерттеудің негізгі әдістемелік принциптері. Соңғы жылдары, әсіресе 20
жыл ішінде зерттеушілер айтарлықтай тәжірибе жинақтады және бұрғылау
ұңғымасының қабырғаларын құрайтын тау жыныстарының шиеленіскен күйінің
ерекшеліктерін ескере отырып, әртүрлі критерийлерді ұсынды {[}11, 12,
13, 14, 15{]}.

Ұңғымаларды бұрғылау, мұнай мен газ кен орындарын игеру және пайдалану,
әсіресе сейсмикалық белсенділіктің жоғарылауы, сондай-ақ,
балшық-вулкандық белсенділік жағдайында, көп жағдайда жойылатын тау
жыныстарының (бұзылу объектісі) физикалық-механикалық қасиеттерін,
геологиялық жағдайларды және онымен байланысты қиындықтарды зерттеуді
қажет етеді {[}12, 13, 14, 15{]}. Бұл деректерді тау жыныстары
массивтерінің кернеулі-деформацияланған күйіне байланысты есептеу
нәтижелерін нақтылау үшін бастапқы ақпарат ретінде пайдалану қажет және
өзекті болып табылады, бірақ қазіргі теориялық түсініктерге сүйенетін
нақты жағдайлар үшін есептеу әдістерін талдауды, дамытуды және
бейімдеуді қажет етеді {[}17, 18, 19, 20{]}.

Талдаулардың нәтижесі бойынша, қазіргі уақытта Қазақстанның мұнай
өнеркәсібі қарқынды дамып келеді. Мұнай мен газ өндіру өсуде, игеруге
және пайдалануға жаңа кен орындары енгізіліп, мұнай кен орындары
жабдықталуда, мұнай өндіру, жинау және дайындау қондырғылары өсуде,
мұнайды тасымалдау үшін кәсіпшілік және магистральдық құбырлардың
ұзындығы ұлғаюда. Сонымен қатар, экологиялық проблемалар да туындайды:
ағынды сулардың, бұрғылау қалдықтарының өсіп келе жатқан көлемін жою
қажеттілігі, бұрғылау ерітінділеріне жаңа экологиялық таза және
экономикалық тиімді қоспаларды іздеу және оларды зерттеу. Жабдықтарды
коррозиядан қорғау мәселесі де тұр. Тиімді қорғаныс құралдарын қолдану
жабдықтары мен коммуникациялардың қызмет ету мерзімін ұзартып қана
қоймай, олардың пайдалану сенімділігін арттырады, сондықтан қоршаған
ортаны мұнай, газ және ағынды сулардың апаттық ағып кетуінен қорғау
міндеттерін шешуге ықпал етеді.

Әдебиеттерге қысқаша шолу көрсеткендей, жаңа тиімділігі жоғары
химреагенттерді алу және құрылымдарында әртүрлі функционалды топтардың
болуына байланысты әртүрлі құрамдар мен қосылыстардың реологиялық,
құрылымдық-механикалық, ингибиторлық қасиеттерін терең іргелі ғылыми
зерттеулер жүргізіліп, көптеген процестердің тиісті механизмдері
құрылды.

Қазіргі уақытта металдардың коррозиясын тежеу теориясын жетілдіруге және
дамытуға, кешенді әсер ететін жаңа жоғары тиімді ингибиторларды іздеуге
және дамытуға, сондай-ақ, олардың әсер ету механизмін орнатуға арналған
көптеген зерттеулер жинақталды. Айта кету керек, іс жүзінде орын алған
көптеген проблемалар, тіпті импорттық химиялық реагенттерді қолданумен
де шешілмейді, бұл айтарлықтай қаржылық және материалдық шығындарға
әкеледі. Академиялық, салалық институттарда, жоғары оқу орындарында,
ғылыми-техникалық кешендерде жүргізілген ғылыми және техникалық
әзірлемелер екі, үш және одан да көп функциялары бар көп мақсатты
реагенттердің құрылуына әкелді. Жоғары полифункционалды қасиеттерге
қарамастан, көптеген жаңа реагенттер оларды алу процестерінің
күрделілігіне, мақсатты өнімдерді өндіруге жоғары қаржылық шығындарға
байланысты талап етілмейді және тек жекелеген реагенттер мен қалдықтар
бұрғылау ерітінділерін өндіруге арналған рецептуралар жасалатын негізгі
өнімдерге айналады.

{\bfseries Нәтижелер мен талқылау.} Жалпы, ғылыми және мерзімді
әдебиеттерге жасалған қысқаша шолу мұнай және газ өнеркәсібі үшін жаңа
химиялық өнімдерді әзірлеу саласындағы әртүрлі компаниялар мен ғылыми,
жобалау ұйымдарының қарқынды жұмысын көрсетеді. Бұған әртүрлі
компаниялардың, атап айтқанда «MI Дриллинг Флюидз К ЛТД», Baroid,
ресейлік компаниялардың тік және көлденең ұңғымаларды бұрғылауға және
аяқтауға арналған химиялық реагенттер мен сазды, сазсыз ерітінділердің
бірегей жүйелерін өндіруі дәлел болып табылады. Олардың кейбіреулері
бұрғылау жылдамдығын барынша арттыратын және қиындықтарды барынша
азайтатын ең жаңа бұрғылау ерітінділері жүйелерін әзірлеу саласындағы
көшбасшылардың бірі болып қала береді.

Соңғы жылдары жүргізілген зерттеулердің салыстырмалы талдауы бұрынғы
өнеркәсіптік реагенттерді тиімдірек алмастыратынын көрсетеді.

Атап өтілгендердің куәсі қазіргі уақытта М. Әуезов атындағы Оңтүстік
Қазақстан мемлекеттік университетінде орындалып жатқан «Мұнай-газ
өнеркәсібі қалдықтарынан мұнай-газ саласы үшін жаңа тиімді материалдар
алу технологиясын әзірлеу» тақырыбы бойынша мақсатты қаржыландыру
бағдарламасы, мұнай-газ ұңғымаларын бұрғылау үшін бірқатар өзекті
мәселелерді шешуді көздейді.

Алайда, осындай әр түрлі зерттеулер болған кезде, бұрғылау
ерітінділерінің ең жақсы құрамдарын таңдау тұрғысынан мұнай
өнеркәсібінің қажеттіліктерін толық қанағаттандыру әлі де қосымша
зерттеулерді қажет етеді, өйткені технологиялық, экономикалық,
экологиялық және географиялық себептерге байланысты қолжетімді
рецептуралар мен бұрғылау ерітінділеріне қоспаларды іздеу мәселесі
өзекті болып қала береді, бірқатар кешенді зерттеулерді жүргізуді талап
етеді. Олардың әртүрлілігіне байланысты бұл зерттеулер тиісті
әдіснамалық зерттеуді қажет етеді.

Осы зерттеулер ең алдымен бұрғылау ерітінділерінің негізгі құрамдарын
реттеудің ғылыми негіздерін, сондай-ақ оларды қолдану саласын негіздеуді
дамытуға және жетілдіруге бағытталуы тиіс. Олар сондай-ақ, зерттеу
процесінде туындайтын жаңа құбылыстарды білу, бұрын белгісіз
заңдылықтарды түсіндіру, бұрын жүргізілген зерттеулердің
жеткіліксіздігінің себептерін анықтау, қарастырылып отырған проблеманы
зерттеудегі олқылықтардың орнын толтыру және т.б. жолдарын зерттеу және
іздеу қажеттілігін қарастыруы керек. Бұрғылау ерітінділерінің
рецептураларын іздеумен және негіздеумен байланысты жаңа ғылыми
шешімдерді іздеу процесінде туындайтын қиындықтар қолданыстағы құрамдар,
олардың зерттелу дәрежесі, бұрын тұжырымдалған ғылыми ережелер, олардың
деңгейі, зерттеу процесінде қолданылатын әдістер жаңа мәселелерді шешу
үшін жеткіліксіз болған жағдайларда айқын көрінеді. Реагенттерді қолдану
үшін болжамды толық зерттеу және бұл міндеттерді қою бұрын алынған
тәжірибені талдаудан туындайды, осы уақытқа дейін жинақталған
зерттеулерде қайшылықтарды анықтау (егер олар бар болса), сондай-ақ,
проблеманың жекелеген бағыттарын одан әрі дамыту қажеттілігін негіздеу
болып табылады.

Кез-келген ғылыми зерттеу сияқты, бұрғылау процестерін зерттеу де қарау
және талдау процесінде басқалардың пайда болуына әкелетін проблеманы
ұсынудан басталады және бұл өз кезегінде барлық жаңа проблемаларды
тудырады. Біздің жағдайда мұндай проблема зерттеушілердің назарында,
бұрғылау ерітінділері проблемасы, олардың негізінде экономикалық және
экологиялық тұрғыдан тиімді реагенттер сәтті қолданыла алады.
Бұрғылаудың геологиялық-технологиялық жағдайларының алуан түрлілігіне,
олардың күрделілігінің әртүрлі дәрежесіне байланысты бұрғылау
ерітінділерін таңдау осы жағдайға сәйкес жүргізілуі керек, сонымен
қатар, реагенттерді сәтті жеткізуге мүмкіндік беретін бұрғылау жұмыстары
ауданының географиялық жағдайына сәйкес экономикалық және экологиялық
талаптарға жауап беруі керек. Ұңғымаларды бұрғылау тәжірибесі,
белгіленген талаптарды ескере отырып, әртүрлі бұрғылау ерітінділерін
қолдану нәтижелерін талдау маңызды.

Ғылыми зерттеулердің нәтижелері\emph{.} Қазақстан кен орындарында
ұңғымаларды бұрғылау нәтижелері мен шарттарын талдау және талқылау
негізінде одан әрі зерттеудің мақсатын айқындайтын, атап айтқанда, терең
ұңғымаларды бұрғылау үшін жергілікті сала қалдықтарынан шикізатты
қолдануға негізделген жаңа құрамдардың қасиеттерін әзірлеу және реттеу
мәселесіне назар аудару қажет.

Осы мақсатқа жетуге деген ұмтылыс - бұл шешілетін міндеттердің өзіндік
логикалық дәйектілігі, тиісті кезеңдері мен деңгейлері бар күрделі
процесс. Әдістемелік тұрғыдан осы зерттеулерді әртүрлі деңгейлерде
орналасқан элементтері бар тұтас жүйе аясында қарастыруға болады.

Осы жүйенің элементтері ретінде: зерттеу объектісі, зерттеу міндеттері,
оларды шешудің әдістері мен құралдары қызмет ете алады. Алдыңғы бөлімде
қарастырылған мәселеге арналған зерттеулерді талдау зерттеудің негізгі
міндеттерін тұжырымдауға мүмкіндік берді:

\begin{itemize}
\item
  жергілікті өнеркәсіптік қалдықтар негізінде бұрғылау ерітінділерінің
  ұңғымаларды бұрғылаудың техникалық-экономикалық көрсеткіштеріне әсерін
  зерделеудің қазіргі деңгейі мен жай-күйін талдау;
\item
  зерттелетін кен орнының және қолданылатын бұрғылау ерітінділерінің
  геологиялық жағдайларын талдау;
\item
  бұрғылау ерітінділерінің құрамы мен қасиеттері арасындағы байланысты
  зерттеу;
\item
  өнеркәсіптік қалдықтарға негізделген бұрғылау ерітінділерінің
  реологиялық сипаттамаларын зерттеу;
\item
  қолдану бойынша практикалық ұсыныстарды әзірлеу.
\end{itemize}

Ең қиыны - табиғи және техникалық-технологиялық факторларды ескере
отырып, қалыптан тыс жоғары қабат қысымының болуымен қиындаған
аймақтарда ұңғымаларды өткізудің оңтайлы технологиясын таңдау.

{\bfseries Қорытынды.} Қазақстанның күрделі жағдайларында ұңғымаларды
бұрғылаудың ағымдағы жай-күйін талдау көрсеткендей, ұңғымаларды бұрғылау
көрсеткіштерін арттыруға ұмтылуда үлкен рөл бұрғылау ерітінділерінің
құрамын таңдауға және жетілдіруге бағытталған әдістер жатады. Бұрғылау
ерітінділерінің құрамы мен параметрлерін таңдау және оларды тиімді
пайдалану шарттарын таңдау мәселелері бойынша ғылыми-техникалық және
патенттік әдебиеттердің деректерін жалпылама зерттеуді жүргізу кезіндегі
назардың аударылуы, тиісті құрамдар мен технологияларды анықтауға
мүмкіндік берді.

Әртүрлі мұнай-газ өндіретін өңірлерде қолданылатын технологиялардың
жоғары тиімділігін қамтамасыз етудің және оларды дамытуға инвестиция
салудың негізгі шарттарының бірі ықтимал қиындықтардың алдын алып қана
қоймай, сонымен қатар, ұңғыманың жылдамдығын жеткілікті жоғары деңгейде
ұстап тұруға мүмкіндік беретін үнемді және экологиялық қауіпсіз
реагенттерді қолдану болып табылады. Осыған байланысты бірқатар жұмыстар
қалдықтар негізінде композициялар жасауға арналған. Бұл жұмыстарға шолу
жергілікті өндірістік қалдықтар негізінде бұрғылау ерітінділерінің
құрамын жетілдіру бойынша зерттеулер жүргізу қажеттілігін негіздеді.

Сондықтан кейбір зерттеулер реологиялық сипаттамаларды зерттеу арқылы
осы құрамдардың тиімділігін бағалауға бағытталған.

Жалпы, күрделі жағдайларда қолдануға арналған бұрғылау ерітінділерінің
құрамын әзірлеуге байланысты мәселелерді зерттеудің қазіргі жағдайын
талдау келесілерді анықтауға мүмкіндік берді.

1. Зерттелетін қалдықтардың сипаты мен концентрациясына байланысты
бұрғылау ерітінділерінің құрамын кешенді зерттеуге мүмкіндік беретін
жүйенің негізін құру мәселесі жеткілікті түрде пысықталмаған. Осы
уақытқа дейін жинақталған жарияланымдарға шолу көрсеткендей, ұңғыма
тереңдігі метрінің құнының мүмкін болатын ең төменгі мәнімен ұңғыманың
жылдамдығын арттыруға бағытталған шешімдер қабылдау процесінің
критерийлердің түсініксіздігі, дәлсіздіктер және толық емес кірістер
түрінде көрінетін әртүрлі белгісіздіктерге, сондай-ақ, деректерді өңдеу
қажеттілігіне байланысты айтарлықтай күрделене түседі. Сондықтан талдау
үшін қажетті ақпарат заманауи математикалық әдістерді қолдана отырып,
оның сапасын арттыру үшін зерттеуді қажет етеді.

2. Ең дұрыс және негізделген технологиялық шешімдерді қабылдау үшін:
геологиялық-технологиялық және геофизикалық ақпаратты, қарастырылып
отырған кен орнының ерекшеліктерін және ұңғымаларды бұрғылаудың тиімді
технологиясын таңдауға әсер ететін факторларды талдау қажет.

Бұрғылаудың геологиялық шарттары әртүрлі факторлары бар күрделі жүйе
екені белгілі, осыған байланысты белгілі бір құрам мен технологияны
таңдау белгісіздік жағдайында шешім қабылдау процедурасын білдіреді.
Осыған байланысты эксперименттік зерттеулер жүргізу, ақпаратты алу,
талдау және белгіленген жағдайларда шешім қабылдау қызығушылық тудырады,
бұл зерттеу міндеттерін қоюды негіздейді.

Бұл зерттеулер Қазақстан Республикасы Ғылым және жоғары білім
Министрлігінің Ғылым және жоғары білім Комитетінің қолдауымен жүргізілді
(МҚ АР14869314).

{\bfseries Әдебиеттер}

1. Жантасов М.К., Орынбасаров А.К., Лю Цинь Цзе, Сапаров К. Определение
реологических параметров ингибирующего бурового раствора на основе
хлопкового гудрона. //Национальная ассоциация ученых (НАУ), Науки о
земле.- 2015.- № III(8).- С.145-147.

2.
\href{http://neft-gas.kz/f/no_5_2016_neft_i_gaz-dlya_sajta2-1.pdf}{Бондаренко
В.П., Надиров К.С., Бимбетова Г.Ж. Использование модифицированного
гудрона хлопкового масла для приготовления буровых растворов}//Журнал
«Нефть и газ».-2016.- № 5(95).- С.45-56.

3.Петров Б.Ф. Обоснование возможности использования жировых отходов
рыбоперерабатывающих производств в составе антифрикционной композиции //
Фундаментальные исследования.-2010.-№ 12.- С.136-141

https://fundamental-research.ru/ru/article/view?id=17445

4.Каменских С. В., Логачёв Ю. Л., Нор А.В, Уляшева Н.М., Фомин А.С.
Осложнения и аварии при строительстве нефтяных и газовых скважин. 2014,
УГТУ.- 231 с.

5.Осложнения и аварии при строительстве нефтяных и газовых скважин :
учеб. пособие / С. В. Каменских, Ю. Л. Логачёв, А. В. Нор, Н. М.
Уляшева, А. С. Фомин. - Ухта : УГТУ, 2014. -231 с.

6.Borivoje Pašić, Nediljka Gaurina-Međimurec, Davorin Matanović.
Wellbore instability: causes and consequences Nestabilnost kanala
bušotine: uzroci i posljedice// Rudarsko-geološko-naftni
zbornik.-2007.-Vol19.-P.87-98

7.Petroleum Related Rock Mechanics, 2nd edition/E. Fjaer, R.M. Holt,
P.~Horsrud, R. Risnes. -The Netherlands, Amsterdam: Elsevier.// -2008,
Vol. 53, ISBN~9780444502605

8.\href{javascript:;}{Richard Plumb}, \href{javascript:;}{Stephen
Edwards}, \href{javascript:;}{Gary Pidcock}, \href{javascript:;}{Donald
Lee}, \href{javascript:;}{Brian Stacey} The Mechanical Earth Model
Concept and its Application to~High-Risk Well Construction
Projects//IADC/SPE Drilling Conference.-2000.
\href{https://doi.org/10.2118/59128-MS}{DOI 10.2118/59128-MS}

9. Bichakshan Borah, Borkha Mech Das. A review on applications of
bio-products employed in drilling fluids to minimize environmental
footprint// Environmental Challenges.-2022.- Vol. 6, DOI
10.1016/j.envc.2021.100411

10.Wajheeuddin M., Hossain M.E. Development of an
Environmentally-Friendly Water-Based Mud System Using Natural
Materials//
\href{https://www.researchgate.net/journal/Arabian-Journal-for-Science-and-Engineering-2191-4281?_tp=eyJjb250ZXh0Ijp7ImZpcnN0UGFnZSI6InB1YmxpY2F0aW9uIiwicGFnZSI6InB1YmxpY2F0aW9uIn19}{Arabian
Journal for Science and Engineering}.-2017.-Vol.43(6) DOI
\href{https://doi.org/10.1007/s13369-017-2583-2}{10.1007/s13369-017-2583-2}

11.\href{javascript:;}{Andrianov} V.,\href{javascript:;}{Solovyanchik}
V.,\href{javascript:;}{Aleshkov} V., \href{javascript:;}{Akhmetov} M.,
\href{javascript:;}{Kostin} S.,\href{javascript:;}{Mazaev} K. Extended
Reach Exploratory well Successfully Drilled on D-41 Structure of Baltic
Sea Shelf (Russian Sector)// SPE Arctic and Extreme Environments
Technical Conference and Exhibition.-2013.- Paper
Number:~SPE-166918-MS{\bfseries .}
\href{https://doi.org/10.2118/166918-MS}{DOI 10.2118/166918-MS}

12.Б.А. Растегаев Современный подход к проектированию ингибирующих
свойств буровых растворов для проводки скважин в сложных
геолого-технических условиях // Территория нефтегаз.-2009.- № 6.- С.
34-39.
\url{https://cyberleninka.ru/article/n/sovremennyy-podhod-k-proektirovaniyu-ingibiruyuschih-svoystv-burovyh-rastvorov-dlya-provodki-skvazhin-v-slozhnyh-geologo-tehnicheskih}

13.Растегаев, Б.А. Обеспечение устойчивости глинистых отложений в
искривлённых (горизонтальных) скважинах / Б. А. Растегаев, В. Н.
Гнибидин, О. В. Ножкина {[}и др.{]}: // SPE - 171286-RU.

14.Растегаев, Б.А. Физико-химические и геомеханические принципы
устойчивости глинистых отложений в пологих скважинах (на примере
Мухановского месторождения) / Б. А. Растегаев, А. В. Ульшин, М. С.
Гвоздь, О. В. Ножкина {[}и др.{]}: Сб. трудов XX научно-практической
конференции «Реагенты и материалы для строительства, эксплуатации и
ремонта нефтяных, газовых и газоконденсатных скважин: производство,
свойства и опыт применения». - Владимир: Аркаим, 2016. - С. 141-150.

15.Свинцицкий С.Б. Прогнозирование горно-геологических условий проводки
скважин в соленосных и глинистых отложениях с аномально высокими
давлениями флюидов : диссертация ... доктора геолого-минералогических
наук : 25.00.12 / Свинцицкий Святослав Брониславович; {[}Место защиты:
ГОУВПО "Северо-Кавказский государственный технический университет"{]}.-
Ставрополь, 2007.- 210 с.

16. \href{javascript:;}{Nygaard}, R., \href{javascript:;}{Hareland} G.
Prediction of directional changes in well drilling based on formation
rock strength//The 42nd U.S. Rock Mechanics Symposium (USRMS),San
Francisco -2008.- Paper Number:~ARMA-08-230.

17.Plumb, R. The mechanical earth model concept and its application to
high-risk well construction projects / R. Plumb, S. Edwards, G. Pidcock
// IADC/SPE 59128 paper presented at the IADC/SPE drilling conference
(23--25 February 2000, New Orleans). New Orleans, 2000. - 13 p.
\href{https://doi.org/10.2118/59128-MS}{DOI 10.2118/59128-MS}

18.Каменев П.А. Исследование геомеханических параметров массивов
осадочных пород Сахалина на основе данных каротажа и бурения.
Канд.дисс., Новосибирск.// -2016, C.160.

19.Bambang P. Istadi, Handoko T. Wibowo, Edy Sunardi, Soffian Hadi and
Nurrochmat Sawolo. Mud Volcano and Its Evolution, Earth Sciences.2012.-
DOI 10.5772/24944

20. Эфендиев Г.М., Маммадов В.Н. Статистический анализ влияния грязевых
вулканов на показатели бурения скважин и частоту осложнений// Тр.
Института геологии НАН Азерб. -2010.- № 36.- C. 52-58.

{\bfseries References}

1. Zhantasov M.K., Orynbasarov A.K., Lju Cin'{} Cze,
Saparov K. Opredelenie reologicheskih parametrov ingibirujushhego
burovogo rastvora na osnove hlopkovogo gudrona.
//Nacional' naja associacija uchenyh (NAU), Nauki o
zemle.- 2015.- № III(8).- S.145-147.{[}in Russian{]}

2. Bondarenko V.P., Nadirov K.S., Bimbetova G.Zh.
Ispol' zovanie modificirovannogo gudrona hlopkovogo masla
dlja prigotovlenija burovyh rastvorov//Zhurnal «Neft'{} i
gaz».-2016.- № 5(95).- S.45-56.{[}in Russian{]}

3.Petrov B.F. Obosnovanie vozmozhnosti ispol' zovanija
zhirovyh othodov rybopererabatyvajushhih proizvodstv v sostave
antifrikcionnoj kompozicii // Fundamental' nye
issledovanija.-2010.-№ 12.- S.136-141.{[}in Russian{]}

https://fundamental-research.ru/ru/article/view?id=17445

4.Kamenskih S. V., Logachjov Ju. L., Nor A.V, Uljasheva N.M., Fomin A.S.
Oslozhnenija i avarii pri stroitel' stve neftjanyh i
gazovyh skvazhin. 2014, UGTU.- 231 s. .{[}in Russian{]}

5.Oslozhnenija i avarii pri stroitel' stve neftjanyh i
gazovyh skvazhin : ucheb. posobie / S. V. Kamenskih, Ju. L. Logachjov,
A. V. Nor, N. M. Uljasheva, A. S. Fomin. - Uhta : UGTU, 2014. -231 s.
{[}in Russian{]}

6.Borivoje Pašić, Nediljka Gaurina-Međimurec, Davorin Matanović.
Wellbore instability: causes and consequences Nestabilnost kanala
bušotine: uzroci i posljedice// Rudarsko-geološko-naftni
zbornik.-2007.-Vol19.-P.87-98

7.Petroleum Related Rock Mechanics, 2nd edition/E. Fjaer, R.M. Holt,
P.~Horsrud, R. Risnes. -The Netherlands, Amsterdam: Elsevier.// -2008,
Vol. 53, ISBN~9780444502605

8.\href{javascript:;}{Richard Plumb}, \href{javascript:;}{Stephen
Edwards}, \href{javascript:;}{Gary Pidcock}, \href{javascript:;}{Donald
Lee}, \href{javascript:;}{Brian Stacey} The Mechanical Earth Model
Concept and its Application to~High-Risk Well Construction
Projects//IADC/SPE Drilling Conference.-2000.
\href{https://doi.org/10.2118/59128-MS}{DOI 10.2118/59128-MS}

9. Bichakshan Borah, Borkha Mech Das. A review on applications of
bio-products employed in drilling fluids to minimize environmental
footprint// Environmental Challenges.-2022.- Vol. 6, DOI
10.1016/j.envc.2021.100411

10.Wajheeuddin M., Hossain M.E. Development of an
Environmentally-Friendly Water-Based Mud System Using Natural
Materials//
\href{https://www.researchgate.net/journal/Arabian-Journal-for-Science-and-Engineering-2191-4281?_tp=eyJjb250ZXh0Ijp7ImZpcnN0UGFnZSI6InB1YmxpY2F0aW9uIiwicGFnZSI6InB1YmxpY2F0aW9uIn19}{Arabian
Journal for Science and Engineering}.-2017.-Vol.43(6) DOI
\href{https://doi.org/10.1007/s13369-017-2583-2}{10.1007/s13369-017-2583-2}

11.\href{javascript:;}{Andrianov} V.,\href{javascript:;}{Solovyanchik}
V.,\href{javascript:;}{Aleshkov} V., \href{javascript:;}{Akhmetov} M.,
\href{javascript:;}{Kostin} S.,\href{javascript:;}{Mazaev} K. Extended
Reach Exploratory well Successfully Drilled on D-41 Structure of Baltic
Sea Shelf (Russian Sector)// SPE Arctic and Extreme Environments
Technical Conference and Exhibition.-2013.- Paper
Number:~SPE-166918-MS{\bfseries .}
\href{https://doi.org/10.2118/166918-MS}{DOI 10.2118/166918-MS}

12.B.A. Rastegaev Sovremennyj podhod k proektirovaniju ingibirujushhih
svojstv burovyh rastvorov dlja provodki skvazhin v slozhnyh
geologo-tehnicheskih uslovijah // Territorija neftegaz.-2009.- № 6.- S.
34-39.
\url{https://cyberleninka.ru/article/n/sovremennyy-podhod-k-proektirovaniyu-ingibiruyuschih-svoystv-burovyh-rastvorov-dlya-provodki-skvazhin-v-slozhnyh-geologo-tehnicheskih}.
{[}in Russian{]}

13.Rastegaev, B.A. Obespechenie ustojchivosti glinistyh otlozhenij v
iskrivljonnyh (gorizontal' nyh) skvazhinah / B. A.
Rastegaev, V. N. Gnibidin, O. V. Nozhkina {[}i dr.{]}: // SPE -
171286-RU. {[}in Russian{]}

14.Rastegaev, B.A. Fiziko-himicheskie i geomehanicheskie principy
ustojchivosti glinistyh otlozhenij v pologih skvazhinah (na primere
Muhanovskogo mestorozhdenija) / B. A. Rastegaev, A. V.
Ul' shin, M. S. Gvozd', O. V. Nozhkina
{[}i dr.{]}: Sb. trudov XX nauchno-prakticheskoj konferencii «Reagenty i
materialy dlja stroitel' stva, jekspluatacii i remonta
neftjanyh, gazovyh i gazokondensatnyh skvazhin: proizvodstvo, svojstva i
opyt primenenija». - Vladimir: Arkaim, 2016. - S. 141-150. {[}in
Russian{]}

15.Svincickij S.B. Prognozirovanie gorno-geologicheskih uslovij provodki
skvazhin v solenosnyh i glinistyh otlozhenijah s
anomal' no vysokimi davlenijami fljuidov : dissertacija
... doktora geologo-mineralogicheskih nauk : 25.00.12 / Svincickij
Svjatoslav Bronislavovich; {[}Mesto zashhity: GOUVPO "Severo-Kavkazskij
gosudarstvennyj tehnicheskij universitet"{]}.-
Stavropol', 2007.- 210 s. {[}in Russian{]}

16. \href{javascript:;}{Nygaard}, R., \href{javascript:;}{Hareland} G.
Prediction of directional changes in well drilling based on formation
rock strength//The 42nd U.S. Rock Mechanics Symposium (USRMS),San
Francisco -2008.- Paper Number:~ARMA-08-230.

17.Plumb, R. The mechanical earth model concept and its application to
high-risk well construction projects / R. Plumb, S. Edwards, G. Pidcock
// IADC/SPE 59128 paper presented at the IADC/SPE drilling conference
(23--25 February 2000, New Orleans). New Orleans, 2000. - 13 p.
\href{https://doi.org/10.2118/59128-MS}{DOI 10.2118/59128-MS}.

18.Kamenev P.A. Issledovanie geomehanicheskih parametrov massivov
osadochnyh porod Sahalina na osnove dannyh karotazha i burenija.
Kand.diss., Novosibirsk.// -2016, C.160. {[}in Russian{]}

19.Bambang P. Istadi, Handoko T. Wibowo, Edy Sunardi, Soffian Hadi and
Nurrochmat Sawolo. Mud Volcano and Its Evolution, Earth Sciences.2012.-
DOI 10.5772/24944

20. Jefendiev G.M., Mammadov V.N. Statisticheskij analiz vlijanija
grjazevyh vulkanov na pokazateli burenija skvazhin i chastotu
oslozhnenij// Tr. Instituta geologii NAN Azerb. -2010.- № 36.- C. 52-58.
{[}in Russian{]}

\emph{{\bfseries Авторлар туралы мәліметтер}}

Ерменов С.М. {\bfseries -} PhD докторант, М.Әуезов атындағы Оңтүстік
Қазақстан университеті, Шымкент, Қазақстан, e-mail:
\href{mailto:\%20sapmax80@mail.ru}{sapmax80@mail.ru};

Эфендиев Г.М.- техника ғылымдарының докторы, профессор, Әзірбайжан
Республикасының Мұнай және газ институты, Әзірбайжан, e-mail:
galib\_2000@yahoo.com;

Садырбаева А.С. - техника ғылымдарының кандидаты, қауымдастырылған
профессор, М.Әуезов атындағы Оңтүстік Қазақстан университеті, Шымкент,
Қазақстан, e-mail: a.sadyrbaeva@mail.ru;

Жантасов М.К. {\bfseries -} техника ғылымдарының кандидаты,
қауымдастырылған профессор, кафедра меңгерушісі, М.Әуезов атындағы
Оңтүстік Қазақстан университеті, Шымкент, Қазақстан, e-mail:
\href{mailto:manapjan_80@mail.ru}{\nolinkurl{manapjan\_80@mail.ru}};

Байботаева С.Е. {\bfseries -} PhD докторы, доцент, М.Әуезов атындағы
Оңтүстік Қазақстан университеті, Шымкент, Қазақстан, e-mail:
sbaibotaeva@mail.ru;

\emph{{\bfseries Information about the authors}}

Yermenov S.M. - PhD doctoral, M.Auezov South Kazakhstan University,
Shymkent, Kazakhstan,e-mail: sapmax80@mail.ru;

Afandiyev G. M.-Doctor of technical sciences, professor, Institute of
oil and gas of the Republic of Azerbaijan, Azerbaijan, e-mail:
galib\_2000@yahoo.com;

Sadyrbayeva A.S. - Candidate of Technical Sciences, Associate Professor,
Head of the Department. M.Auezov South Kazakhstan University, Shymkent,
Kazakhstan,e-mail: a.sadyrbaeva@mail.ru;

Zhantasov M. K. - Candidate of Technical Sciences, Associate Professor,
Head of the Department. M.Auezov South Kazakhstan University, Shymkent,
Kazakhstan, e-mail:
\href{mailto:manapjan_80@mail.ru}{\nolinkurl{manapjan\_80@mail.ru}};

Baibotaeva S.E. - PhD, associate professor, M.Auezov South Kazakhstan
University, Shymkent, Kazakhstan,e-mail: sbaibotaeva@mail.ru.